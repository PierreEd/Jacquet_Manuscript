% !TeX encoding = UTF-8
% !TeX spellcheck = fr_FR
% !TeX root = mythesis.tex

% \chapter*{Appendix 2:  }\label{app:spectra}

\chapter*{Appendix A: Two-photon derivations}\label{app:App}
\addcontentsline{toc}{chapter}{Appendix A: Two-photon derivations} % ensures it appears in the main TOC


\section*{Field Quantization}
\addcontentsline{toc}{section}{Field Quantization}
\subsubsection*{From discrete to continuous modes}

We consider the quantised electromagnetic field in a volume $V$ along a single polarization direction. We assume the field to be a gaussian beam such that the quantization volume is written as $\mathcal{V} = \mathcal{A}L$, with $\mathcal{A} $ the effective mode cross-sectional area, normal to the propagation direction $z$. The electric field operator can be written as  
\begin{equation}
\hat{\mathbf{E}}(\mathbf{r}, t) 
= i \sum_{ \ell } \sqrt{\dfrac{\hbar \omega_{\ell}}{2 \varepsilon_0 \mathcal{V}}} \,
\left[ \hat{a}^{\vphantom{\dagger}}_{ \omega_\ell }\,\mathbf{f}_{\ell }(\mathbf{r})\,e^{-i\omega_{\ell} t} 
- \hat{a}_{\omega_\ell  }^\dagger\,\mathbf{f}_{\ell }^*(\mathbf{r})\,e^{+i\omega_{\ell} t} \right],
\end{equation}
The index $\ell$ then labels the different modes, discrete at this point. The bosonic operators satisfy the canonical commutation relations
\[
[\hat{a}_{\omega_{\ell\vphantom{'}}}^{\vphantom{\dagger}}, \hat{a}_{\omega_{\ell'}}^\dagger] = \delta_{\ell \ell'} \,, \quad
[\hat{a}_{\omega_\ell}^{\vphantom{\dagger}}, \hat{a}_{\omega_{\ell'}}^{\vphantom{\dagger}} ] = [\hat{a}_{\omega_\ell}^\dagger, \hat{a}_{\omega_\ell'}^\dagger] = 0.
\]
We consider a the polarization along the $\hat{\mathbf{x}}$ direction where the hat denotes the unit vector and not an operator. The mode function can then be written as $\mathbf{f}_{\ell }(\mathbf{r}) = \text{f}_{\ell}(\mathbf{r})\hat{\mathbf{x}}$. We consider 1D wavevectors along the $+z$ direction i.e. positive $k_\ell$ only, such that in the limit of quantization volumes tending to infinity i.e. $L \to \infty$, the discrete sum over k modes turns into an integral over frequencies
\[
\sum_{\ell} (...) \quad \to \quad  \frac{L}{2\pi} \int^{\infty}_0 dk (...) = \frac{1}{\Delta f} \int^{\infty}_0 \dfrac{d\omega}{2\pi} (...) \quad \text{with} \quad \Delta f = \dfrac{c}{L}
\]
using the dispersion relation $\omega = c |k|$. We then simply relabel $\hat a_{\omega_\ell} \to \hat a_\omega$,  $\text{f}_{\ell}^{\vphantom{*}}(\mathbf{r})\to \text{f}^{\vphantom{*}}(\mathbf{r}, \omega)$ and plug back into the original expression to have
\[
\hat{\mathbf{E}}(\mathbf{r}, t)
= i \int^{\infty}_0 \dfrac{d\omega}{2\pi} \sqrt{\dfrac{\hbar \omega}{2 \varepsilon_0 \mathcal{A} c}} \,
\left[ \lim_{L \to \infty}\dfrac{\hat a^{\vphantom{\dagger}}[\Omega]}{\sqrt{\Delta f}}\,\text{f}^{\vphantom{*}}(\mathbf{r}, \omega)\,e^{-i\omega t}
-  \lim_{L \to \infty}\dfrac{\hat a^{\dagger}[\Omega]}{\sqrt{\Delta f}}\,\text{f}^{*}(\mathbf{r}, \omega)\,e^{+i\omega t} \right] \hat{\mathbf{x}}.
\]
and we can define the continuous bosonic operators as
\[\hat{a}[\omega] =  \lim_{L \to \infty}\dfrac{\hat{a}^{\vphantom{\dagger}}_{ \omega}}{\sqrt{\Delta f}} \quad \text{and} \quad \hat{a}^\dagger[\omega] =  \lim_{L \to \infty} \dfrac{\hat{a}^{\dagger}_{ \omega}}{\sqrt{\Delta f}}\]
such that the electric field operator reads
\[
\hat{\mathbf{E}}(\mathbf{r}, t) = i \int^{\infty}_0 \dfrac{d\omega}{2\pi} \mathcal{E} \,\left[ \hat{a}^{\vphantom{\dagger}}[\omega]\,\text{f}^{\vphantom{*}}(\mathbf{r}, \omega)\,e^{-i\omega t}
- \hat{a}^\dagger[\omega]\,\text{f}^{*}(\mathbf{r}, \omega)\,e^{+i\omega t} \right] \hat{\mathbf{x}}, \quad \text{with} \quad \mathcal{E} = \sqrt{\dfrac{\hbar \omega}{2 \varepsilon_0 \mathcal{A} c}}.
\]

\subsubsection*{Commutation relations }
Using standard complex analysis techniques, the kronecker delta can be expressed as
\[\delta_{\ell \ell'} =  \int^{+\pi}_{-\pi} dt \, \dfrac{e^{i(\ell - \ell') t}}{2\pi}. \]
Upon the aforementioned assumptions, we can introduce the frequency spacing $\Delta \omega = 2\pi \Delta f$ such that the discrete angular frequencies are written as $\omega_\ell = \ell \Delta \omega$. It then follows that $\ell - \ell' = (\omega_\ell - \omega_{\ell'})/\Delta \omega$. By changing the variable of integration from $t$ to $t' = t / \Delta \omega$, we can rewrite the kronecker delta as
\[\delta_{\ell \ell'} =  \int^{+L/2c}_{-L/2c} dt' \Delta f\, e^{i(\omega_\ell - \omega_{\ell'}) t'}.\]
We can then see that in the limit of $L \to \infty$ i.e. $\Delta \omega \to 0$, the integral limits tend to infinity and the kronecker delta turns into a dirac delta such that
\[ \lim_{L \to \infty} \dfrac{\delta_{\ell \ell'}}{\Delta f} =  \int^{+\infty}_{-\infty} dt'  \, e^{i(\omega- \omega') t'}= 2\pi \delta(\omega - \omega').
\]
where we relabeled $\omega_\ell \to \omega$ and $\omega_{\ell'} \to \omega'$. The commutation relations for the continuous bosonic operators then read
which satisfy the commutation relations
\[
[\hat{a}[\omega], \hat{a}^\dagger[\omega']] = \lim_{L \to \infty} \dfrac{[\hat{a}_{\omega_{\ell\vphantom{'}}}^{\vphantom{\dagger}}, \hat{a}_{\omega_{\ell'}}^\dagger] }{\Delta f} =  2\pi\delta(\omega - \omega'), \quad [\hat{a}[\omega], \hat{a}[\omega']] = [\hat{a}^\dagger[\omega], \hat{a}^\dagger[\omega']] = 0.
\] 



\section*{Two photon formalism}
\addcontentsline{toc}{section}{Two photon formalism}
\subsubsection{Quadratures}

We will now consider mode field frequencies $\omega = \omega_0 + \Omega$ around a carrier frequency $\omega_0$, such that the integral term becomes
\[
\int^{\infty}_0 \dfrac{d\omega}{2\pi}(...) \quad \to \quad \int_{-\omega_0}^{\infty} \dfrac{d\Omega}{2\pi}(...) \sim \int_{-B}^{B} \dfrac{d\Omega}{2\pi}(...) \sim \int_{-\infty}^{\infty} \dfrac{d\Omega}{2\pi}(...)
\]
where $B$ is the detection bandwidth, which is always much smaller than the optical frequency $\omega_0$. We can then safely extend the integral limits to infinity. Assuming that the mode function $\text{f}(\mathbf{r}, \omega)$ does not vary significantly over the bandwidth $B$, we can approximate it by its value at the carrier frequency $\text{f}(\mathbf{r}, \omega_0) \equiv \text{f}(\mathbf{r})$. Pulling out this term from the integral, one can then project the electric field operator onto both the proper polarization axis and this mode function such that the electric field operator becomes spatially independent and reads
\begin{equation}
  \begin{split}
\hat{E}(t) = i  \mathcal{E}_0 \int^{\infty}_0 \dfrac{d\Omega}{2\pi} \, \Big[&\hat{a}_+^{\vphantom{\dagger}}\,e^{-i(\omega_0 + \Omega )t}
- \hat{a}_+^{\dagger}\,e^{+i(\omega_0 + \Omega )t} \\
& + \hat{a}_-^{\vphantom{\dagger}}\,e^{-i(\omega_0 - \Omega )t}
- \hat{a}^{\dagger}_-\,e^{+i(\omega_0 - \Omega )t} \Big]
\end{split}
\end{equation}
with $\mathcal{E}_0 = \sqrt{\hbar \omega/2 \varepsilon_0 \mathcal{A} c}$, and where we additionally split the integral term in two, introducing negative and positive sideband frequencies whose annihilation and creation operators are written as
\[\hat{a}^{\vphantom{\dagger}}_{ \pm} \equiv c_\pm\hat{a}^{\vphantom{\dagger}}[\omega_0 \pm \Omega] \quad \text{and} \quad \hat{a}^\dagger_{ \pm} \equiv c_\pm \hat{a}^\dagger[\omega_0 \pm \Omega]\quad \text{with}\quad  c_\pm = \sqrt{\dfrac{\omega_0 \pm \Omega}{\omega_0}}.\]
The commutators then read
\[[\hat{a}_{\pm}^{\vphantom{\dagger}}, \hat{a}_{\pm}^\dagger] = 2 \pi c_\pm^2 \delta(\Omega - \Omega') \,, \quad
[\hat{a}_{\pm}^{\vphantom{\dagger}}, \hat{a}_{\pm}^{\vphantom{\dagger}} ] = [\hat{a}_{\pm}^\dagger, \hat{a}_{\pm}^\dagger] = 0\]
\[[\hat{a}_{\pm}^{\vphantom{\dagger}}, \hat{a}_{\mp}^\dagger] = 2 \pi c_+ c_- \delta(\Omega + \Omega') \,, \quad
[\hat{a}_{\pm}^{\vphantom{\dagger}}, \hat{a}_{\mp}^{\vphantom{\dagger}} ] = [\hat{a}_{\pm}^\dagger, \hat{a}_{\mp}^\dagger] = 0\]

Computing expectation values for these operators in vacuum yields $\langle \hat{a}^{\vphantom{\dagger}}_{ \pm} \rangle = \langle \hat{a}^{\dagger}_{ \pm}\rangle = \langle 0 | \hat{a}^{\dagger}_{\pm} \hat{a}^{\vphantom{\dagger}}_{\pm} | 0 \rangle = 0$ and $\langle 0 | \hat{a}^{\vphantom{\dagger}}_{\pm} \hat{a}^{\dagger}_{\pm} | 0 \rangle = 2 \pi c_\pm^2 \delta(0)$, which is consistent with the fact that no photons are present in these modes. We then regroup the terms along common quadratures $\cos \omega_0 t$ and $\sin \omega_0 t$ such that we get
\begin{equation*}
  \begin{split}
   \hat{E}(t) = i  \mathcal{E}_0 \bigg[ \cos \omega_0 t  \int^{\infty}_0 \dfrac{d\Omega}{2\pi} \, \Big[& (\hat{a}_+^{\vphantom{\dagger}} - \, \hat{a}_-^{\dagger})\,e^{-i\Omega t} + (\hat{a}_-^{\vphantom{\dagger}} - \, \hat{a}_+^{\dagger})\,e^{+i \Omega t} \Big] \\ 
    - i   \sin \omega_0 t  \int^{\infty}_0 \dfrac{d\Omega}{2\pi} \, \Big[&(\hat{a}_+^{\vphantom{\dagger}} + \, \hat{a}_-^{\dagger})\,e^{-i\Omega t} + (\hat{a}_-^{\vphantom{\dagger}} + \, \hat{a}_+^{\dagger})\,e^{+i \Omega t} \Big] \bigg] 
  \end{split}
\end{equation*}
We now define the two-photon quadrature operators as
\[\hat{p}^{\vphantom{\dagger}}[\Omega] = \hat{a}_+^{\vphantom{\dagger}} + \hat{a}_-^{\dagger} \,, \quad \hat{q}^{\vphantom{\dagger}}[\Omega] = i(\hat{a}_-^{\dagger} - \hat{a}_+^{\vphantom{\dagger}} )\]
such that the electric field operator reads
\begin{equation}
  \begin{split}
   \hat{E}(t) = \mathcal{E}_0 \bigg[  \cos ( \omega_0 t - \dfrac{\pi}{2}) \int^{\infty}_0 \dfrac{d\Omega}{2\pi} \, \Big[& \hat{p}^{\vphantom{\dagger}}[\Omega] \,e^{-i\Omega t} +\hat{p}^{\dagger}[\Omega] \,e^{+i \Omega t} \Big] \\ 
    +  \sin ( \omega_0 t - \dfrac{\pi}{2})  \int^{\infty}_0 \dfrac{d\Omega}{2\pi} \, \Big[&\hat{q}^{\vphantom{\dagger}}[\Omega]\,e^{-i\Omega t} + \hat{q}^{\dagger}[\Omega]\,e^{+i \Omega t} \Big] \bigg] 
  \end{split}
\end{equation}
where we used the fact that $\hat{p}^\dagger[\Omega] = \hat{p}^{\vphantom{\dagger}}[-\Omega]$ and $\hat{q}^\dagger[\Omega] = \hat{q}^{\vphantom{\dagger}}[-\Omega]$. The $\pi/2$ phase shifts originate from the leading factor $i$ in the electric-field operator. Had the field operator been written without that prefactor (and without the minus sign in the creation-term), the resulting cosine and sine components would contain no such phase offset. The commutation relations for these quadrature operators read
\begin{equation*}
  \begin{split}
[\hat{p}^{\vphantom{\dagger}}[\Omega], \hat{q}^\dagger[\Omega']] & = [\hat{q}^{\vphantom{\dagger}}[\Omega], \hat{p}^\dagger[\Omega']] = 4 \pi i  \delta(\Omega - \Omega') \, \\ 
[\hat{p}^{\vphantom{\dagger}}[\Omega], \hat{p}^\dagger[\Omega']] & = [\hat{q}^{\vphantom{\dagger}}[\Omega], \hat{q}^\dagger[\Omega']] = 4 \pi  \dfrac{\Omega}{\omega_0}\delta(\Omega - \Omega') \sim 0 \,\quad \text{if} \quad  \Omega \ll \omega_0  \\
[\hat{p}^{\vphantom{\dagger}}[\Omega], \hat{q}^{\vphantom{\dagger}}[\Omega']] & = [\hat{p}^\dagger[\Omega], \hat{q}^\dagger[\Omega']] = 0 \,.
  \end{split}
\end{equation*}
In the limit where the sideband frequencies are small compared to the carrier frequency i.e. $\Omega \ll \omega_0$, we can approximate these prefactors by $c_\pm \sim 1$. 
\subsubsection*{Expectations values in vacuum}
We now proceed to evaluate the first and second momenta of our field operators in the vacuum state $|0\rangle$. As expected, the annihilation and creation operators have zero mean in vacuum, such that
\[\langle 0 | \hat{a}^{\vphantom{\dagger}}_{+} | 0 \rangle = \langle 0 | \hat{a}^\dagger_{-} | 0 \rangle = 0\]
so it follows that
\[\langle 0 | \hat{p}^{\vphantom{\dagger}}[\Omega] | 0 \rangle = \langle 0 | \hat{q}^{\vphantom{\dagger}}[\Omega] | 0 \rangle = 0. \]
Building the two-photon quadrature column vector as
\[\mathbf{\hat{u}}^{\vphantom{\dagger}}[\Omega] = \begin{pmatrix}
\hat{p}^{\vphantom{\dagger}}[\Omega] \\[4pt]
\hat{q}^{\vphantom{\dagger}}[\Omega] \\
\end{pmatrix}, \quad \text{we have} \quad \langle\mathbf{\hat{u}}^{\vphantom{\dagger}}[\Omega]\rangle = \begin{pmatrix}
0 \\[4pt]
0 \\
\end{pmatrix}\]
where we see that, for a vacuum state, the full operator $\mathbf{\hat{u}}^{\vphantom{\dagger}}[\Omega]$ actually equates the fluctuating part $\delta \mathbf{\hat{u}}^{\vphantom{\dagger}}[\Omega] = \mathbf{\hat{u}}^{\vphantom{\dagger}}[\Omega] - \langle \mathbf{\hat{u}}^{\vphantom{\dagger}}[\Omega] \rangle$ since the mean value is zero. In the following, we will assume that expectation values are always computed in the vacuum state unless otherwise specified (we will omit the $|0\rangle$ notation for clarity). 
We only wrote the results for the $\hat a^{\vphantom{\dagger}}_+$ and $\hat a_-^\dagger$ operators as there are the ones composing the $\hat p$ and $\hat q$ quadratures, but the same results hold for the other sideband operators as well.
We can then compute the second momenta of the annihilation and creation operators, yielding
\[\langle 0 | \hat{a}^{\dagger}_{-} \hat{a}^{\vphantom{\dagger}}_{-} | 0 \rangle = \langle 0 | \hat{a}^{\vphantom{\dagger}}_{+} \hat{a}^{\vphantom{\dagger}}_{-} | 0 \rangle = \langle 0 | \hat{a}^{\dagger}_{-} \hat{a}^{\dagger}_{+} | 0 \rangle = 0 \]
\[ \langle 0 | \hat{a}^{\vphantom{\dagger}}_{\pm} \hat{a}^{\dagger}_{\pm} | 0 \rangle = 2 \pi \delta(\Omega - \Omega') \,.\]
Using these relations, we can compute the second momenta for the two-photon quadrature operators as 
\begin{align*}
\langle 0 | \hat{p}^{\vphantom{\dagger}}[\Omega] \hat{p}^{\dagger}[\Omega'] | 0 \rangle & =  \langle 0 | \hat{a}^{\vphantom{\dagger}}_{+}\hat{a}^{\dagger}_{+} +  \hat{a}^{\vphantom{\dagger}}_{+} \hat{a}^{\vphantom{\dagger}}_{-} +   \hat{a}^{\dagger}_{-} \hat{a}^{\dagger}_{+}  + \hat{a}^{\dagger}_{-}\hat{a}^{\vphantom{\dagger}}_{-}|0 \rangle \\
& = 2 \pi \delta(\Omega - \Omega') \,, \\
\langle 0 | \hat{q}^{\vphantom{\dagger}}[\Omega] \hat{q}^{\dagger}[\Omega'] | 0 \rangle & = 2 \pi \delta(\Omega - \Omega') \,. \\
\end{align*}
as well as 
\[\langle 0 | \hat{p}^{\vphantom{\dagger}}[\Omega] \hat{q}^{\dagger}[\Omega'] | 0 \rangle = - \langle 0 | \hat{q}^{\dagger}[\Omega] \hat{p}^{\vphantom{\dagger}}[\Omega'] | 0 \rangle = i 2\pi \delta (\Omega - \Omega'). \]
Using the expression for the symmetrized double sided covariance matrix given in the main text, we can compute the covariance matrix for the two-photon quadrature operators in vacuum as
\begin{align*}
  \mathbf{S}[\Omega]& =  \dfrac{1}{2}  \int \dfrac{\delta \Omega'}{2\pi} \langle\{  \delta \mathbf{\hat{u}}{\vphantom{\dagger}}[\Omega]\, , \delta \mathbf{\hat{u}}^{\dagger}[\Omega'] \}  \rangle \\
  & = \dfrac{1}{2}  \int \dfrac{\delta \Omega'}{2\pi}  \begin{pmatrix}
  \langle \{ \hat{p}^{\vphantom{\dagger}}[\Omega]\,, \hat{p}^{\dagger}[\Omega'] \}\rangle & \langle \{ \hat{p}^{\vphantom{\dagger}}[\Omega]\,, \hat{q}^{\dagger}[\Omega'] \} \rangle \\[4pt]
  \langle \{ \hat{q}^{\vphantom{\dagger}}[\Omega]\,, \hat{p}^{\dagger}[\Omega'] \} \rangle & \langle \{ \hat{q}^{\vphantom{\dagger}}[\Omega]\,, \hat{q}^{\dagger}[\Omega'] \} \rangle \\
  \end{pmatrix} \\
  & = \dfrac{1}{2}  \int \dfrac{\delta \Omega'}{2\pi}  \begin{pmatrix}
  2 \cdot 2 \pi \delta(\Omega - \Omega') & 0 \\[4pt]
  0 & 2 \cdot 2 \pi \delta(\Omega - \Omega') \\
  \end{pmatrix} \\
  & = \begin{pmatrix}
  1 & 0 \\[4pt] 
  0 & 1 \\
  \end{pmatrix} = \mathbf{1}.
\end{align*}
The vacuum state then features vacuum fluctuations of unity in both quadratures, across all sideband frequencies $\Omega$, and no correlations between the quadratures. 

\subsubsection*{States and Operators in the Two-Photon Formalism}
In a similar fashion as in the single-mode case, we can define the displacement operator as
\[\hat{D}(\alpha) = \exp\left( \int^{\infty}_{-\infty} \dfrac{d\Omega}{2\pi} \, \left[ \alpha(\Omega) \hat{a}^\dagger_- - \alpha^*(\Omega) \hat{a}^{\vphantom{\dagger}}_+  \right] \right)\]
as well as a squeezing operator
\[\hat{S}(r, \theta) = \exp\left( r \int^{\infty}_{-\infty} \dfrac{d\Omega}{2\pi} \,\left[ e^{- i 2\theta(\Omega)} \hat{a}^{\vphantom{\dagger}}_+ \hat{a}^{\vphantom{\dagger}}_- - e^{ i 2\theta(\Omega)} \hat{a}^\dagger_+ \hat{a}^\dagger_- \right] \right)\]
where $r$ is the squeezing factor and $\theta(\Omega)$ the squeezing angle. Here we assumed the squeezing parameter to be frequency independent, but one can easily generalize to a frequency dependent squeezing parameter $r(\Omega)$. Using the sidebands annihilation operators defined previously, we can compute the action of the displacement and squeezing operators on the annihilation operator as
\begin{align*}
\hat{D}^\dagger(\alpha) \,\hat{a}^{\vphantom{\dagger}}_+ \,\hat{D}(\alpha) & = \hat{a}^{\vphantom{\dagger}}_+ + \alpha(\Omega) \,, \\
\hat{S}^\dagger(r, \theta) \,\hat{a}^{\vphantom{\dagger}}_+ \,\hat{S}(r, \theta) & = \hat{a}^{\vphantom{\dagger}}_+ \cosh r - e^{i 2 \theta(\Omega)} \hat{a}^\dagger_- \sinh r \,.
\end{align*} 
We consider a intial vacuum state $|0\rangle$, and we displace it by a coherent amplitude $\alpha(\Omega) = \alpha \delta(\Omega)$ i.e. a carrier, monochromatic field of complex amplitude $\alpha$ sitting at frequency 0 (we are in the frame rotating at $\omega_0$ already since we factored out the $e^{-i\omega_0t} $ term). The displacement operator then acts on the two photon quadrature operators as 
\[ \hat{D}^\dagger(\alpha) \,\hat{p}^{\vphantom{\dagger}}[\Omega] \, \hat{D}(\alpha) = \hat{p}^{\vphantom{\dagger}}[\Omega] + 2 \Re{\alpha} \delta(\Omega) \,, \]
\[ \hat{D}^\dagger(\alpha) \,\hat{q}^{\vphantom{\dagger}}[\Omega] \, \hat{D}(\alpha) = \hat{q}^{\vphantom{\dagger}}[\Omega] + 2 \Im{\alpha} \delta(\Omega) \,.\]
or in matrix form
\[\hat{D}^\dagger(\alpha) \,\mathbf{\hat{u}}[\Omega] \, \hat{D}(\alpha) = \mathbf{\hat{u}}[\Omega] + 2 \begin{pmatrix}
\Re{\alpha} \\[4pt]
\Im{\alpha} \\
\end{pmatrix} \delta(\Omega) \,.\]
In a similar fashion, the squeezing operator acts as
\[\hat{S}^\dagger(r, \theta) \,\hat{p}^{\vphantom{\dagger}}[\Omega] \, \hat{S}(r, \theta) = \hat{p}^{\vphantom{\dagger}}[\Omega] (\cosh r - \sinh r \cos 2\theta) - \hat{q}^{\vphantom{\dagger}}[\Omega] \, \sin 2\theta \sinh r \,,\]
\[\hat{S}^\dagger(r, \theta) \,\hat{q}^{\vphantom{\dagger}}[\Omega] \, \hat{S}(r, \theta) = \hat{q}^{\vphantom{\dagger}}[\Omega] (\cosh r + \sinh r \cos 2\theta) - \hat{p}^{\vphantom{\dagger}}[\Omega] \, \sin 2\theta \sinh r \,.\]
and its matrix form reads
\[\hat{S}^\dagger(r, \theta) \,\mathbf{\hat{u}}[\Omega] \, \hat{S}(r, \theta) = \mathbf{S}(r, \theta) \, \mathbf{\hat{u}}[\Omega] \,, \quad \text{with} \quad \mathbf{S}(r, \theta) = \begin{pmatrix}
\cosh r - \sinh r \cos 2\theta & - \sin 2\theta \sinh r \\[6pt]
- \sin 2\theta \sinh r & \cosh r + \sinh r \cos 2\theta \\
\end{pmatrix}.\]
The state resulting from applying both operators onto the vacuum is written as
\[|\psi\rangle = \hat{S}(r, \theta) \hat{D}(\alpha)  |0\rangle\]
and describes a squeezed coherent state, or bright squeezed state. One can then set the coherent amplitude to 0 as to get a vacuum squeezed state, or set the squeezing parameter to 0 to get a coherent state. This is one of the most generic gaussian state one can define in quantum optics.
We write the operator product as $\hat{D}\hat{S}$ and we drop the $\Omega$ dependencies to lighten the notation, such that applying them to the field operators yields 
\begin{align*}
\hat{D}^\dagger \hat{S}^\dagger  \,\hat{a}^{\vphantom{\dagger}}_+ \, \hat{S} \hat{D} & = \hat{a}^{\vphantom{\dagger}}_+ \cosh r - e^{i 2\theta} \hat{a}^\dagger_- \sinh r + \gamma \delta (\Omega) \, \\
\hat{D}^\dagger \hat{S}^\dagger  \,\hat{a}^{\dagger}_- \, \hat{S} \hat{D} & = \hat{a}^\dagger_- \cosh r - e^{-i 2\theta} \hat{a}^{\vphantom{\dagger}}_+ \sinh r + \gamma^* \delta (\Omega) \,.
\end{align*}
as well as the quadratures 
\begin{align*}
\hat{D}^\dagger \hat{S}^\dagger  \,\hat{p}^{\vphantom{\dagger}}\, \hat{S} \hat{D} & = \hat{p}^{\vphantom{\dagger}} (\cosh r - \cos 2\theta \sinh r) - \hat{q}^{\vphantom{\dagger}} \, \sin 2\theta \sinh r + 2 \Re{\gamma} \delta(\Omega) \,, \\
\hat{D}^\dagger \hat{S}^\dagger  \,\hat{q}^{\vphantom{\dagger}} \, \hat{S} \hat{D} & = \hat{q}^{\vphantom{\dagger}} (\cosh r + \cos 2\theta \sinh r) - \hat{p}^{\vphantom{\dagger}} \, \sin 2\theta \sinh r + 2 \Im{\gamma} \delta(\Omega) \,.
\end{align*}
where we introduced the scalar part of these transformed operators as 
\begin{align*}
  \gamma = \alpha \cosh r - \alpha^* e^{i 2\theta} \sinh r \,, \\
  \gamma^* = \alpha^* \cosh r - \alpha e^{-i 2\theta} \sinh r \,.
\end{align*}
The matrix form then reads
\[\hat{D}^\dagger \hat{S}^\dagger  \,\mathbf{\hat{u}} \, \hat{S} \hat{D} = \mathbf{S}(r, \theta) \, \mathbf{\hat{u}} + 2 \begin{pmatrix}
\Re{\gamma} \\[4pt]
\Im{\gamma} \\
\end{pmatrix} \delta(\Omega) \,.\]
The mean values is then straightforward to compute
\begin{align*}
  \langle \hat{D}^\dagger \hat{S}^\dagger  \,\mathbf{\hat{u}} \, \hat{S} \hat{D} \rangle & =  \mathbf{S}(r, \theta) \, \langle \mathbf{\hat{u}}\rangle  + 2 \begin{pmatrix}
\Re{\gamma} \\[4pt]
\Im{\gamma} \\
\end{pmatrix} \delta(\Omega)  \\
  & = 2 \begin{pmatrix}
\Re{\gamma} \\[4pt]
\Im{\gamma} \\
\end{pmatrix} \delta(\Omega) \,.
\end{align*} 
such that the fluctuating part reads
\[\delta \mathbf{\hat{u}}^{\vphantom{\dagger}} = \hat{D}^\dagger \hat{S}^\dagger  \,\mathbf{\hat{u}} \, \hat{S} \hat{D} - \langle \hat{D}^\dagger \hat{S}^\dagger  \,\mathbf{\hat{u}} \, \hat{S} \hat{D} \rangle = \mathbf{S}(r, \theta) \, \mathbf{\hat{u}}^{\vphantom{\dagger}} \, \quad \text{and} \, \quad \delta \mathbf{\hat{u}}^{\dagger} =  \mathbf{\hat{u}}^{\dagger}  \mathbf{S}(r, \theta) .\]
where we used the fact that the squeezing matrix is symmetric, i.e. $\mathbf{S} = \mathbf{S}^T$.
The covariance matrix for this squeezed coherent state then reads
\begin{align*}
  \mathbf{S}[\Omega] & =  \dfrac{1}{2}  \int \dfrac{\delta \Omega'}{2\pi} \langle\{  \delta \mathbf{\hat{u}}{\vphantom{\dagger}}[\Omega]\, , \delta \mathbf{\hat{u}}^{\dagger}[\Omega'] \}  \rangle \\
  & =  \dfrac{1}{2}  \int \dfrac{\delta \Omega'}{2\pi} \langle\{  \mathbf{S}(r, \theta) \, \mathbf{\hat{u}}^{\vphantom{\dagger}}[\Omega]\, ,  \mathbf{\hat{u}}^{\dagger}[\Omega']  \mathbf{S}(r, \theta) \}  \rangle \\
  & =  \mathbf{S}(r, \theta)  \left(  \dfrac{1}{2}  \int \dfrac{\delta \Omega'}{2\pi} \langle\{   \mathbf{\hat{u}}^{\vphantom{\dagger}}[\Omega]\, ,  \mathbf{\hat{u}}^{\dagger}[\Omega']  \}  \rangle \right)  \mathbf{S}(r, \theta)  \\
  & =  \mathbf{S}(r, \theta)  \cdot \mathbf{1}  \cdot  \mathbf{S}(r, \theta)  =  \mathbf{S}(r, \theta)^2 \\
  & = \begin{pmatrix}
  \cosh 2r - \sinh 2r \cos 2\theta & - \sin 2\theta \sinh 2r \\[6pt]
  - \sin 2\theta \sinh 2r & \cosh 2r + \sinh 2r \cos 2\theta \\
  \end{pmatrix} \,.
\end{align*}

such that the expectation values are computed as
\begin{align*}
\langle \hat{a}^{\vphantom{\dagger}}_{+} \rangle & = \gamma \delta(\Omega) \\
\langle \hat{a}^{\dagger}_{-} \rangle & = \gamma^* \delta(\Omega) \,\\ 
\langle \hat{p} \rangle & = 2 \Re{\gamma}\delta(\Omega) \\
\langle \hat{q} \rangle & = 2 \Im{\gamma} \delta(\Omega) \,.
\end{align*}
and we compute the expectation value of our two-photon annihilation operator as
\[\langle \hat{a}^{\vphantom{\dagger}}_{+} \rangle = \alpha \delta(\Omega) \, \quad  \text{and} \quad   \langle \hat{a}^{\dagger}_{-} \rangle = \alpha^* \delta(\Omega) \]
as well as their second momenta as 

The electric field operator finally reads
\begin{equation}
  \begin{split}
\hat{\mathbf{E}}(\mathbf{r}, t)
= i \sqrt{\dfrac{\hbar \omega_0}{\varepsilon_0 \mathcal{A} c}} &  \Bigg[ \,\int_{-\infty}^{\infty} \dfrac{d\Omega}{2\pi} \,\left[ \hat{a}^{\vphantom{\dagger}}_{ \Omega}\,e^{-i(\omega_0 + \Omega) t}
- \hat{a}^{\dagger}_{ \Omega}\,e^{+i(\omega_0 + \Omega) t} \right] \\ 
& \Bigg]
  \end{split}
\end{equation}

such that the classical part of the electric field reads


We start from the standard single-mode field quantization in terms of annihilation and creation operators \(\hat a\) and \(\hat a^\dagger\):
\[
\hat E(t) = \sqrt{\frac{\hbar \omega_0}{2 \varepsilon_0}} \left( \hat a e^{-i\omega_0 t} + \hat a^\dagger e^{i\omega_0 t} \right).
\]  
and we now make our bosonic operators time-dependent, \(\hat a \to \hat a(t)\), to account for sidebands around the carrier frequency \(\omega_0\). Using the Fourier transform convention
\[
\hat a(t) = \int_{-\infty}^{+\infty} \frac{d\Omega}{2\pi} \hat a[\Omega] e^{-i\Omega t},
\]
we rewrite the field operator as
\[
\hat E(t) = \sqrt{\frac{\hbar \omega_0}{2 \varepsilon_0}} \int_{-\infty}^{+\infty} \frac{d\Omega}{2\pi} \left( \hat a[\Omega] e^{-i(\omega_0 + \Omega) t} + \hat a^\dagger[\Omega] e^{i(\omega_0 + \Omega) t} \right).
\]  



\begin{align}
\hat p[\Omega]
&= 2\,|\alpha| \Big( \delta[\Omega] + \Re\{\varepsilon[\Omega]\} \Big)
  + \delta \hat p[\Omega] \,,
\\[4pt]
\hat p[\Omega]\;\hat p[\Omega']
&= 4|\alpha|^{2}\Big(
      \delta[\Omega]S[\Omega']
    + \delta[\Omega]\Re\{\varepsilon[\Omega']\}
    + \delta[\Omega']\Re\{\varepsilon[\Omega]\}
    + \Re\{\varepsilon[\Omega]\}\Re\{\varepsilon[\Omega']\}
  \Big)
  + \delta \hat p[\Omega]\;\delta \hat p[\Omega'] \,,
\\[6pt]
\big\langle \cdots \big\rangle
&= 4|\alpha|^{2}\Big(
      \delta(\Omega)\,\delta(\Omega')
    + \frac{\varepsilon}{2}\,\delta(\Omega)\,\delta(\Omega'-\Omega_m)
    + \frac{\varepsilon}{2}\,\delta(\Omega)\,\delta(\Omega'+\Omega_m)
\\[-2pt]&\hphantom{= 4|\alpha|^{2}\Big(}
    + \frac{\varepsilon}{2}\,\delta(\Omega')\,\delta(\Omega-\Omega_m)
    + \frac{\varepsilon}{2}\,\delta(\Omega')\,\delta(\Omega+\Omega_m)
\\[-2pt]&\hphantom{= 4|\alpha|^{2}\Big(}
    + \frac{\varepsilon^{2}}{4}\Big[
          \delta(\Omega-\Omega_m)\,\delta(\Omega'+\Omega_m)
        + \delta(\Omega-\Omega_m)\,\delta(\Omega'-\Omega_m)
\\[-2pt]&\hphantom{= 4|\alpha|^{2}\Big(+ \frac{\varepsilon^{2}}{4}\Big[}
        + \delta(\Omega+\Omega_m)\,\delta(\Omega'+\Omega_m)
        + \delta(\Omega+\Omega_m)\,\delta(\Omega'-\Omega_m)
      \Big]
  \Big)
  + \big\langle \delta p[\Omega]\;\delta p[\Omega'] \big\rangle \,.
\end{align}





\section*{Derivation of the optimal angle}
\addcontentsline{toc}{section}{Derivation of the optimal angle}
\subsection*{Optimal fixed homodyne angle with complex \texorpdfstring{$\mathcal{K}$}{K}}

Assume the measured (reflected) quadrature is
\[
\delta q_r \;=\; \delta q_{\rm in} \;+\; \mathcal{K}\,\delta p_{\rm in},
\]
so that, for any input covariance matrix \(S^{\rm in}\),
\[
S^{r}_{qq} \;=\; S^{\rm in}_{qq} + |\mathcal{K}|^{2}S^{\rm in}_{pp} + 2\,\mathrm{Re}\{\mathcal{K}\}\,S^{\rm in}_{pq}.
\]

For an input squeezed state of strength \(R\) and angle \(\theta\),
\[
S^{\rm in}(r,\theta)=
\begin{pmatrix}
\cosh 2r + \sinh 2r \cos 2\theta & \,-\sinh 2r \sin 2\theta\\[2pt]
\,-\sinh 2r \sin 2\theta & \cosh 2r - \sinh 2r \cos 2\theta
\end{pmatrix}.
\]
Hence
\begin{align}
S^{r}_{qq}(\theta)
&= \cosh 2r - \sinh 2r \cos 2\theta
 + |\mathcal{K}|^{2}\!\left(\cosh 2r + \sinh 2r \cos 2\theta\right)
 - 2\,\mathrm{Re}\{\mathcal{K}\}\,\sinh 2r \sin 2\theta \nonumber\\[2pt]
&= (1+|\mathcal{K}|^{2})\cosh 2r
  - (1-|\mathcal{K}|^{2})\sinh 2r \cos 2\theta
  - 2\,\mathrm{Re}\{\mathcal{K}\}\,\sinh 2r \sin 2\theta .
\label{eq:Sqq_linearform}
\end{align}

\paragraph{Optimal fixed angle.}
Differentiate \eqref{eq:Sqq_linearform} w.r.t.~\(\theta\) and set to zero:
\[
\frac{\partial S^{r}_{qq}}{\partial \theta}
= 2\sinh 2r\Big[(1-|\mathcal{K}|^{2})\sin 2\theta - 2\,\mathrm{Re}\{\mathcal{K}\}\cos 2\theta\Big]=0,
\]
which gives the optimal fixed readout angle
\begin{equation}
\tan(2\theta_{\rm opt})
=  \frac{2\,\mathrm{Re}\{\mathcal{K}\}}{\,1-|\mathcal{K}|^{2}\,}
\qquad
\label{eq:theta_opt}
\end{equation}
Writing \(\mathcal{K}=|\mathcal{K}|e^{i\varphi_m}\) one may also use
\[
\tan(2\theta_{\rm opt})=\frac{2|\mathcal{K}|\cos\varphi_m}{\,1-|\mathcal{K}|^{2}\,}.
\]

\paragraph{Minimum attained value.}
Plugging the optimal angle back into \eqref{eq:Sqq_linearform} then yields
\begin{equation}
\quad
S^{r}_{qq,\min}
=(1+|\mathcal{K}|^{2})\cosh 2r - \sqrt{(1-|\mathcal{K}|^{2})^{2}+\big(2\,\mathrm{Re}\{\mathcal{K}\}\big)^{2}}\,\sinh 2r,
\label{eq:Smin_general}
\end{equation}

\paragraph{Lower bound and the real-\(\mathcal{K}\) case.}
In the free mass limit, \(\mathcal{K}\) is purely real, so that \(\varphi_m=0\) and \(\mathrm{Re}\{\mathcal{K}\}=|\mathcal{K}|\). In this case, the minimum variance \eqref{eq:Smin_general} reduces to
\[
S^{r}_{qq,\min}=(1+|\mathcal{K}|^{2})e^{-2r} 
\]


\paragraph{resultst to be used}

\begin{equation}
\frac{d\mathcal K}{d\Omega}(\Omega)
=\frac{C^2}{2}\hbar\;
\frac{2\Omega+i\gamma_m}{m\left(\Omega_m^2-\Omega^2-i\gamma_m\Omega\right)^2}.
\end{equation}

\begin{equation}
\left.\frac{d\mathcal K}{d\Omega}\right|_{\Omega=\Omega_m}
=
-\frac{C^2\hbar}{2m}\;
\frac{2\Omega_m+i\gamma_m}{\gamma_m^2\Omega_m^2}
=
-\frac{C^2\hbar}{2m}\left(\frac{2}{\gamma_m^2\Omega_m}
+i\,\frac{1}{\gamma_m\Omega_m^2}\right).
\end{equation}








\chapter*{Appendix B: Error Signals}\label{app:AppB}
\addcontentsline{toc}{chapter}{Appendix B: Error Signals} % ensures it appears in the main TOC

In this appendix, we detail calculation details not mentionned in the main text regarding the detection of optical fields and error signals. 

\section*{Direct detection error signals}
\addcontentsline{toc}{section}{Direct detection error signals}

We describe the completely generic photocurrent obtained from direct detection of two optical fields interfering on a photodetector. We consider two fields with field operators $\hat a$ and $\hat a'$, with classical amplitudes $|\bar{\alpha}|$ and $|\bar{\alpha}'| e^{-i(\Delta \omega t + \phi)}$ as well as fluctuations $\delta \hat a$ and $\delta \hat a' e^{-i(\Delta \omega t + \phi)}$ i.e. $\bar \alpha$ is real. The photocurrent operator is then given by
\begin{equation*}
\hat I = \left( |\bar{\alpha}| + \delta \hat a^\dagger + |\bar{\alpha}'| e^{i(\Delta \omega t + \phi)} + \delta \hat a'^\dagger e^{i(\Delta \omega t + \phi)} \right) \left( |\bar{\alpha}| + \delta \hat a + |\bar{\alpha}'| e^{-i(\Delta \omega t + \phi)} + \delta \hat a' e^{-i(\Delta \omega t + \phi)} \right)
\end{equation*}
We remind here the expression for the amplitude and phase quadratures for both fields
\[
\delta \hat p = \delta \hat a + \delta \hat a^\dagger \,, \quad \delta \hat q = -i \left( \delta \hat a - \delta \hat a^\dagger \right) \,,
\]
and 
\[
\delta \hat p' = e^{-i(\Delta \omega t + \phi)} \delta \hat a' + e^{i(\Delta \omega t + \phi)} \delta \hat a'^\dagger \,, \quad \delta \hat q' = -i \left( e^{-i(\Delta \omega t + \phi)} \delta \hat a' - e^{i(\Delta \omega t + \phi)} \delta \hat a'^\dagger \right) \,.
\]
Expanding this expression and keeping only terms up to first order in the fluctuations, we find
\begin{equation*}
\begin{aligned}
\hat I(t) \approx & |\bar{\alpha}|^2 + |\bar{\alpha}'|^2 + 2 |\bar{\alpha}| |\bar{\alpha}'| \cos (\Delta \omega t + \phi) \\
& + |\bar{\alpha}| \left( \delta \hat p + \delta \hat p'\right)  \\
& + |\bar{\alpha}'| \cos (\Delta \omega t + \phi) \left( \delta \hat p + \delta \hat p'\right) \\  
& + |\bar{\alpha}'|\sin (\Delta \omega t + \phi) \left( \delta \hat q - \delta \hat q'\right)\\
\end{aligned}
\end{equation*}
The first line corresponds to the classical DC and beatnote terms, while the remaining lines correspond to the fluctuations. We will now explore the different detection regimes covered in the main text.

\subsection*{Single field detection}
Let's first consider the single field case where we get rid of all terms related to $\hat a'$. The photocurrent operator then reduces to
\begin{equation*}
\hat I   \approx |\bar{\alpha}|^2 + |\bar{\alpha}|\delta \hat p.
\end{equation*}
The photocurrent fluctuations are then directly proportional to the amplitude quadrature fluctuations of the input field, scaled by the classical amplitude.

\subsection*{Two fields detection}
Let's first consider two fields with the same frequency, i.e. $\Delta \omega = 0$. The photocurrent operator then reads
\begin{equation*}
\begin{aligned}
\hat I   \approx & |\bar{\alpha}|^2 + |\bar{\alpha}'|^2 + 2 |\bar{\alpha}| |\bar{\alpha}'| \cos (\phi) \\
& + |\bar{\alpha}| \left( \delta \hat p + \delta \hat p'\right)  \\
& + |\bar{\alpha}'| \cos (\phi) \left( \delta \hat p + \delta \hat p'\right) \\  
& + |\bar{\alpha}'|\sin (\phi) \left( \delta \hat q - \delta \hat q'\right)\\
\end{aligned}
\end{equation*}
where the mean field is a simple interference between the two fields, while the fluctuations depend on both amplitude and phase quadratures of the two fields. By adjusting the relative phase $\phi$, one can select which quadrature is measured. For example, setting $\phi = 0$ selects the amplitude quadratures, while setting $\phi = \pi/2$ selects the phase quadratures. The issue is that in this case, both fields contribute to the measured quadrature fluctuations, which is not desired when probing sub shotnoise fluctuations of a signal (the LO will add its own fluctuations).

\subsection*{Two fields detection with frequency offset}
Now, we consider the case where the two fields have a frequency offset $\Delta \omega \neq 0$. The mean photocurrent then contains a beatnote at frequency $\Delta \omega$ and reads 
\begin{equation*}
\bar I= |\bar{\alpha}|^2 + |\bar{\alpha}'|^2 + 2 |\bar{\alpha}| |\bar{\alpha}'| \cos (\Delta \omega t + \phi).
\end{equation*}
such that demodulating the photocurrent at frequency $\Delta \omega' \sim \Delta \tilde \omega$ with phase $\tilde \phi$ and low-pass filtering yields
\begin{equation*}
\bar I_{\rm demod} \approx |\bar{\alpha}| |\bar{\alpha}'| \cos ((\Delta \omega - \Delta \tilde \omega)t + \phi - \tilde \phi).
\end{equation*}
This very signal can then be used to lock the frequency of an auxiliary laser to the desired frequency offset $\Delta \tilde \omega$ from the main laser. However, this signal featuring many zero crossings, one needs to tune the auxiliary laser frequency close enough to the desired offset so that it ensures the feedback loop locks to the correct zero crossing. This is generally done manually by scanning the auxiliary laser frequency until the right zero crossing is found, confirmed by monitoring the beatnote on a spectrum analyzer.


\section*{PDH error signal}
\addcontentsline{toc}{section}{PDH error signal} 

the Pound-Drever-Hall (PDH) error signal starting from the real, quantum-normalized phase-modulated electric field expression. We aim to show how the demodulated signal is a linear combination of the real and imaginary parts of the cavity reflection coefficient, with the demodulation phase selecting the appropriate quadrature for locking.


\subsection*{Input Phase-Modulated Field}

The electric field at the input of the cavity is assumed to be a coherent state that has been phase-modulated at frequency \( \Omega \), such that the classical (real) electric field takes the form:
\begin{equation}
    E_{\text{cl}}^{(\text{PM})}(t) = i \sqrt{\frac{\hbar \omega_0}{2 \varepsilon_0}} \, \alpha_0 \left[
        e^{-i\omega_0 t} - e^{i\omega_0 t}
        + \frac{i \epsilon_\phi}{2} \left( e^{-i(\omega_0 - \Omega)t} + e^{i(\omega_0 - \Omega)t} \right)
        + \frac{i \epsilon_\phi}{2} \left( e^{-i(\omega_0 + \Omega)t} + e^{i(\omega_0 + \Omega)t} \right)
    \right]
    \label{eq:pm_field}
\end{equation}
where \( \alpha_0 \) is the coherent amplitude of the carrier, \( \epsilon_\phi \ll 1 \) is a small modulation index (related to the phase modulation depth), and \( \omega_0 \) is the optical carrier frequency. This field includes both the positive and negative frequency components, as expected for a physical (Hermitian) electric field operator.

\subsection*{Reflection from the Cavity}

Each frequency component of the field is reflected with a complex frequency-dependent amplitude reflection coefficient \( r(\omega) \), such that the reflected field is:
\begin{equation}
\begin{aligned}
    E_{\text{refl}}(t) = i \sqrt{\frac{\hbar \omega_0}{2 \varepsilon_0}} \, \alpha_0 \Big[
    & r(\omega_0) e^{-i\omega_0 t} - r^*(\omega_0) e^{i\omega_0 t} \\
    & + \frac{i \epsilon_\phi}{2} \left( r(\omega_0 - \Omega) e^{-i(\omega_0 - \Omega)t} + r^*(\omega_0 - \Omega) e^{i(\omega_0 - \Omega)t} \right) \\
    & + \frac{i \epsilon_\phi}{2} \left( r(\omega_0 + \Omega) e^{-i(\omega_0 + \Omega)t} + r^*(\omega_0 + \Omega) e^{i(\omega_0 + \Omega)t} \right)
    \Big]
\end{aligned}
\label{eq:refl_field}
\end{equation}

\subsection*{Photodetected Intensity}

The photodetector measures the intensity:
\[
I(t) \propto |E_{\text{refl}}(t)|^2
\]
We isolate the terms oscillating at \( \Omega \), which arise from the interference between the carrier and sideband components. Keeping only the beat terms between the carrier and sidebands, we find:
\begin{equation}
I(t) \supset \epsilon_\phi \cdot \Re\left[ A_+ - A_- \right] \cos(\Omega t)
+ \epsilon_\phi \cdot \Im\left[ A_+ - A_- \right] \sin(\Omega t)
\label{eq:intensity_beats}
\end{equation}
where we define:
\[
A_\pm = r(\omega_0) r^*(\omega_0 \pm \Omega)
\]

\subsection*{Demodulation with Arbitrary Phase}

The signal is demodulated using a local oscillator \( \cos(\Omega t + \phi) \), where \( \phi \) is the demodulation phase. Using trigonometric identities:
\[
\cos(\Omega t + \phi) = \cos(\Omega t)\cos\phi - \sin(\Omega t)\sin\phi
\]
we multiply Equation~\eqref{eq:intensity_beats} and low-pass filter to obtain:
\begin{equation}
\epsilon_{\text{PDH}}(\phi) \propto \epsilon_\phi \left\{
\Re[A_+ - A_-] \cos\phi + \Im[A_+ - A_-] \sin\phi
\right\}
\label{eq:error_signal_general}
\end{equation}

\subsection*{Sidebands Far Off-Resonance Approximation}

In the standard PDH regime, the modulation frequency is much greater than the cavity linewidth:
\[
\Omega \gg \kappa
\]
so the sidebands are far off-resonance. This means:
\[
r(\omega_0 \pm \Omega) \approx 1 \quad \Rightarrow \quad A_\pm \approx r(\omega_0)
\]
and therefore:
\[
A_+ - A_- \approx 0
\]
However, if we retain the asymmetry between the sidebands (e.g., due to dispersion), or keep the finite detuning contribution, we approximate:
\[
A_+ - A_- \approx r(\omega_0) \left[ r^*(\omega_0 + \Omega) - r^*(\omega_0 - \Omega) \right] = r(\omega_0) \Delta r^*
\]

\subsection*{Final Result}

Substituting into Equation~\eqref{eq:error_signal_general}, we obtain:
\begin{equation}
\epsilon_{\text{PDH}}(\phi) \propto \epsilon_\phi \left\{
\Re[r(\omega_0) \Delta r^*] \cos\phi + \Im[r(\omega_0) \Delta r^*] \sin\phi
\right\}
\label{eq:error_signal_deltar}
\end{equation}

In the limit where \( \Delta r^* \rightarrow 1 \) (normalized, symmetric sidebands), this simplifies to:
\begin{equation}
\boxed{
\epsilon_{\text{PDH}}(\omega_0, \phi) \propto \cos\phi \cdot \Re[r(\omega_0)] + \sin\phi \cdot \Im[r(\omega_0)]
}
\label{eq:pdh_final}
\end{equation}

\subsection*{7. Interpretation}

Equation~\eqref{eq:pdh_final} shows that the demodulated error signal is a linear superposition of the real and imaginary parts of the complex reflection coefficient. The demodulation phase \( \phi \) selects the detected quadrature:
\begin{itemize}
    \item \( \phi = 0 \): error signal is proportional to \( \Re[r] \) — symmetric around resonance, not suitable for locking.
    \item \( \phi = \pi/2 \): error signal is proportional to \( \Im[r] \) — antisymmetric, ideal dispersive error signal.
    \item \( \phi \ne 0, \pi/2 \): mixes quadratures, possibly introducing offset or distortion.
\end{itemize}

\bigskip

This derivation makes explicit how the PDH method uses phase-sensitive detection to extract the component of the reflection coefficient that varies linearly with detuning, enabling precise feedback locking of the laser to the cavity resonance.
  



\chapter*{Appendix C: Three Mirror cavities}\label{app:AppC}
\addcontentsline{toc}{chapter}{Appendix C: Three Mirror cavities} % ensures it appears in the main TOC

In this appendix, we detail calculation details not mentionned in the main text regarding three-mirror cavities.

\section*{Three-mirror cavity fields}

We consider an input coupler mirror with amplitude reflectivity and transmissivity $r_1$ and $t_1$, a second mirror with $r_m$ and $t_m$ (to be consistent with the main text) and a third mirror with $r_2$ and $t_2$. We will consider the input and output mirrors to be HR mirrors i.e. $R_i = |r_i|^2 \sim 1$ and $T_i = |t_i|^2 \ll 1$ for $i=1,2$. This will allow us to neglect terms in $t_1^2$ and $t_2^2$ in the following calculations. The input-output relation at various coupler interfaces read
\begin{align*}
\hat a_R &= t_m \hat a_L + r_m \hat a'_R \, \\
\hat a'_L &= t_m \hat a'_R + r_m \hat a_L \, \\
\end{align*}
as well as the reflections on the input and output couplers
\begin{align*}
\hat a_L &= - r_1 \hat a_R e^{i \phi_L} + t_1 \hat a_{\rm in} \sim - \hat a_R e^{i \phi_L} + t_1 \hat a_{\rm in} \\
\hat a'_{R} &= - r_2 \hat a_{R} e^{i \phi_R} \sim \hat a_R e^{i \phi_R} \, \\
\end{align*}
where $\phi_L = 2 k L_1$ and $\phi_R = 2 k L_2$ are the lengths of the two sub-cavities. Here we will consider the reflection coefficient of the input/output couplers to be $r_{i} \sim -1$ for $i=1,2$, and $r_m = |r_m|$ and $t_m = i|t_m|$ for the middle mirror / membrane. Injecting the second system in the first one yields
\begin{align*}
  (1-r_m e^{i \phi_R}) \hat a_R & = t_m e^{i \phi_L} \hat a'_L + t_m t_1 \hat a_{\rm in} \, \\
  (1 - r_m e^{i \phi_L}) \hat a'_L & = t_m e^{i \phi_R} \hat a_R + r_m t_1 \hat a_{\rm in} \, \\
\end{align*}
such that isolating the $\hat a_{\rm in}$ and considering the mean fields yileds 
\begin{align*}
  \dfrac{\alpha_R}{\alpha'_L} = \dfrac{|t_m|(2|r_m|\sin \phi_L + i )}{|r_m|^2 - e^{i\phi_R}}
\end{align*}
and the power ration for a driven cavity yields 
\begin{align*}
  \dfrac{P_R}{P'_L} = \dfrac{|t_m|^2(4|r_m|^2 \sin^2 \phi_L +1)}{|r_m|^4 - 2 |r_m|^2 \cos \phi_R +1 } \,
\end{align*}
