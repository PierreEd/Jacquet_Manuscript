\chapter{Theory: Frequency Dependent Squeezing \& Membrane based Optomechanics}
This chapter will cover the elementary concepts required to describe an membrane based optomechanical system in a quantum regime. We will first recall basics on optical field quantization as well describing coherent and squeezed light field, to then turn to the more specific frequency dependent squeezed light field. Secondly, we will cover the mathematical description of a mechanical resonator interacting with a generic coherent optical field, highlighting the differences with the seminal optomechanical system of a mirror on a spring. Finally, we will derive the equations of motions of a membrane based optomechanical system with frequency dependent squeezed optical fields. 
\minitoc
\newpage


\section{Frequency Dependent Squeezing}

\section{Cavity Optomechanics with Membrane based systems }
\subsection{Radiation Pressure Coupling}
\subsection{Quantum Langevin Equations}


\subsubsection*{Set-up and notation}
We consider a three–mode optical model for a membrane-at-the-edge (MATE) cavity with a \emph{highly transmissive} middle membrane. The long cavity mode is denoted by \(a\); the short cavity contributes two nearby modes, \(b_{+}\) and \(b_{-}\), centered \(\pm \lambda/4\) away in displacement. In the mode basis \((a,b_+,b_-)^\top\) we take (with \(\hbar=1\))
\begin{equation}
\label{eq:H}
\mathbf H=
\begin{pmatrix}
\delta_a & J & -J\\[2pt]
J & \delta_+ & 0\\[2pt]
-\,J & 0 & \delta_-
\end{pmatrix},
\qquad
\begin{aligned}
&\delta_a = r_m G_1\,\Delta x,\\
&\delta_\pm = r_m G_2\!\left(\Delta x \mp \frac{\lambda}{4}\right),
\end{aligned}
\end{equation}
with
\begin{equation}
J=\frac{c\,t_m}{2\sqrt{L_1L_2}},\qquad
t_m^2+r_m^2=1,\qquad
G_1=\frac{\omega_0}{L_1},\qquad
G_2=-\frac{\omega_0}{L_2}.
\end{equation}
Here \(\Delta x\) is the membrane displacement from the symmetry point, \(\lambda\) the optical wavelength, \(t_m\) (\(r_m\)) the middle-membrane amplitude transmission (reflection), and \(L_{1,2}\) the long/short sub-cavity lengths. High transmissivity means \(r_m\ll 1\) while \(J=O(t_m)\) can be sizable.

The exact normal modes are eigenoperators \(A_k=\alpha_k a+\beta_k b_+ + \gamma_k b_-\) obtained from \((\mathbf H-\omega_k\mathbb I)\,(\alpha_k,\beta_k,\gamma_k)^\top=0\). From the lower rows one finds the exact amplitude ratios
\begin{equation}
\label{eq:ratios}
\frac{\beta_k}{\alpha_k}=-\frac{J}{\delta_+-\omega_k},
\qquad
\frac{\gamma_k}{\alpha_k}=+\frac{J}{\delta_--\omega_k}.
\end{equation}
The ``physical'' orange branch in the figures is the one continuously connected to the long-cavity mode \(a\).

\section*{Time-domain adiabatic elimination}
Away from the two avoided crossings at \(\Delta x\approx\pm \lambda/4\), the short-cavity detunings \(|\delta_\pm-\omega|\) are large compared to the coupling:
\begin{equation}
\varepsilon_\pm \equiv \frac{J}{|\delta_\pm-\omega|}\ll 1 .
\end{equation}
The Heisenberg equations generated by \eqref{eq:H} read
\begin{equation}
\label{eq:EOM}
\begin{aligned}
i\dot a &= \delta_a a + J b_+ - J b_-,\\
i\dot b_+ &= \delta_+ b_+ + J a,\\
i\dot b_- &= \delta_- b_- - J a.
\end{aligned}
\end{equation}
The fast spectators \(b_\pm\) can be slaved to the slow variable \(a\) by setting \(\dot b_\pm\simeq 0\) to leading order:
\begin{equation}
\label{eq:slaving}
b_+ \simeq -\frac{J}{\delta_+}\,a,\qquad
b_- \simeq \phantom{-}\frac{J}{\delta_-}\,a.
\end{equation}
Substituting \eqref{eq:slaving} into the \(a\) equation in \eqref{eq:EOM} gives an effective single-mode dynamics
\begin{equation}
\label{eq:ae-effective}
i\dot a = \Bigg[\delta_a - J^2\!\left(\frac{1}{\delta_+}+\frac{1}{\delta_-}\right)\Bigg] a .
\end{equation}
Equation \eqref{eq:ae-effective} shows that, in the dispersive region, the spectators do not acquire population to leading order; they merely induce a frequency (phase) shift of the \(a\) mode of order \(J^2/\delta_\pm\).

If optical losses are included as \(\kappa_a,\kappa_\pm\) (phenomenologically via \(\delta_a\to\delta_a-i\kappa_a/2\) etc.), the same elimination yields
\begin{equation}
\label{eq:ae-loss}
i\dot a = \left[\delta_a-\frac{i\kappa_a}{2} - J^2
\left(\frac{1}{\delta_+ - i\kappa_+/2}+\frac{1}{\delta_- - i\kappa_-/2}\right)\right] a,
\end{equation}
and the validity condition strengthens to \(J\ll \sqrt{\Delta_\pm^2+\kappa_\pm^2/4}\) with \(\Delta_\pm=\Re(\delta_\pm-\omega)\).

\paragraph{Connection to eigenvectors.}
Using \eqref{eq:ratios}, for the branch connected to \(a\) one has \(|\beta/\alpha|,|\gamma/\alpha|=O(\varepsilon_\pm)\ll1\). Thus the \(b_\pm\) weights in the physical eigenoperator are \(O(\varepsilon_\pm^2)\), fully consistent with the slaving picture \eqref{eq:slaving}.

\subsubsection*{Closed form for the physical eigenfrequency}
The exact eigenvalue equation for the orange branch obtained from the first row of \((\mathbf H-\omega\mathbb I)v=0\) together with \eqref{eq:ratios} is
\begin{equation}
\label{eq:selfconsistent}
\omega \;=\; \delta_a - J^2\!\left(\frac{1}{\delta_+ - \omega}+\frac{1}{\delta_- - \omega}\right).
\end{equation}
In the dispersive regime \(|\delta_\pm|\gg |\omega|\) one can set \(\omega\to 0\) in the denominators at first order, giving the explicit approximation
\begin{equation}
\label{eq:omega-phys-first}
\boxed{\;
\omega_{\rm phys}(\Delta x)\;\approx\; r_m G_1\,\Delta x
- J^2\!\left[\frac{1}{r_m G_2(\Delta x-\tfrac{\lambda}{4})}
+\frac{1}{r_m G_2(\Delta x+\tfrac{\lambda}{4})}\right].
\;}
\end{equation}
Combining the two fractions yields a compact dispersive form
\begin{equation}
\label{eq:omega-phys-compact}
\boxed{\;
\omega_{\rm phys}(\Delta x)\;\approx\;
r_m G_1\,\Delta x
-\frac{2J^2}{r_m G_2}\,
\frac{\Delta x}{\Delta x^2-(\lambda/4)^2}\;.
\;}
\end{equation}
Close to the symmetry point \(|\Delta x|\ll \lambda/4\), \eqref{eq:omega-phys-compact} becomes nearly linear:
\begin{equation}
\label{eq:omega-linear}
\boxed{\;
\omega_{\rm phys}(\Delta x)\;\approx\;
\underbrace{\left[r_m G_1+\frac{32J^2}{r_m G_2\,\lambda^2}\right]}_{\text{renormalized slope}}
\,\Delta x .
\;}
\end{equation}
In the usual MATE limit \(L_1\gg L_2\) (hence \(|G_1|\ll |G_2|\)), the second term typically dominates the slope; this analytic form explains the gentle ``tilt'' of the orange branch between the two avoided crossings.

\subsubsection*{Schrieffer--Wolff (block-diagonal) derivation}
For completeness, write \(H=H_0+V\) with
\(
H_0=\mathrm{diag}(\delta_a,\delta_+,\delta_-)
\)
and
\(
V=\begin{psmallmatrix}
0 & J & -J\\ J & 0 & 0\\ -J & 0 & 0
\end{psmallmatrix}.
\)
Let \(S\) be anti-Hermitian satisfying \([H_0,S]=-V\).
A suitable choice is
\begin{equation}
S = 
\begin{pmatrix}
0 & \frac{J}{\delta_a-\delta_+} & -\frac{J}{\delta_a-\delta_-}\\[2pt]
-\frac{J}{\delta_a-\delta_+} & 0 & 0\\[2pt]
\frac{J}{\delta_a-\delta_-} & 0 & 0
\end{pmatrix}.
\end{equation}
The transformed Hamiltonian \(\tilde H=e^{S}He^{-S}=H_0+\frac{1}{2}[S,V]+O(J^3/\Delta^2)\) is block-diagonal to second order, with the \(a\) block
\begin{equation}
\label{eq:SW-effective}
H_{\rm eff}^{(a)}=\delta_a
- J^2\left(\frac{1}{\delta_+-\delta_a}+\frac{1}{\delta_--\delta_a}\right),
\end{equation}
which reduces to \eqref{eq:omega-phys-first} when \(|\delta_\pm|\gg |\delta_a|\). Residual \(a\!\leftrightarrow\!b_\pm\) couplings are suppressed to \(O(J^3/\Delta^2)\).

\subsubsection*{Local avoided crossings (breakdown of elimination)}
Near \(\Delta x\simeq +\lambda/4\), only \(b_+\) is near resonant; the relevant subspace is \((a,b_+)\) with
\begin{equation}
H_{\rm loc}^{(+)}=
\begin{pmatrix}
\delta_a & J\\ J & \delta_+
\end{pmatrix},
\qquad
\Rightarrow\qquad
\omega_{\pm}^{(+)}=\frac{\delta_a+\delta_+}{2}\pm
\sqrt{\left(\frac{\delta_a-\delta_+}{2}\right)^2+J^2}.
\end{equation}
The orange branch is the one connecting continuously to \eqref{eq:omega-phys-compact} away from the crossing. The same holds at \(\Delta x\simeq -\lambda/4\) with \(b_-\).
Adiabatic elimination is invalid in windows where \(\varepsilon_\pm\not\ll1\).

\section*{Validity conditions and practical rule}
The small parameter governing all steps is
\(
\varepsilon_\pm = J/|\delta_\pm-\omega|
\).
With losses,
\(
\varepsilon_\pm=J/\sqrt{\Delta_\pm^2+\kappa_\pm^2/4}
\).
A conservative working criterion is
\begin{equation}
\boxed{\;
\max\{\varepsilon_+,\varepsilon_-\}\lesssim 0.2\!-\!0.3
\quad\Rightarrow\quad
\text{errors in }\omega_{\rm phys}\text{ are }O(\varepsilon^2),\
\text{and }|b_\pm|^2/|a|^2=O(\varepsilon^2).
\;}
\end{equation}

\subsubsection*{Optional bright/dark re-basis}
Defining \(b_s=(b_+-b_-)/\sqrt2\) and \(b_d=(b_++b_-)/\sqrt2\), one finds that \(a\) couples only to the \emph{bright} mode \(b_s\) with strength \(\sqrt2\,J\), while \(b_d\) is dark to first order. In this basis the cubic spectrum becomes a quadratic (for \(a,b_s\)) plus a spectator \(b_d\) whose frequency lies near \(r_m G_2\Delta x\) and mixes weakly via \(O(r_m G_2\lambda/2)\). This re-basis is often convenient for fitting and for visualizing how the orange branch acquires its dispersive tilt.

\paragraph{Summary.}
In a high-\(T\) middle-membrane MATE system, the short-cavity modes are far detuned for most \(\Delta x\). They can be adiabatically eliminated, yielding the explicit orange-branch dispersion
\eqref{eq:omega-phys-compact} (or \eqref{eq:omega-linear} near the center), with controlled accuracy quantified by \(\varepsilon_\pm\). Only in narrow windows around \(\Delta x=\pm\lambda/4\) is a \(2\times2\) avoided-crossing description required.

\subsection{Mechanical Resonators}{Mechanical Resonators}
\subsection{Noise spectra}
We will derive the Hamiltonian formalism of a three mirror cavity, and show how it can be used to describe the optomechanical coupling of a membrane in the cavity.
We now have to consider two optical modes coupled to one another through the membrane transmitivities. The Hamiltonian of the system can be written as:
\begin{equation}
\hat{H} = \hbar \omega_1 \hat{a}_1^\dagger \hat{a}_1 + \hbar \omega_2 \hat{a}_2^\dagger \hat{a}_2 + \hbar g(\hat{a}_1^\dagger \hat{a}_2 + \hat{a}_2^\dagger \hat{a}_1) + \frac{\hat{p}^2}{2m} + \frac{1}{2} m \omega_m^2 \hat{x}^2
\end{equation}
where $\hat{a}_1$ and $\hat{a}_2$ are the annihilation operators of the two optical modes, $\omega_1$ and $\omega_2$ their respective frequencies, $g$ the optomechanical coupling strength, $\hat{p}$ and $\hat{x}$ the momentum and position operators of the membrane, $m$ its mass and $\omega_m$ its mechanical frequency. The optomechanical coupling strength $g$ is defined as:
\begin{equation}
g = \frac{\omega_1}{L} \sqrt{\frac{\hbar}{2 m \omega_m}} \left( T_1 + T_2 \right)
\end{equation}
where $T_1$ and $T_2$ are the transmitivities of the two optical modes through the membrane. The Hamiltonian can be diagonalized by introducing the normal modes of the system, which are the eigenstates of the Hamiltonian. The normal modes can be expressed as:
\begin{equation}
\hat{b}_1 = \frac{1}{\sqrt{2}} \left( \hat{a}_1 + \hat{a}_2 \right), \quad \hat{b}_2 = \frac{1}{\sqrt{2}} \left( \hat{a}_1 - \hat{a}_2 \right)
\end{equation}
The normal modes $\hat{b}_1$ and $\hat{b}_2$ are the symmetric and antisymmetric modes of the system, respectively. The Hamiltonian can then be rewritten in terms of the normal modes as:
\begin{equation}
\hat{H} = \hbar \omega_1 \hat{b}_1
^\dagger \hat{b}_1 + \hbar \omega_2 \hat{b}_2^\dagger \hat{b}_2 + \hbar g(\hat{b}_1^\dagger \hat{b}_2 + \hat{b}_2^\dagger \hat{b}_1) + \frac{\hat{p}^2}{2m} + \frac{1}{2} m \omega_m^2 \hat{x}^2
\end{equation}


\paragraph{Diagonalisation of two non-degenerate, tunnel-coupled optical cavities.}
Let \(a_{1}\) and \(a_{2}\) (with the usual bosonic commutation relations) annihilate photons in the first and second cavity, whose bare resonance frequencies are \(\omega_{1}\neq\omega_{2}\).  Photon tunnelling at rate \(J>0\) through the semi-transparent middle mirror couples the two modes, giving the second-quantised Hamiltonian
\[
H
  =\hbar
    \begin{pmatrix}
      a_{1}^{\dagger} & a_{2}^{\dagger}
    \end{pmatrix}
    \underbrace{\begin{pmatrix}
      \omega_{1} & J\\
      J          & \omega_{2}
    \end{pmatrix}}_{\,\mathbf M}
    \begin{pmatrix}
      a_{1}\\ a_{2}
    \end{pmatrix}.
\]
Diagonalising the \(2\times2\) Hermitian matrix \(\mathbf M\) one finds the normal–mode (super-mode) eigenfrequencies
\begin{equation}
\label{eq:split}
\omega_{\pm}= \frac{\omega_{1}+\omega_{2}}{2}\;\pm\;
              \sqrt{J^{2}+\bigl(\tfrac{\omega_{1}-\omega_{2}}{2}\bigr)^{2}},
\end{equation}
and introduces a mixing angle \(\theta\) via
\[
\tan 2\theta \;=\;\frac{2J}{\,\omega_{2}-\omega_{1}\,}, 
\qquad 0<\theta<\pi/2.
\]
The corresponding canonical operators
\[
A_{+}= \cos\theta\,a_{1}+\sin\theta\,a_{2},
\qquad
A_{-}=-\sin\theta\,a_{1}+\cos\theta\,a_{2},
\]
obey \([A_{\mu},A_{\nu}^{\dagger}]=\delta_{\mu\nu}\) and bring the Hamiltonian to the diagonal form
\[
H=\hbar\omega_{+}\,A_{+}^{\dagger}A_{+}\;+\;\hbar\omega_{-}\,A_{-}^{\dagger}A_{-},
\]
revealing two independent harmonic oscillators whose frequency splitting \(\omega_{+}-\omega_{-}=2\sqrt{J^{2}+[(\omega_{1}-\omega_{2})/2]^{2}}\) interpolates smoothly between the strong-coupling limit (\(\omega_{1}\approx\omega_{2}\)) and the large-detuning regime where each cavity mode retains its individuality and the admixture of its neighbour is suppressed by the small parameter \(J/|\omega_{2}-\omega_{1}|\ll1\).


