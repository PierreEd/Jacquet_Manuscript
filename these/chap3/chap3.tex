\chapter{Theory: Squeezed Light \& Optomechanics}
This chapter will cover the elementary concepts required to describe a membrane based optomechanical system interacting with quantum light. We will first recall basics on optical field quantization as well describing coherent and squeezed light field, to then turn to the more specific frequency-dependent squeezed light field. Secondly, we will cover the mathematical description of a mechanical resonator interacting with a generic coherent optical field, highlighting the differences with the seminal optomechanical system of a mirror on a spring. 
\minitoc
\newpage


\section{Squeezed Light and Optomechanics}

We will now introduce the concept of the Standard Quantum Limit (SQL) in the context of optomechanical measurements, and show how frequency-dependent squeezed light can be used to go beyond this limit. \\ 

For the rest of this section we will assume the following:
\begin{itemize}
  \item A cavity on resonance: $\Delta=0$.
  \item A single port optomechanical cavity: $\kappa_1 \sim \kappa$ .
  \item The unresolved sideband regime: $(\Omega, \Omega_m) \ll \kappa/2$.
\end{itemize}
\subsection{Standard Quantum Limit}
The question of interest is now:
\begin{center}
  \textbf{what is the best displacement sensitivity one can achieve?}
\end{center}  

We start by recalling the reflected phase fluctuation of an optomechanical cavity from section I.2.5 under the aforementioned assumptions:
\begin{equation*}
\delta \hat{q}_{\mathrm{ref}}[\Omega] = \, \delta \hat{q}_{\mathrm{in}}[\Omega]  + \mathcal{K}[\Omega]\, \delta \hat{p}_{\mathrm{in}}[\Omega] \quad  \text{with} \quad \mathcal{K}[\Omega] = \frac{\mathcal{C}^2}{2}   \,\hbar  \chi[\Omega] =  \dfrac{128 \mathcal{F}^2 \bar I_{\rm in}}{\lambda^2}  \,  \hbar  \chi[\Omega]
\end{equation*}
where $\mathcal{C}$ is now positive and frequency independent. The resulting reflected phase spectrum reads
\begin{equation*}
  S_{qq}^{\mathrm{ref}}[\Omega] = S_{qq}^{\rm in}[\Omega] + |\mathcal{K}[\Omega]|^2 S_{pp}^{\rm in}[\Omega] + 2 \, \Re\Big[ \mathcal{K}[\Omega] S_{pq}^{\rm in}[\Omega]\Big].
\end{equation*}

The phase to displacement transduction relation with an optomechanical escape efficiency of 1 is:
\begin{equation*}
   \delta \hat q_x = \mathcal{C} \delta \hat x [\Omega] = \frac{16 \mathcal{F}\sqrt{\bar I_{\rm in}}}{\lambda}\delta \hat{x}[\Omega].
\end{equation*}
Using these two relations, we can then express displacement fluctuations in terms of input amplitude and phase fluctuations, assuming the reflected field is a perfect probe of the mechanical resonator position fluctuations i.e. $\delta \hat q_{\mathrm{ref}}[\Omega] = \delta \hat q_x[\Omega]$. This treatment is formally equivalent to considering the output phase as a statistical estimator of the position fluctuations being a stationary random process as done in quantum measurement theory \cite{clerk_introduction_2010}. We then write
\begin{equation}
  \delta \hat{x}[\Omega] =\mathcal{C}^{-1} \, \delta \hat{q}_{\mathrm{in}}[\Omega]  + \dfrac{\mathcal{C} }{2} \hbar \chi[\Omega] \, \delta \hat{p}_{\mathrm{in}}[\Omega]
\end{equation}
such that the associated displacement spectrum reads
\begin{equation}
      S_{xx}[\Omega] = \mathcal{C}^{-2} S_{qq}^{\rm in}[\Omega] + \bigg(\dfrac{\mathcal{C} }{2} \hbar |\chi[\Omega]| \bigg)^2 S_{pp}^{\rm in}[\Omega]   + \hbar |\chi[\Omega]| \Re\Big[  e^{i\phi_m[\Omega]} S_{pq}^{\rm in}[\Omega]\Big]
  \label{eq:total_displacement_spectrum}
\end{equation}

We then identify three contributions to the displacement spectrum:
\begin{itemize}
  \item The first term is the laser phase noise (also called historically 'shot noise', or more recently 'imprecision noise') scaling inversely with the input power $\bar{I}_\mathrm{in}$, arising from the input phase fluctuations $S_{qq}^{\rm in}[\Omega]$ and given by
  \begin{equation}
    S_{xx}^{\rm SN}[\Omega] = \dfrac{\lambda^2}{256 \mathcal{F}^2 \bar{I}_\mathrm{in}} S_{qq}^{\rm in}[\Omega]
  \end{equation}
  \item The second term is the radiation pressure noise (or backaction noise), scaling linearly with the input power $\bar{I}_\mathrm{in}$, arising from the input amplitude fluctuations $S_{pp}^{\rm in}[\Omega]$ driving the mechanical resonator via radiation pressure given by 
  \begin{equation}
    S_{xx}^{\rm RPN}[\Omega] = \, \dfrac{64 \mathcal{F}^2 \bar{I}_\mathrm{in}}{\lambda^2}\hbar^2 |\chi[\Omega]|^2 S_{pp}^{\rm in}[\Omega]
  \end{equation}
  \item The third term is a correlation term between amplitude and phase fluctuations $S_{pq}^{\rm in}[\Omega]$, which is non-zero for a detuned cavity and/or for arbitrary squeezed states as seen in the previous section, and given by
  \begin{equation}
    S_{xx}^{\rm cor}[\Omega] = \hbar |\chi[\Omega]|\Re\Big[ e^{i\phi_m[\Omega]} S_{pq}^{\rm in}[\Omega]\Big]
  \end{equation}
\end{itemize}
And we write the total displacement spectrum as the sum of these three contributions
\begin{equation}
  S_{xx}[\Omega] = S_{xx}^{\rm SN}[\Omega] + S_{xx}^{\rm RPN}[\Omega] + S_{xx}^{\rm cor}[\Omega]
\end{equation}

We now consider vacuum/coherent fluctuations such that $S_{qq}^{\rm in}[\Omega]=S_{pp}^{\rm in}[\Omega]=1$ and $S_{pq}^{\rm in}[\Omega]=0$, so that the displacement spectrum simplifies to
\begin{equation}
  S_{xx}[\Omega] = \mathcal{C}^{-2}  + \bigg(\dfrac{\mathcal{C} }{2} \hbar |\chi[\Omega]| \bigg)^2
  \label{eq:total_displacement_spectrum}
\end{equation}
and we look at what noise dominates the displacement spectrum around the mechanical resonance $\Omega \sim \Omega_m$. In this frequency range, there are two frequencies at which the displacement noise contributions are equal, given by the condition $S_{xx}^{\rm SN}[\Omega] = S_{xx}^{\rm RPN}[\Omega]$, leading to the frequency $\Omega_{\text{SQL}}$ defined as
\begin{equation}
  \Omega^{\pm}_{\text{SQL}}  =\;\sqrt{\Omega_m^2-\dfrac{\Gamma_m^2}{2}
\;\pm\;\dfrac{1}{2}\sqrt{\Gamma_m^4-4\Gamma_m^2\Omega_m^2+\left(\frac{\hbar \mathcal{C}^2}{m }\right)^2}}
\end{equation}
and consistent with the LIGO/Virgo notation \cite{harry_advanced_2010, aasi_enhanced_2013}. Over the frequency range of interest $\Omega \in [\Omega_m - \Omega_{\text{SQL}}, \Omega_m + \Omega_{\text{SQL}}]$, the displacement noise is dominated by radiation pressure noise, while outside this range, the noise is dominated by shot noise. However, for every sideband frequency, there exists an optimal input power $\bar{I}_\mathrm{in}^{\rm SQL}[\Omega]$ at which both contributions are equal, minimizing the total displacement noise. This limit is called the Standard Quantum Limit (SQL) and is a direct consequence of Heisenberg's inequality applied to continuous position measurements \cite{braginsky_quantum_1992, clerk_introduction_2010}. This SQL intensity is given by
\begin{equation}
  S_{xx}^{\rm SN}[\Omega] = S_{xx}^{\rm RPN}[\Omega] \implies \bar{I}_\mathrm{in}^{\rm SQL}[\Omega] = \dfrac{\lambda^2}{128\mathcal{F}^2 \hbar |\chi[\Omega]|}
\end{equation}
such that plugging back in this SQL intensity in \eqref{eq:total_displacement_spectrum} gives the SQL displacement spectrum as
\begin{equation}
  S_{xx}^{\rm SQL}[\Omega] = \hbar |\chi[\Omega]| \implies S_{xx}^{\rm SN}[\Omega] + S_{xx}^{\rm RPN}[\Omega] \geq \hbar |\chi[\Omega]|
\end{equation}
which is the fundamental limit to continuous position measurements with coherent light. We also note that for high-Q resonators, $\Omega_{SQL} \gg \Gamma_m$, so approximating the succeptibility by its real part holds over a relatively large frequency range but fails at resonance where the succeptibility is purely imaginary.

\subsubsection{Thermal Noise}
Thermal noise is a major limitation in optomechanical experiments, as it can mask the quantum effects. The mechanical resonator is coupled to a thermal bath at temperature $T>0$, which drives the resonator into a thermal state with mean phonon occupation number $\bar n_{\mathrm{th}} \simeq k_B T / (\hbar \Omega_m)$ in the high temperature limit $k_B T \gg \hbar \Omega_m$. The position fluctuations induced by this thermal force is given by
\begin{equation}
  S_{xx}^{\mathrm{th}}[\Omega] = \dfrac{2\hbar}{1 - e^{-\hbar \Omega / k_B T}} \Im \chi[\Omega] \simeq 2 m \Gamma_m k_B T |\chi[\Omega]|^2 \quad \text{if} \quad k_B T \gg \hbar \Omega
\end{equation}
where we used the identity $\Im \chi[\Omega] = m\Gamma_m \Omega |\chi[\Omega]|^2$. At $T=0$ K, this reduces to the zero-point fluctuations spectrum $S_{xx}^{\mathrm{ZPF}}[\Omega] =\, m \Gamma_m \hbar \Omega_m  |\chi[\Omega]|^2 < S_{xx}^{\rm SQL}[\Omega]$, such that is often neglected in the total displacement spectrum. However, at finite temperature, the thermal noise can be much larger than the SQL. Therefore, the total displacement spectrum now reads
\begin{equation}
  S_{xx}[\Omega] = S_{xx}^{\rm SN}[\Omega] + S_{xx}^{\rm RPN}[\Omega] + S_{xx}^{\rm cor}[\Omega] + S_{xx}^{\mathrm{th}}[\Omega]
\end{equation}
In order to experimentally probe these quantum limits without being limited by various technical noises, one would then need: 
\begin{itemize}
  \item A high finesse cavity, such that the shot noise $S_{xx}^{\rm SN} \propto \mathcal{F}^{-2}$ level is low, and the radiation pressure noise $S_{xx}^{\rm RPN} \propto \mathcal{F}^{2}$ is high. One should however ensured the cavity bandwidth $\kappa$ is still much larger than the mechanical frequency $\Omega_m$. This can be ensured tuning the cavity length $L$ and mirror transmissions.
  \item A low mass, low frequency, high quality factor mechanical resonator, such that the susceptibility modulus at resonance $|\chi[\Omega_m]|=Q/m \Omega_{m}^2$ is high, and it comes out of the shot noise level significantly. 
  \item A low temperature environment, such that the thermal noise $S_{xx}^{\mathrm{th}} \propto T$ is low and does not mask the quantum effects. This can be ensured by cryogenic cooling of the mechanical resonator, as well as using high quality factor resonators to reduce the mechanical linewidth $\Gamma_m$.
\end{itemize}

\begin{figure}[!htbp]
  \centering
  \includegraphics[width=\textwidth]{./chap3/fig/SQL0.pdf}
  \caption{Displacement spectrum contributions for a coherent input field (vacuum fluctuations) with typical table-top experiment parameters. (a) Displacement spectrum contributions as a function of the input power at the mechanical resonance frequency $\Omega_m$. The shot noise (red) decreases with increasing input power, while the radiation pressure noise (yellow) increase with increasing input power. The SQL power (dashed line) is the input power at which both contributions are equal, minimizing the total displacement spectrum (black). (b) Displacement spectrum as a function of frequency for a fixed input power on a large frequency band. The shot noise is flat as long as the resonator sits in the unresolved sideband regime, and the radiation pressure noise peaks around the mechanical resonance frequency $\Omega_m$. The zero point motion sits orders of magnitude below the SQL. (c) Displacement spectrum as a function of frequency in the vicinity of the mechanical resonance frequency $\Omega_m$. Only two frequencies $\Omega_m \pm \Omega_{\rm SQL}$ (dashed lines) exist where both contributions are equal, with radiation pressure noise dominating in between, and shot noise dominating outside this range. The total displacement spectrum (black) reaches the SQL (dotted line) at these two frequencies for the given input power.}
    \label{fig:freq_indep_squeezing}
\end{figure}

We now want to derive the displacement spectrum of an optomechanical system driven by a squeezed light field, whether frequency-independent or frequency-dependent. 

\subsection{Frequency-Independent Squeezing in Optomechanical Cavities}
We first recall the (idealized) covariance matrices for both a phase squeezed field and an amplitude squeezed field 
\begin{equation*}
  \mathbf{S}^{0}_{\rm OPO}[\Omega] = 
  \begin{pmatrix}
    e^{+2r} & 0\\
    0 & e^{-2r}
  \end{pmatrix}, \quad
  \mathbf{S}^{\pi}_{\rm OPO}[\Omega] = 
  \begin{pmatrix}
    e^{-2r} & 0\\
    0 & e^{+2r}
  \end{pmatrix}
\end{equation*}
For a phase squeezed field, the displacement spectrum reads
\begin{equation}
  S_{xx}^{0}[\Omega] =   \mathcal{C}^{-2}  e^{-2r} + \bigg(\dfrac{\mathcal{C} }{2} \hbar |\chi[\Omega]| \bigg)^2 e^{+2r}
\end{equation}
while for an amplitude squeezed field, the displacement spectrum reads
\begin{equation}
  S_{xx}^{\pi}[\Omega] =  \mathcal{C}^{-2}  e^{+2r} + \bigg(\dfrac{\mathcal{C} }{2} \hbar |\chi[\Omega]| \bigg)^2 e^{-2r}
\end{equation}
We then see that phase squeezing reduces the shot noise contribution but increases the radiation pressure noise contribution, while amplitude squeezing reduces the radiation pressure noise contribution but increases the shot noise contribution. The input cross correlations being zero, this is completely equivalent to the coherent state with a rescaled input intensity $e^{\pm 2r} \bar{I}_\mathrm{in} $ (hidden in $\mathcal{C}$) for phase/amplitude squeezing respectively, as seen in Fig.\ref{fig:SQLVirgo}.(a) and Fig.\ref{fig:SQL_LKB}.(a). However, neither of these two configurations can reduce both contributions simultaneously, and therefore cannot improve the SQL limit. This is illustrated in figure \ref{fig:freq_indep_squeezing}. \\ 

Now consider an input squeezed state with a frequency-independent squeezing angle $\theta=\pi/4$ with covariance matrix
\begin{equation*}
  \mathbf{S}^{\pi/4}_{\rm OPO}[\Omega] = \begin{pmatrix}
         \cosh 2r   & -\sinh 2r  \\[10pt]
        -\sinh 2r  & \cosh 2r  
      \end{pmatrix}.
\end{equation*}
The resulting displacement spectrum then reads
\begin{equation}
  S_{xx}^{\pi/4}[\Omega] = \Bigg(\mathcal{C}^{-2}  + \bigg(\dfrac{\mathcal{C} }{2} \hbar |\chi[\Omega]| \bigg)^2\Bigg)\cosh 2r - \hbar |\chi[\Omega]| \sinh 2r \cos \phi_m[\Omega]
\end{equation}
and we seek the frequency range where the displacement spectrum is below the SQL, i.e. $S_{xx}^{\pi/4}[\Omega] < S_{xx}^{\rm SQL}[\Omega]$. This condition is satisfied when
\begin{equation}
  \tanh r < \cos \phi_m[\Omega] < 1
\end{equation}
Because $\tanh r$ tends to 1 as $r$ increases, the frequency range where the displacement spectrum is below the SQL decreases with increasing squeezing factor $r$: the larger the effect, the smaller the frequency band where it is effective. Furthermore, due to the interplay between quadrature correlations and the projection of the $\pi/4$ ellipse onto the output quadrature axis, acting as an effective increase of the shot noise floor with effective intensity $\bar{I}_\mathrm{in} \cosh^{-1} r $, there is an effective range of $r$ above which the displacement spectrum is always above the SQL (for a fixed input intensity). This is illustrated in figure \ref{fig:freq_indep_squeezing}. \\ 

\begin{figure}[!htbp]
  \centering
  \includegraphics[width=\textwidth]{./chap3/fig/SQLVirgo.pdf}
  \caption{Displacement spectrum for a Virgo-LIGO type interferometer. (a) Spectra for increasing powers which doesn't allow to beat the SQL. (b) Spectra for the various squeezing angles at fixed power allowing to beat the SQL around the mechanical resonance frequency only at $\theta= \pi/4$. Phase and amplitude squeezing are equivalent to rescaling the input power, and cannot beat the SQL. The grey area indicates the range over which sub-SQL sensitivity is achieved for a given squeezing factor.} 
  \label{fig:SQLVirgo}
\end{figure}

\begin{figure}[!htbp]
  \centering
  \includegraphics[width=\textwidth]{./chap3/fig/SQL_LKB.pdf}
  \caption{Displacement spectrum for a table-top optomechanical experiment. (a) As in the Virgo-LIGO type interferometer, spectra for increasing powers don't allow to beat the SQL. (b) Spectra for the various squeezing angles at fixed power allowing to beat the SQL around the mechanical resonance frequency only at $\theta= \pi/4$.}
  \label{fig:SQL_LKB}
\end{figure}

Additionally, and as seen in Fig.\ref{fig:FDS_angle}, the optimal angle to maximally reduce the displacement spectrum varies with frequency, being 0 at frequencies outside the resonator's bandwidth, $\pi/2$ at the mechanical resonance frequency $\Omega_m$, and about $\pm\pi/4$ at $\Omega_m \pm \Omega_{\rm SQL}$. \\

This motivates the use of frequency-dependent squeezed states to reduce the displacement spectrum below the SQL over a broad frequency range, where every sideband frequency needs to be rotated by a different angle to minimize the displacement spectrum. More specifically, sideband noises contributing to both shot noise and radiation pressure noise need to be correlated in a frequency-dependent manner to optimally lower the total displacement noise in the vicinity of the mechanical resonance. 

\subsection{Frequency-Dependent Squeezing in Optomechanical Cavities}
We now consider a squeezed state with a frequency-dependent angle whose covariance matrix is given by
\begin{equation*}
      \mathbf{S}^{\theta}_{\rm OPO}[\Omega] =\begin{pmatrix}
         \cosh 2r  + \sinh 2r \, \cos 2\theta[\Omega]  & -\sinh 2r \, \sin 2\theta[\Omega]  \\[10pt]
        -\sinh 2r \, \sin 2\theta[\Omega]  & \cosh 2r  - \sinh 2r \, \cos 2\theta[\Omega] 
      \end{pmatrix}
\end{equation*}
The resulting displacement spectrum then reads
\begin{equation}
  \begin{split}
      S_{xx}[\Omega] = & \, \mathcal{C}^{-2} (\cosh 2r  - \sinh 2r \, \cos 2\theta[\Omega])\\
      & +  \bigg(\dfrac{\mathcal{C} }{2} \hbar |\chi[\Omega]| \bigg)^2( \cosh 2r  + \sinh 2r \, \cos 2\theta[\Omega]) \\
      & - \hbar |\chi[\Omega]| \sinh 2r \, \sin 2\theta[\Omega] \, \cos \phi_m[\Omega]
  \end{split}
\end{equation}
As shown in the Annex A, picking the squeezing angle as
\begin{equation}
  2 \theta[\Omega] = \arctan \Big[\dfrac{2|\mathcal{K}[\Omega]|\cos \phi_m[\Omega]}{1 - |\mathcal{K}[\Omega]|^2}\Big]
\end{equation}
minimizes the displacement spectrum at every sideband frequency, leading to
\begin{equation}
\begin{split}
  S_{xx}[\Omega] = & \cosh 2r \Bigg(\mathcal{C}^{-2}  + \bigg(\dfrac{\mathcal{C} }{2} \hbar |\chi[\Omega]| \bigg)^2\Bigg)  \\
  & -  \sinh 2r \sqrt{\Bigg(\mathcal{C}^{-2}  - \bigg(\dfrac{\mathcal{C} }{2} \hbar |\chi[\Omega]| \bigg)^2\Bigg)^2 + \bigg(\hbar |\chi[\Omega]|\cos \phi_m[\Omega]\bigg)^2}.
\end{split}
\end{equation}
This broadband reduction of the displacement spectrum below the SQL is illustrated in figure \ref{fig:freq_dep_squeezing}. However, for a resonant optomechanical cavity i.e. $\Delta=0$, it is impossible to beat the SQL at the mechanical resonance, where the succeptibility is purely imaginary $\phi_m[\Omega_m] = \pi/2$. \\

\begin{figure}[!htbp]
  \centering
  \includegraphics[width=\textwidth]{./chap3/fig/FDS_angle.pdf}
  \caption{Justification of the use of frequency-dependent squeezing to beat the SQL, shown for both a table-top optomechanical experiment (a) and a Virgo-LIGO type interferometer (b). At each frequency, a different angle is required to minimize the displacement spectrum. The optimal angle varies from 0 far from the mechanical resonance to $\pi/2$ at resonance, passing through $\pm \pi/4$ at the SQL frequencies $\Omega_m \pm \Omega_{\rm SQL}$.}
  \label{fig:FDS_angle}
\end{figure}

\noindent \textbf{Convergence to VIRGO/LIGO notation:} We once again show that this general treatment converges to the one used in the context of gravitational wave detectors. In the free mass regime, $\mathcal{K}[\Omega]$ is real, such that $\phi_m[\Omega]=0$. One can then rewrite the optimal squeezing angle as
\begin{equation}
  2 \theta[\Omega] = \arctan \Big[\dfrac{2\mathcal{K}[\Omega]}{1 - \mathcal{K}^2[\Omega]}\Big] = 2 \arctan \mathcal{K}[\Omega]
\end{equation}
where we used the identity $\arctan 2x/(1-x^2) = 2\arctan x \, \, (\text{mod}\, \pi)$, such that this comes down to the expression used in the context of gravitational wave detectors \cite{harry_advanced_2010, aasi_enhanced_2013}. Furthermore, the mechanical frequency and damping rate will be significantly smaller than the $\hbar \mathcal{C}^2/m$ term such that using the free-mass succeptibility $\chi[\Omega] = -1/m \Omega^2$ boils down the SQL frequency to the known expression
\begin{equation}
  \Omega_{\text{SQL}} = \sqrt{\dfrac{\hbar \mathcal{C}^2}{2 m }} \implies \mathcal{K}[\Omega] = \left( \dfrac{\Omega_{\text{SQL}}}{\Omega}\right)^2
\end{equation}
The displacement spectrum then reduces to the common expression
\begin{equation}
  S_{xx}[\Omega] = \mathcal{C}^{-2} \Bigg( 1  + \left( \dfrac{\Omega_{\text{SQL}}}{\Omega}\right)^2 \Bigg)e^{-2r}
\end{equation}
which is the free-mass approximation result used in the GW community. 

\subsection{Filter Cavities for Frequency-Dependent Squeezing}

To generate frequency-dependent squeezed states, one can use a detuned optical cavity called a filter cavity \cite{kimble_conversion_2001}. The principle is to reflect a frequency-independent squeezed state off a single-sided detuned cavity, such that only the sidebands resonant with the cavity will undergo a phase shift, effectively rotating the squeezing ellipse by a frequency-dependent angle as seen in Fig.\cite{FDS_QSD} \\ 

In the table-top optomechanical experiment context where the mechanical resonance frequency $\Omega_m$ is much larger than the mechanical linewidth $\Gamma_m$, the optimal squeezing angle varies rapidly around the mechanical resonance frequency, requiring a filter cavity with a bandwidth on the order of the mechanical linewidth $\kappa_{\rm fc} \sim \Gamma_m$. Since it needs to match the phase response of the optomechanical system dictated by the mechanical succeptibility, the filter cavity needs to be detuned by $\Delta_{\rm fc} = - \Omega_m$ to provide the right rotation direction. We then want to derive the derivative of the optimal squeezing angle at the mechanical resonance frequency to relate it to the filter cavity bandwidth. We start from the optimal squeezing angle expression
\begin{equation*}
  \theta[\Omega] = \dfrac{1}{2} \arctan \Big[\dfrac{2|\mathcal{K}[\Omega]|\cos \phi_m[\Omega]}{1 - |\mathcal{K}[\Omega]|^2}\Big]
\end{equation*}
and we give the derivative at the mechanical resonance frequency $\Omega = \Omega_m$ derived in the Annex A:
\begin{equation}
  \left.\frac{d\theta_{\rm opt}}{d\Omega}\right|_{\Omega_m}
=
-\frac{2|\mathcal K(\Omega_m)|}{\Gamma_m\left(1-|\mathcal K(\Omega_m)|^2\right)}
\end{equation}
As seen in \ref{sec:simple_cavities}, the derivative of the phase shift picked up by the two-photon quadratures in a high-finesse cavity is given by 
\begin{equation*}
  \left.\dfrac{\partial \bar \phi}{\partial \Omega}\right|_{\Omega = -\Delta} = -\dfrac{2}{\kappa}
\end{equation*}
and simply equating both derivatives leads to the filter cavity bandwidth required to match the optimal squeezing angle rotation:
\begin{equation}
  \kappa_{\rm fc} = \Gamma_m   \dfrac{1 - |\mathcal K(\Omega_m)|^2}{|\mathcal K(\Omega_m)|}. 
\end{equation}
Upon the single-port cavity assumption, this expression can be used to compute the required input mirror transmittivity of the filter cavity to achieve the desired bandwidth $\kappa_{\rm fc}$, this is 
\begin{equation}
  T_{\rm fc} = \dfrac{L_{\rm fc} \kappa_{\rm fc}}{c} 
\end{equation}
where $L_{\rm fc}$ is the filter cavity length. Although this expression gives a first estimate for the required length and finesse necessary for such a task, a more accurate treatment taking into account losses and non-ideal escape efficiency is required to optimize the filter cavity design. This additionally motivates the use of three-mirror cavities as filter cavities to have more degrees of freedom to optimize the cavity parameters. The properties of such cavities are detailed in the next subsection. \\

For a Virgo-LIGO type interferometer, where both the bandwidth of the filter cavity are sideband frequencies of interest are comparable to the mechanical resonance frequency $\kappa_{\rm fc}, \Omega \sim \Omega_m$, and where the free-mass approximation holds, the relative optimal squeezing angle varies slowly with frequency, requiring a filter cavity bandwidth on the order of the mechanical resonance frequency $\kappa_{\rm fc} \sim \Omega_m$. The filter cavity detuning is then set to $\Delta_{\rm fc} = - \Omega_{\rm SQL}$ to provide the right rotation direction around the SQL frequency. In this regime, the optimal filter cavity parameters were derived by C. Whittle \textit{et al.} \cite{whittle_optimal_2020}, leading to the expressions

\newpage 

Following Whittle \textit{et al.} (Phys.\ Rev.\ D \textbf{102}, 102002 (2020)), in the single-port (single-ended) and low-loss regime, the filter cavity can be parametrized by a coupler-limited bandwidth $\gamma$ and a loss-limited bandwidth $\lambda$,
\[
\gamma=\frac{c\,T_{\rm in}}{4L_{\rm fc}},\qquad \lambda=\frac{c\,A}{4L_{\rm fc}},
\]
where $L_{\rm fc}$ is the cavity length, $T_{\rm in}$ the input-mirror power transmissivity, and $A$ the round-trip power loss. Requiring the quadrature rotation produced by the cavity reflection to match the optimal frequency-dependent rotation $\alpha_{\rm opt}(\Omega)=\arctan(\Omega_{\rm SQL}^2/\Omega^2)$ leads to the phase-matching conditions
\[
2\gamma\,\Delta\omega_{\rm fc}=\Omega_{\rm SQL}^2,\qquad
\gamma^2-\lambda^2-\Delta\omega_{\rm fc}^{\,2}=0,
\]
which admit the closed-form solution (expressed in terms of $\lambda$ and $\Omega_{\rm SQL}$):
\[
\gamma=\sqrt{\frac{\lambda^2+\sqrt{\lambda^4+\Omega_{\rm SQL}^4}}{2}},\qquad
|\Delta\omega_{\rm fc}|=\sqrt{\frac{-\lambda^2+\sqrt{\lambda^4+\Omega_{\rm SQL}^4}}{2}}.
\]
The \emph{sign} of $\Delta\omega_{\rm fc}$ is then chosen to obtain the desired \emph{rotation direction} (in many GW-detector conventions one sets $\Delta\omega_{\rm fc}=-|\Delta\omega_{\rm fc}|$ around $\Omega_{\rm SQL}$). Finally, the corresponding optimal input coupler transmissivity follows directly from the bandwidth,
\[
T_{\rm in}^{\rm(opt)}=\frac{4L_{\rm fc}}{c}\,\gamma,
\]
(with losses entering through $\gamma(\lambda,\Omega_{\rm SQL})$ above).


  


Eventually, the assumptions we made to derive the table-top optomechanical experiment filter cavity parameters break down, since both derivatives $\partial_\Omega \phi_+$ and $\partial_\Omega \phi_-$ contribute significantly to the optimal squeezing angle rotation rate. 


\begin{figure}[!htbp]
  \centering
  \includegraphics[width=\textwidth]{./chap3/fig/QSD_sqzFDS.pdf}
  \caption{}
  \label{fig:FDS_QSD}
\end{figure}

\subsection{Numerical simulations}

\section{Cavity Optomechanics with Membrane based systems }
\subsection{Classical Description}
To gain intuition and derive elementary parameters used in the next section, we first describe the classical fields propagating in a three-mirror cavity where a membrane with complex amplitude reflection and transmission coefficients \(r_m=|r_m|e^{i\phi_r}\) and \(t_m=|t_m|e^{i\phi_t}\) is placed between two high reflectivity mirrors of amplitude reflection coefficients \( \sim -1 \). The membrane splits the cavity in two sub-cavities of lengths \(L_1\) and \(L_2\), with \(L=L_1+L_2\) the total cavity length. The membrane is initially at mean position $x=0$, and is modelled as a thin dielectric slab of thickness \(d\) and refractive index \(n\), with amplitude reflection and transmission coefficients \(r_m\) and \(t_m\) given by \cite{Thompson2008Nature, jayich_dispersive_2008}:
\begin{equation}
r_m = \frac{(n^2-1)\sin k n d}{2 i n \cos k n d  + (n^2+1)\sin k n d}, \qquad t_m = \frac{2 n}{2 i n \cos k n d  + (n^2+1)\sin k n d}. 
\label{eq:membrane_rt}
\end{equation}
In the lossless case, we will assume the index of refraction $n$ is real, such that \(|r_m|^2 + |t_m|^2 = 1\). The right-moving mean field amplitudes in the left and right sub-cavities are denoted \(\bar \alpha_L\) and \(\bar \alpha_R\), while the left-moving mean field amplitudes are denoted \(\bar \alpha_L'\) and \(\bar \alpha_R'\). The cavity fields are then related by 
\begin{equation}
\begin{split}
\bar \alpha_R  &= t_m \bar \alpha_L + r_m \bar \alpha_R' \\
\bar \alpha_L' &= t_m \bar \alpha_R'+ r_m \bar \alpha_L. \\
\end{split}
\end{equation}
In this case, energy conservation i.e. $|\bar \alpha_L|^2 + |\bar \alpha_R'|^2 = |\bar \alpha_L'|^2 + |\bar \alpha_R|^2$ imposes that $2(\phi_t-\phi_r) = \pi$ such that we can choose $r_m = |r_m|$ and $t_m = i|t_m|$. We rewrite the the cavity fields by injecting the identities $\bar \alpha_L = - \bar \alpha_L' e^{2ikL_1}$ and $\bar \alpha_R' = - \bar \alpha_R e^{2ikL_2}$ such that we get the useful system 
\begin{equation}
\begin{split}
  (1+ |r_m| e^{2ikL_2}) \bar \alpha_R & = -i |t_m| \, e^{2ikL_1}  \, \bar \alpha_L'\\
  (1+ |r_m| e^{2ikL_1}) \bar \alpha_L' & = -i |t_m| \, e^{2ikL_2} \,  \bar \alpha_R. 
  \label{eq:cavity_fields_system}
    \end{split}
\end{equation}

\subsubsection{Resonance Frequencies}
By eliminating the right and left fields in the above system, we arrive at the transcendental equation \cite{jayich_dispersive_2008}:
\begin{equation}
  -\cos kL = |r_m|\cos (k \Delta L), \quad \text{with} \quad \Delta L = L_2 - L_1.
\end{equation}
Following the method in Sankey \textit{et al.} \cite{sankey_strong_2010}, we now proceed to derive the cavity resonance frequencies as a function of the membrane position \(x\) around its mean position \(x=0\). We will also always consider a long cavity such that \( L \gg \lambda,|x| \). The cavity sublenghts considering a non zero mean membrane position are then \(L_1 \rightarrow L_1 +  x\) and \(L_2 \rightarrow L_2 -  x\). It follows that $\Delta L \rightarrow \Delta L - 2x$. We will consider the effect of this displacement on the cavity wavenumbers/frequencies as a perturbation $k(x) = k_N + \delta k(x)$ with $k_N = N \pi/L$, that is the membrane displacement does not change the longitudinal mode index \(N\) but modulates it by at most $\pi/L$ (or equivalently by one empty cavity FSR in the frequency domain). We will omit the $x$ dependency in both $k$ and $\delta k$ for ease of notation. It then follows than terms in $k\, L$ and $k\, x$ can be expanded as
\begin{equation*}
    \cos \big( k \, L \big)  = (-1)^N \cos \big( \delta k \,L \big) \quad \text{and} \quad      \cos \big( k \, x \big)  \sim \cos \big( k_N \,  x \big)  
\end{equation*}
where we assumed that $\delta k \, x \sim 0$. The transcendental equation becomes
\begin{equation}
  \begin{split}
  (-1)^{N+1} \cos \big( \delta k \,L \big) = |r_m|\cos (k_N \,\Delta L) \Big[ &\cos( \delta k \,  \Delta L) \cos (2 k_N\, x) \\ 
  & +  \sin(\delta k \,  \Delta L) \sin(2 k_N x) \Big]  
  \end{split}
\end{equation}
 and where we have simplified the sines terms already equal to zero. 
We will now consider the Membrane At The Edge (MATE) model where $L_1 \sim L \gg L_2 \rightarrow \Delta L \sim L$. Solving for $\delta k$ reinjecting in the dispersion relation $\omega_c(x) = c k(x)$ leads to
\begin{equation}
    \omega_c(x)  \simeq \omega_{FSR} \Bigg( N  + \frac{1}{\pi} \arctan \bigg( -  \dfrac{1 + |r_m|\cos 2 k_N  x}{|r_m|\sin 2 k_N x} \bigg)\Bigg)
\end{equation}
where $\omega_{FSR} = \pi c / L$ is the empty cavity free spectral range. When the laser is resonant with the cavity, we then substitute $N \omega_{FSR}$ and $k_N$ by $\omega_0$ and $k$ the laser angular frequency and wavenumber. Taking the derivatives of these resonance frequencies with respect to the membrane position \(x\) gives the linear and quadratic dispersive optomechanical couplings \(G^{(1)}(x) = \partial \omega_c/\partial x\) and \(G^{(2)}(x) = \partial^2 \omega_c/\partial x\) as
\begin{equation}
  \begin{split}
     G^{(1)}(x) & = \dfrac{2 |r_m|  k_N \omega_{FSR}}{\pi}  \dfrac{|r_m|  + \cos( 2 k_N  x)}{1 + |r_m|^2 - 2 |r_m| \cos( 2 k_N  x)} \\
  G^{(2)}(x) & = -\dfrac{4 |r_m|  k_N^2 \omega_{FSR}}{\pi}  \dfrac{|r_m| (1- |r_m|^2 ) \sin( 2 k_N  x)}{(1 + |r_m|^2 - 2 |r_m| \cos( 2 k_N  x))^2}
  \end{split}
\end{equation}

\color{red} use of G2 ??  \color{black}

\begin{figure}[!htbp]
  \centering
  \includegraphics[width=\textwidth]{./chap3/fig/modelsMATE.pdf}
  \caption{Resonance frequencies as a function of membrane position for various membrane reflectivities in the MATE configuration for a realistic system ($r_m = 0.7$, $L=30mm$). (a) Schematic of the MATE system. (b) Schematic of the two approaches used to derive the resonance frequencies: field nodes weakly modulated by a dielectric layer (top) and two coupled sub-cavities (bottom). (c) Resonance frequencies as a function of membrane position for various membrane reflectivities. The green and yellow lines correspond to the two eigenmodes of the uncoupled sub-cavities. (d) Absolute values of the linear and quadratic optomechanical couplings as a function of membrane position, displaying regions where the quadratic coupling dominates.}
    \label{fig:models_MATE} 
\end{figure}



\begin{figure}[!htbp]
  \centering
  \includegraphics[width=\textwidth]{./chap3/fig/MATEresTot.pdf}
  \caption{Behaviour of the MATE resonance frequencies as a function of membrane reflectivity for a realistic system ($L=30mm$, $T_1=100$ppm, $T2=20$ppm) with different membrane reflectivities. (a) Resonance frequencies as a function of membrane position. (b) Linear optomechanical coupling as a function of membrane position. (c) outcoupling efficiency as a function of membrane position. (d) Cavity finesse as a function of membrane position.}
  \label{fig:mate_res}
\end{figure}


\subsubsection{Cavity transmission and reflection}
From the system in \eqref{eq:cavity_fields_system}, and having derived just above the resonant cavity wavevectors $k$, we can compute the power ratio of the two sub-cavity fields as a function of $x$ when the MATE system is on resonance. This is
\begin{equation}
  \dfrac{|\bar \alpha_R|^2}{|\bar \alpha_L'|^2} = \dfrac{P_R}{P_L} = \dfrac{1 + 2|r_m|\cos (2k L_1+2kx) + |r_m|^2}{1- |r_m|^2}. 
\end{equation}
with $P_{L,R} \propto |\bar \alpha_{L,R}|^2$. It then follows that the power fraction leaking from the left and right mirrors, i.e. the resonant reflection and transmission coefficients $R(\Delta=0,x)$ and $T(\Delta = 0, x)$ are given by:
\begin{equation}
  \begin{split}
  R(\Delta=0,x)& = \dfrac{|t_1|^2 P_L}{|t_1|^2 P_L + |t_2|^2 P_R} \\
  & = \frac{|t_1|^2(1-|r_m|^2)}{|t_1|^2(1-|r_m|^2)  +  |t_2|^2\bigl(1 + |r_m|^2 + 2|r_m|\cos 2kx\bigr)}\\
  T(\Delta=0,x)& = \dfrac{|t_2|^2 P_R}{|t_1|^2 P_L + |t_2|^2 P_R} \\
  & = \frac{|t_2|^2 \bigl(1 + |r_m|^2 + 2|r_m|\cos 2kx \bigr)}{|t_1|^2 (1-|r_m|^2)
   + |t_2|^2\bigl(1 + |r_m|^2 + 2|r_m|\cos 2kx \bigr)  }
  \end{split}
\end{equation}
and we get the expected relation $R(\Delta=0,x) + T(\Delta=0,x) = 1$.

\subsubsection{Cavity Linewidth and Finesse}
Here, and similarly as in the next section detailing the quantum description of the MATE system, we can derive the cavity linewidth and finesse considering two different approaches. The firts one, best for high membrane reflectivities, consists in considering two coupled subcavities, each with their own low linewidth/ high finesse, coupled by photon transmission through the membrane. The second one, appropriate for MATE geometries and low membrane reflectivities, consists in considering the whole cavity as a single optical mode, where the back short cavity act as an effective mirror with position dependent reflectivity. We will derive both and compare them. \\

To derive the position-dependent cavity linewidth $\kappa(x)$ and finesse $\mathcal{F}(x)$ in the two cavity approach, we once again resort to the Sankey \textit{et al.} method \cite{sankey_strong_2010}. The total energy stored in the cavity is given by
\begin{equation}
  E = \dfrac{2(L_1+x)}{c} P_L + \dfrac{2(L_2 - x)}{c} P_R
\end{equation}
and the rate at which energy leaves the cavity 
\begin{equation}
  \partial_t E = - |t_1|^2 P_L  -|t_2|^2 P_R = - \kappa(x) E.
\end{equation}
such that the cavity energy decay rate is given by
\begin{equation}
  \begin{split}
  \kappa(x) & = - \dfrac{\partial_t E}{E} = \dfrac{c (|t_1|^2 + |t_2|^2 P_R/P_L)}{2(L_1+x) + 2(L_2 - x) P_R/P_L} \\
  & = \dfrac{c|t_1|^2 (1-|r_m|^2) + c|t_2|^2 \bigl(1 + |r_m|^2 + 2|r_m|\cos 2kx \bigr)}{2(L_1+x)(1-|r_m|^2) + 2(L_2 - x)\bigl(1 + |r_m|^2 + 2|r_m|\cos 2kx \bigr)}.
  \end{split}
  \label{eq:kappa_mate}
\end{equation}
We can then derive the cavity finesse as
\begin{equation}
  \mathcal{F}(x) = \dfrac{\pi c}{ L \kappa(x)}.
\end{equation}

\color{red} add comparison with effective mirror approach \color{black}

In the single cavity, effective mirror approach, the membrane acts as a position dependent mirror modifying the cavity resonance frequency as derived above. The back cavity then acts as an effective mirror with complex reflection coefficient given by 
\begin{equation}
r_{\rm eff}(x) = r_m + \dfrac{t_m^2 r_2 e^{2ik(L_2 - x)}}{1 - r_m r_2 e^{2ik(L_2 - x)}}
\end{equation}
such that the resulting cavity finesse is given by 
\begin{equation}
  \mathcal{F}(x) \sim \dfrac{2\pi }{|t_1|^2 + (1- |r_{\rm eff}(x)|^2)}
\end{equation} and the associated linewidth $\kappa(x) = \pi c / L \mathcal{F}(x)$.\\



\subsection{Quantum Description}
We now turn to the quantum description of the membrane based optomechanical system. A question that naturally arises is how to describe best this three-mirror cavity quantum mechanically: should we consider two independent optical modes in each subcavity, coupled by photon tunneling through the membrane? Or should we consider the whole cavity as a single optical mode, whose resonance frequency is modified by the membrane position (and given above)? 


\subsubsection{Two-Cavity Model}
We start by looking at the two-cavity model. Using the same tools as in Chapter I, we can derive the QLE of a membrane based optomechanical system. The membrane position now turns into an operator such that $\hat x \propto \hat c + \hat c^{\dagger} $ with $\hat c$ the mechanical annihilation operator as in the previous section. As seen above, the membrane position modifies the resonance frequencies of the two subcavities, such that they both depend on the membrane position as \(\omega_L(x)\) and \(\omega_R(x)\) but with inverse trend: when one cavity shortens and its FSR increases, the other lengthens and its FSR decreases. To first order, we can linearize the resonance frequencies as
\begin{equation}
\omega_L(\hat x) \simeq \omega_{L} + G_L \hat x, \qquad \omega_R(\hat x) \simeq \omega_{R} + G_R \hat x,
\end{equation}
with \(\omega_{L,R}\) the unperturbed resonance frequencies of the subcavities and \(G_L = \omega_{L}/L_1\) and \(G_R = - \omega_{R}/L_2\) their respective optomechanical couplings. 
The whole system features a network of optical modes varying linearly with the membrane position, coupled by the membrane transmission. \\ 

In V. Dumont's PhD work, quadratic points (where \(G^{(1)} = 0\) and \(G^{(2)} \neq 0\)) were the centerfold of the study, in the high membrane reflectivity regime \cite{dumont_cavity_2017}. It was then sufficient to consider two optical modes coupled by photon tunneling through the membrane. \\ 

However, in our case, we focus on the sole dispersive coupling regime in the MATE configuration, and we additionally consider a low membrane reflectivity. The optimal point to do so is when the first long cavity is on resonance, and when the short one is anti-resonant. With a lowered reflectivity, the coupling between subcavity modes increases, leading to larger mode splittings at the avoided crossings, until the two subcavities are fully hybridized into new cavity modes spanning both subcavities \cite{thompson_strong_2008, thompson_coupling_2013}. \\ 

The short cavity being precisely at an anti-node, is is equally probable for the tunneled photons from the long cavity to populate two short cavity modes on either side of the anti-node. 
We then need to describe the system by a single long cavity mode coupled to two short cavity modes, as illustrated in figure \ref{fig:mate_modes}. We introduce the annihilation operators \(\hat a_L\) for the long cavity mode, and \(\hat a_{R+}\) and \(\hat a_{R-}\) for the two short cavity modes on either side of the anti-node. The Hamiltonian of this system can then be written as
\begin{align}
\hat H
&= \hbar (\omega_L + G_L x ) \hat a_L^\dagger \hat a_L^{\vphantom{\dagger}}
 + \hbar (\omega_{R-} -  G_R x ) \hat a_{R-}^\dagger \hat a_{R-}^{\vphantom{\dagger}}
 + \hbar (\omega_{R+} -  G_R x ) \hat a_{R+}^\dagger \hat a_{R+}^{\vphantom{\dagger}}
\tag*{(= $\hat H_\gamma$)}
\\[2mm]
& \quad + \hbar \Omega_m \hat c^\dagger \hat c^{\vphantom{\dagger}}
\tag*{(= $\hat H_m$)}
\\[2mm]
&\quad + \hbar G_L \hat a_L^\dagger \hat a_L^{\vphantom{\dagger}} \, \delta \hat x
 - \hbar G_R (\hat a_{R+}^\dagger \hat a_{R+}^{\vphantom{\dagger}}
              +\hat a_{R-}^\dagger \hat a_{R-}^{\vphantom{\dagger}})\, \delta \hat x
\tag*{(= $\hat H_{\rm OM}$)}
\\[2mm]
&\quad - \hbar J \!\left[
    \hat a_L^\dagger(\hat a_{R+}^{\vphantom{\dagger}}+\hat a_{R-}^{\vphantom{\dagger}})
   +(\hat a_{R+}^\dagger+\hat a_{R-}^\dagger)\hat a_L^{\vphantom{\dagger}}
\right]
\tag*{(= $\hat H_{\rm tun}$)}
\end{align}
where \(J = c|t_m|/2\sqrt{L_1 L_2}\) is the photon tunneling rate through the membrane \cite{thompson_strong_2008}, and where we linearized the position as before as $\hat x = x + \delta \hat x$. The first line describes the free evolution of the subcavity modes, the second one the mechanical resonator, the third the optomechanical interaction between the membrane position and the subcavity modes, and the last the photon tunneling through the membrane. As before, the commutation relations are given by 
\begin{equation*}
  [\hat a_L^{\vphantom{\dagger}}, \hat a_L^\dagger] = [\hat a_{R \pm}^{\vphantom{\dagger}}  , \hat a_{R\pm}^\dagger] = [\hat c, \hat c^\dagger] = 1 \quad \text{and} \quad [\hat a_L^{\vphantom{\dagger}}, \hat a_{R\pm}^{\vphantom{\dagger}}] = [\hat a_L^{\vphantom{\dagger}},\hat a_{R\pm}^{\dagger}]  = 0
\end{equation*}
We will only consider the photonic part of the Hamiltonian, as to put it in matrix form such that we can diagonalize it and work in the basis of the new eigenmodes. Furthermore, we go the frame rotating at frequency $\omega_0 = \omega_{R-} = \omega_L$ i.e. when the long cavity mode is degenerate with the left short cavity mode, such that the photonic Hamiltonian becomes
\begin{equation}
 \hat H_\gamma = \hbar  G_L x  \hat a_L^\dagger \hat a_L^{\vphantom{\dagger}}
 - \hbar  G_R \bigg(x+ \dfrac{\lambda}{4}\bigg)  \hat a_{R-}^\dagger \hat a_{R-}^{\vphantom{\dagger}}
 + \hbar \Bigg(\omega_{FSR} -  G_R \bigg(x- \dfrac{\lambda}{4}\bigg) \Bigg) \hat a_{R+}^\dagger \hat a_{R+}^{\vphantom{\dagger}}\end{equation}
and we can rewrite both the photonic and tunneling hamiltonian i.e. the photonic manifold in matrix form as
\begin{equation} 
  \hat H_\gamma + \hat H_{\rm tun} = \hbar
  \begin{pmatrix}
    \hat a_L^\dagger & \hat a_{R-}^\dagger & \hat a_{R+}^\dagger
  \end{pmatrix}
  \mathbf{M}
  \begin{pmatrix}
    \hat a_L^{\vphantom{\dagger}} \\
    \hat a_{R-}^{\vphantom{\dagger}} \\
    \hat a_{R+}^{\vphantom{\dagger}}
  \end{pmatrix}
\end{equation}
with 
\begin{equation*}
  \mathbf{M} =
  \begin{pmatrix}
    G_L x & -J & -J \\
    -J & - G_R (x + \lambda/4) & 0 \\
    -J & 0 & \omega_{FSR} -  G_R (x - \lambda/4)
  \end{pmatrix}.
\end{equation*}
One could then diagonalize this \(3 \times 3\) matrix to get the new eigenmodes of the system, and rewrite the optomechanical interaction in this new basis. In the limit where the membrane transmittivity is high such that \( |t_m| \sim 1\) and \( |r_m| \ll 1\), the tunneling rate \(J\) becomes much larger than both optomechanical couplings \(\tilde G_{L,R}\, x\) and the free spectral range \(\omega_{FSR}\). The cumberstone expression of the eigenmodes is not displayed here, but is equivalent to considering an system's eigenstate described by a annihilation operator \(\hat a \) with optomechanical coupling \( G^{(1)}(x)\) and decay rate \(\kappa(x)\) as derived in the previous section. The system's Hamiltonian can then be written as
\begin{equation}
\hat{H} = \hbar \omega_c(x=\lambda/4) \hat a^\dagger \hat{a} + \hbar \Omega_m \hat c^\dagger \hat c + \hbar G^{(1)}(x) \hat a^\dagger \hat a \,  \hat x
\end{equation}

\begin{figure}
  \centering
  \includegraphics[width=\textwidth]{./chap3/fig/models.pdf}
  \caption{0.01, 0.99 }
  \label{fig:mate_modes}
\end{figure}

\subsubsection{Single Mode Model}
If the membrane is more transmissive than reflective, one could ask if the system could be described by a single optical mode, whose resonance frequency is weakly perturbed by the membrane position as derived in the previous section. In this case, the Hamiltonian of the system reads
\begin{equation}
\hat{H} = \hbar \omega_c(x) \hat a^\dagger \hat{a} + \hbar \Omega_m \hat c^\dagger \hat c
\end{equation}
where \(\omega_c(x)\) is given by the expression derived above. This description is then matching the two mode model in the limit of highly reflective membranes as seen in figure \ref{fig:models}, as well as in the limit of low reflectivity membranes where the subcavities are fully hybridized. 

\subsubsection{Comparison to Single Mode Model}
Since we are ineterested in the dispersive coupling regime in the MATE configuration with a low reflectivity membrane, such that we will operate the system where the linear dispersive coupling is dominant over quadratic dispersive coupling and dissipative coupling, we need to compare which model is best suited to describe the system. \\

Obviously the two mode model breaks down in the limit of low reflectivity membranes where the subcavities are fully hybridized, and the single mode model is then more appropriate. In the opposite limit of highly reflective membranes, both models converge to the same description as seen above. Regarding the radiation pressure force acting on the membrane, in the two mode model, the radiation pressure force is given by the sum of the forces exerted by each subcavity mode as \( \hat F_{rp} = - \hbar G_L \hat a_L^\dagger \hat a_L + \hbar G_R (\hat a_{R+}^\dagger \hat a_{R+} + \hat a_{R-}^\dagger \hat a_{R-})\). In the single mode model, the radiation pressure force is given by \( \hat F_{rp} = - \hbar \partial \omega_c(x)/\partial x \, \hat a^\dagger \hat a\). In the limit of highly reflective membranes, the two mode model radiation pressure force is then more appropriate since the optical mode is split in two subcavity modes, each exerting a force on the membrane. In the opposite limit of low reflectivity membranes where the subcavities are fully hybridized and where we focus on the dispersive coupling regime, it would genuinely be of no interest when studying radiation pressure effects, and a relevant description of the radiation pressure force is tricky to derive (because there are actually photons in both subcavities, but the model breaks down). In our middle ground case of moderately reflective membranes, we will assume the optical mode is mostly localized in the long cavity mode such that the single mode description is valid, and that the radiation pressure force is given by the derivative of the cavity resonance frequency with respect to the membrane position as seen in the textbook case of a single mirror cavity. The same QLEs as in the previous chapter can then be derived and used in our case. 










\newpage 