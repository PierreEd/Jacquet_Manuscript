%!TeX encoding = UTF-8
% Metadata for coverpages
\thesisname{Thèse de doctorat}
\gradename{docteur}
\univ{de Sorbonne Université}
\logos{./fig/Sciences_SU}{}{./fig/Logo_LKB}                   % Un à trois logo. Le 1er est celui de \univ
\specialite{Physique}
\ecoledoctnum{564}
\ecoledoct{Physique en Île-de-France}
\title{Squeezed light optomechanics: Theory and Experiments}
\titleen{Optomechanics and squeezed light}

% si nécessaire, pour les métadonnées
%\titlemeta{La laine des Dupondt au Pays de l'or noir} 
%\titlemetaen{Dupondt's wool in "Land of Black Gold" } 

\date{26 Février 2026}
\author{Pierre-Edouard Jacquet }
\advisor{Pierre-François Cohadon}
\atinstitution{à Sorbonne Université}
\atlab{{\Large au Laboratoire Kastler Brossel}}
% Jury as LaTeX Tabular. Pas de président avant soutenance								  
\jury{ %       
M${^\text{me}}$\ &  Agnès MAITRE \ & PU & INSP (SU) & Présidente         \\
M\ & Pierre VERLOT \ & MC &  LuMIn (UPSaclay) & Rapporteur        \\
M\ & Jean-Pierre ZENDRI \  & DR &  INFN (UPadova) & Rapporteur \\
M${^\text{me}}$\ & Sara DUCCI & PU &  MPQ (UPCité) & Examinatrice \\
M\ & Jack HARRIS  & PU &  Yale University & Examinateur   \\
M\  & Pierre-François COHADON & MC & LKB (ENS) & Directeur }

\makeatletter
\@ifpackageloaded{thcover-psl}{\logos{PSL}{}}{\relax}
\makeatother

\resume{Cette thèse de doctorat étudie les limites quantiques à l'oeuvre dans la détection interferométrique de petits déplacements mécaniques, et comment surmonter ces dernières en utilisant de la lumière comprimée. Dans un premier temps, le travail traite théoriquement la faisabilité d'une lumière comprimée dépendante en fréquence via l'utilisation d'une cavité de filtrage/rotation en sortie d'un amplificateur paramétrique optique. Dans un second temps, il se concentre sur la faisabilité d’une détection sous la limite quantique standard (SQL) en détaillant deux expériences réalisées. La première implémente un système « membrane at the edge » (MATE) basé sur une membrane en nitrure de silicium à haut facteur de qualité mécanique monté dans une cavité Fabry Perot de grand finesse. La deuxième expérience présente une source de lumière comprimée indépendante de la fréquence. Ces deux expériences sont pilotées à l'aide de locks optiques digitaux basé sur FPGA et développé au laboratoire, permettant un fonctionnement stable dans les conditions requises pour des mesures à la limite quantique. }
\motscles{Optomecanique, Lumière comprimée, Cavité de grande Finesse, Interferométrie, Bruit thermique, Bruit de grenaille quantique, Resonateur de grand facteur de Qualité, Interféromètres pour la detection d'ondes gravitationnelles, Bruit de pression de radiation quantique }

% si nécessaire, pour les métadonnées
%\titlemetaen{Dupondt's wool in "Land of Black Gold" } 

\abstract{}

\keywords{Optomechanics, Squeezing, High-Finesse cavity, Interferometry, Thermal Noise, Quantum Shot Noise, High-Q Resonator, Gravitational wave Interferometer,Quantum Radiation Pressure Noise}
%\par\hfill (1700 chars max, spaces included)