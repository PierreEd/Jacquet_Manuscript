\newcommand{\adag}[1]{\hat{a}_{#1}^\dagger}
\newcommand{\aop}[1]{\hat{a}_{#1\vphantom{\dagger}}}
\chapter{Theory: Background} \label{chap:theory}
This chapter covers the elementary concepts required to describe a membrane based optomechanical system in a quantum regime. We will first recall basics on optics and field quantization as well as describe the coherent and squeezed light fields, to then turn to the more specific frequency-dependent squeezed light fields. Secondly, we will cover the mathematical description of an optomechanical cavity interacting with a generic coherent optical field. Finally, we will describe optical detection schemes and what they measure in terms of field quadratures.  
\minitoc
\newpage
\section{Optics}
\subsection{Spatial Modes}
The spatial structure of an electromagnetic wave propagating along the $z$-axis can be described by a set of well-defined transverse modes, which are solutions of the paraxial Helmholtz equation \cite{siegman1986lasers}. The most fundamental solution is the Gaussian mode, whose electric field amplitude reads
\begin{equation}
E(\mathbf r) = E_0 \, \frac{w_0}{w(z)} 
\exp\!\left(-\frac{x^2+y^2}{w^2(z)}\right) 
\exp\!\left[-i\!\left(kz + \frac{k(x^2+y^2)}{2R(z)} - \psi(z)\right)\right],
\end{equation}
where $\mathbf r = (x, y, z)$, $E_0$ is the field amplitude at the beam waist, $k = 2\pi/\lambda$ the wavenumber, and $\lambda$ the optical wavelength. The various quantities introduced above are defined as
\begin{equation*}
  \begin{split}
w(z) \equiv w_0 \sqrt{1+(z/z_R)^2}, \quad & \quad
z_R \equiv \pi w_0^2/\lambda, \\
R(z) \equiv z\left[1+(z_R/z)^2\right], \quad &\quad
\psi(z) \equiv \arctan(z/z_R),
  \end{split}
\end{equation*}
with $w_0$ the waist, $z_R$ the Rayleigh range, $R(z)$ the wavefront curvature, and $\psi(z)$ the Gouy phase. 
A compact expression of the Gaussian envelope is written as
\begin{equation}
E(\mathbf r) \;=\; E_0 \,\frac{i z_R}{q(z)} \,
\exp\!\left(-\,\frac{i k (x^2+y^2)}{2\,q(z)}\right) e^{-i k z}  \quad  \text{with} \quad q(z) \equiv z + i z_R,
\label{II.1}
\end{equation}
where we defined the complex beam parameter $q(z)$. 
Beyond the fundamental Gaussian mode, more general solutions of the paraxial equation can be constructed. In Cartesian coordinates, these are the Hermite--Gaussian modes $\mathrm{TEM}_{mn}$, given by
\begin{multline}
E_{mn}(\mathbf{r}) = E_0 \, \frac{w_0}{w(z)} \,
H_m\!\left(\frac{\sqrt{2}x}{w(z)}\right) 
H_n\!\left(\frac{\sqrt{2}y}{w(z)}\right) 
\exp\!\left(-\frac{x^2+y^2}{w^2(z)}\right) \\
\times \exp\!\left[-i\!\left(kz + \frac{k(x^2+y^2)}{2R(z)} - (m+n+1)\psi(z)\right)\right],
\end{multline}
where $H_m, H_n$ are Hermite polynomials. 

\begin{figure}
\centering
\includegraphics[width=\textwidth]{./chap2/fig/gaussianbeam.pdf}
\caption{ Gaussian beam characteristics: (a) Intensity profile of the fundamnental Gaussian mode at different positions along the propagation axis $z$. The various quantities introduced in the text are indicated. (b) Transverse intensity profiles of the first few Hermite--Gaussian modes $\mathrm{TEM}_{mn}$. These modes form an orthonormal basis to describe the spatial structure of paraxial beams: any realistic beam can be decomposed as a superposition of these modes. } 
\end{figure}

\subsection{Quantum Description}
\subsubsection{Quantised Electromagnetic Field }
% \the\textwidth
We will consider both cases of a finite quantisation volume $V$ and an infinite volume: when dealing with cavity modes, we will use the finite volume description, while for propagating modes carrying sideband correlations we will use the infinite volume description, under the two-photon formalism prescription. Both description are linked through the input-output formalism introduced later on. \\

We first consider the quantised electromagnetic field in a volume $V$. The electric field operator can be written as \cite{knightoptics}
\begin{equation}
\hat{\mathbf{E}}(\mathbf{r}, t) 
= i \sum_{\ell} \mathcal{E}_\ell 
\left[ \hat{a}^{\vphantom{\dagger}}_{\ell}\,\mathbf{f}_{\ell}(\mathbf{r})\,e^{-i\omega_{\ell} t} 
- \hat{a}_{\ell}^\dagger\,\mathbf{f}_{\ell}^*(\mathbf{r})\,e^{+i\omega_{\ell} t} \right],
\end{equation}
where   $\mathcal{E}_\ell = \sqrt{\hbar \omega_\ell / 2 \varepsilon_0 V}$ is the field amplitude per photon in mode $\ell$, $\hbar$ is the reduced Planck constant, $\omega_\ell$ is the angular frequency of mode $\ell$, and $\varepsilon_0$ is the vacuum permittivity. The spatial mode functions $\mathbf{f}_{\ell}(\mathbf{r})$ form an orthonormal basis in $V$ according to  
\begin{equation*}
\int_V d^3r\; \mathbf{f}_{\ell}^*(\mathbf{r}) \cdot \mathbf{f}_{\ell'}(\mathbf{r}) 
= \delta_{\ell \ell'} , \quad \mathbf{f}_{\ell}(\mathbf{r}) \propto E_{mn}(\mathbf{r}) \boldsymbol{\epsilon}_x
\end{equation*}
where we assumed a linearly-polarized field along the $x$-axis, with $\boldsymbol{\epsilon}_x$ the corresponding unit vector. The index $\ell = (m,n)$ then labels the different spatial modes, the Hermite--Gaussian modes in our case. \\ 

In the limit of an infinite quantisation volume $V \to \infty$, the discrete mode index $\ell$ becomes a continuous variable i.e. the mode spacing becomes infinitesimal. As detailed in the Appendix A, we now need to consider a continuum of modes with annihilation operator $\hat a[\omega]$ labeled by their angular frequency $\omega$. To come down to the two-photon formalism, we make the following assumptions 
\begin{itemize}
    \item we consider frequencies $\omega = \omega_0 \pm \Omega$ centered around a carrier frequency $\omega_0$, with $\Omega  \in [-2 \pi B,+2 \pi B] \ll \omega_0$ where $B$ is the bandwidth. This is valid since the bandwidth $B$, generally up to tens of GHz, is small compared to $\omega_0/2\pi$ wich is hundreds of THz for optical frequencies. 
    \item we consider a single spatial mode, i.e. we drop the spatial dependence of the field and consider only one transverse mode function $\mathbf{f}(\mathbf{r})$, the fundamental Gaussian mode imposed by the laser source and/or spatial filtering elements. This is valid since the spatial envelope of the beam does not vary significantly over the considered bandwidth. The spatial mode function is then factored out of the integrals over frequency.
    \item we consider the electric field operator to only be dependent on time $t$, having projected the field onto the transverse mode function and integrated over the transverse plane, as well as setting the propagation coordinate $z=0$ for simplicity.
\end{itemize}

Upon these assumptions, the electric field operator reduces to a time-dependent operator (Heisenberg picture) expressed as \cite{Oelker2016PhD,Whittle2023PhD} 
\begin{equation}
  \begin{split}
   \hat{E}(t) =  \mathcal{E}_0 \bigg[ \cos ( \omega_0 \, t - \frac{\pi}{2} )   \int^{\infty}_{-\infty} \dfrac{d\Omega}{2\pi} \, \ & (\hat{a}_+^{\vphantom{\dagger}} + \, \hat{a}_-^{\dagger})\,e^{-i\Omega t}   \\ 
    +  \sin ( \omega_0 \, t - \frac{\pi}{2} )   \int^{\infty}_{-\infty} \dfrac{d\Omega}{2\pi} \, & i(\hat{a}_-^{\dagger} - \hat{a}_+^{\vphantom{\dagger}}  \, )\,e^{-i\Omega t} \bigg] 
  \end{split}
\end{equation}
where we defined the sideband annihilation operators as $\hat{a}_+[\Omega] \equiv \hat{a}[\omega_0 + \Omega]$ and $\hat{a}_-[\Omega] \equiv \hat{a}[\omega_0 - \Omega]$. The field amplitude per photon at the carrier frequency is given by $\mathcal{E}_0 = \sqrt{\hbar \omega_0/2 \varepsilon_0 c A}$, where $A$ is the effective cross-sectional area of the beam and $c$ the speed of light in vacuum. The explicit relationship between the discrete mode operator $\hat a_\ell^{\vphantom{\dagger}}$ and the continuous mode operator $\hat a[\omega]$ (and thus their hermitian conjugate) is given in the Appendix A. \\ 

\noindent \textbf{Note:} Although the electric field operator is written in the Heisenberg picture, the annihilation and creation operators $\hat a[\omega]$ and $\hat a[\omega]^\dagger$ are Schrodinger like operators, i.e. time independent operators. This is because we already factored out the time dependence $e^{-i\omega t}$ associated to each frequency mode when writing our annihilation/creation operators. As seen later on, the time dependence of the field operator defined through the Fourier transform arises from the superposition of many frequency modes, leading to beating at frequency $\Omega$. This is the heart of the two-photon formalism, where a time-dependent field Heinsenberg like operator is built from the superposition of Schrödinger like annihilation/creation operators at different frequencies. In the litterature, the Schrödinger like operators are sometimes written as $\hat a_\omega$ to draw a parallel between the discrete mode case $\hat a_\ell$, but we will stick  to the $\hat a[\omega]$ notation to avoid confusion with time dependent operators. \\ 

When writing annihilation operators, we will often drop the frequency dependence to lighten the notation, but it is implicit that they depend on frequency i.e. $\hat a \equiv \hat a[\omega]$, such that it applies to both sideband operators $\hat a_+$ and $\hat a_-$. 

\subsubsection{Commutation Relations}
As demonstrated in the Appendix, the continuous annihilation and creation operators satisfy the following commutation relations: 
\[
[\hat{a}^{\vphantom{\dagger}}[\omega], \hat{a}^\dagger[\omega']] = 2\pi \, \delta(\omega - \omega') \,, \quad
[\hat{a}^{\vphantom{\dagger}}[\omega], \hat{a}^{\vphantom{\dagger}}[\omega']] = 0, \quad [\hat{a}^\dagger[\omega], \hat{a}^\dagger[\omega']] = 0.
\]
such that the sideband operators satisfy
\[
[\hat{a}^{\vphantom{\dagger}}_\pm[\Omega], \hat{a}^\dagger_\pm[\Omega']] = 2\pi \, \delta(\Omega - \Omega') \,,  \quad
[\hat{a}^{\vphantom{\dagger}}_\pm[\Omega], \hat{a}^{\vphantom{\dagger}}_\pm[\Omega']] = 0, \quad [\hat{a}^\dagger_\pm[\Omega], \hat{a}^\dagger_\pm[\Omega']] = 0.
\]

\subsubsection{Quadrature Operators}
We describe the phase-space properties of a field mode using hermitian quadrature operators. These are linear combinations of the annihilation and creation operators that correspond to measurable observables of the electromagnetic field. Here again, we won't write explicitly the frequency dependence, but it is implicit in the following i.e. $\mathbf{\hat{u}} \equiv \mathbf{\hat{u}}[\Omega] $ and $\mathbf{\hat{a}} \equiv \mathbf{\hat{a}}[\Omega] $.
The two most common quadratures are defined as follows:
\begin{equation}
\mathbf{\hat{u}}\;\equiv\;
\begin{pmatrix}\hat a_1\\[2pt]\hat a_2\end{pmatrix}
=\mathbf\Gamma \, \mathbf{\hat{a}}  \label{II.2}
\qquad \text{with} \quad
\mathbf \Gamma \equiv
\begin{pmatrix}
1 & 1 \\
-i & i
\end{pmatrix}
\quad
\text{and} \quad
\mathbf{\hat{a}}\;\equiv\;
\begin{pmatrix}\hat a_+\\ \hat a_-^\dagger\end{pmatrix}
\end{equation}
where we defined the field vector $\mathbf{\hat{a}}$ and the transfer matrix $\mathbf \Gamma$, later used to switch from \textit{one-photon} to \textit{two-photon} description of optical elements. Mathematically, the integral bounds are the detection bandwidth, but we can safely extend it to infinity. The electric field operator can then be recasted as 
\begin{equation}
\hat{E}(t) = \, \mathcal{E}_0 \,\left[ \cos ( \omega_0 \, t - \frac{\pi}{2} ) \int_  {-\infty}^{\infty} \dfrac{d\Omega}{2\pi}  \,\hat a_1[\Omega] \,e^{-i\Omega t} + \, \sin ( \omega_0 \, t - \frac{\pi}{2} )  \int_ {-\infty}^{\infty} \dfrac{d\Omega}{2\pi}  \,\hat a_2[\Omega] \,e^{-i\Omega t} \right].
\end{equation}
where it is now explicit that the electric field features two orthogonal components oscillating at the carrier frequency $\omega_0$, with amplitudes given by the quadrature operators $\hat a_1$ and $\hat a_2$.\\

\subsubsection{Fourier Transform}
We now come to the aforementioned building of a time-dependent field operator from the superposition of many frequency modes. This is done through the Fourier transform defined as
\begin{equation}
  \begin{split}
    \mathbf{\hat{a}}(t) &=\int_{-\infty}^{+\infty}  \dfrac{d\Omega}{2\pi} \, e^{-i\Omega t} \, \mathbf{\hat{a}}[\Omega] \\
    \mathbf{\hat{a}}[\Omega] &=\int_{-\infty}^{+\infty}  dt \, e^{i\Omega t} \, \mathbf{\hat{a}}(t)
  \end{split}
\end{equation}
In this definition, a notable property is that the hermitian conjugate in the time domain translates into a frequency inversion in the Fourier domain:
\begin{equation}
   \Big[\hat{a}(t)\Big]^{\dagger} = \hat{a}^\dagger(t),  \quad
   \Big[\hat{a}_+\Big]^{\dagger}  = \hat{a}^\dagger_-.
\end{equation}
It then follows that the quadrature operators in the time domain are effectively Hermitian operators, as expected for observables, while the frequency domain quadrature operators satisfy
\begin{equation}
   \hat{a}^\dagger_1[\Omega]  = \hat{a}_1[-\Omega], \quad
   \hat{a}^\dagger_2[\Omega]  = \hat{a}_2[-\Omega]. 
\end{equation}

\subsubsection{Commutation Relations in vector form}

The matrix form commutator in both time and frequency space reads
\begin{equation}
[\mathbf{\hat{a}}, \mathbf{\hat{a}}^{\dagger}] = \mathbf{\sigma_z} \times 
\begin{cases} 
\delta (t-t')  \\
2\pi \delta (\Omega - \Omega') .
\end{cases}
\end{equation}
with $\sigma_z$ the Pauli Z matrix, and where it is implicit that we evaluate both at different frequencies or times respectively.  
An arbitrary rotated quadrature pair is obtained by
\begin{equation}
\mathbf{\hat{u}_\phi}\;\equiv\; \mathbf R(\phi)\,\mathbf{\hat{u}}
= \mathbf R(\phi)\,\mathbf\Gamma \,\mathbf{\hat{a}}
\qquad
\text{with} \quad
 \mathbf R(\phi)\equiv
\begin{pmatrix}
\cos\phi & \sin\phi \\
-\sin\phi & \cos\phi \label{II.4}
\end{pmatrix}.
\end{equation}
and where we identify the useful identity 
\begin{equation*}
\mathbf R(\phi)\, \mathbf \Gamma = \begin{pmatrix}
e^{-i\phi} & e^{i\phi} \\
-i e^{-i\phi} & i e^{i\phi}
\end{pmatrix}.
\end{equation*}
The commutators of the rotated quadrature operators read
\begin{equation}
[\mathbf{\hat{u}}^{\vphantom{\dagger}}_\phi , \mathbf{\hat{u}}_\phi^{\dagger}] = 2i\,\mathbf J  \begin{cases} 
\delta (t-t')  \\
2\pi \delta (\Omega - \Omega') .
\end{cases} 
\qquad \text{with} \quad
\mathbf{J} \equiv
\begin{pmatrix}
0 & 1 \\
-1 & 0
\end{pmatrix}.
\end{equation}
This identity would not be true had we considered large sideband frequencies $\Omega \sim \omega_0$ i.e. it would feature corrections in all $\mathbf{J}$ terms, including diagonal. 

\subsubsection{Linearization of the optical field}

Let's now consider a quantum state living in this continuous mode space $| \psi \rangle$.  We can always linearize the field operators around their mean value, which is particularly useful when dealing with intense fields featuring small quantum fluctuations around a large classical amplitude. This is the case for coherent and squeezed states, which are introduced right below. The annihilation operator is then be decomposed as
\begin{equation}
    \hat{a}  = \bar{\alpha} + \delta\hat{a} \\
\label{II.8}
\end{equation}
where \( \bar{\alpha} = \langle \psi |\hat{a} |\psi \rangle \in \mathbb{C}\) is the mean complex amplitude of the quantum state, and \(\delta\hat{a}\) represents quantum fluctuations with \(\langle \psi | \delta\hat{a}| \psi \rangle = 0\). Note this decomposition is valid for any quantum state, including coherent and squeezed states. We note $\bar{\alpha}$ to distinguish it from the complex amplitude $\alpha$ of a coherent state introduced below, which is a specific case of this decomposition. The field vector is then expressed as
\begin{equation}
\mathbf{\hat{a}} =  \begin{pmatrix} \bar{\alpha}_+  \\ \bar{\alpha}_-^*  \end{pmatrix} + \begin{pmatrix} \delta\hat{a}_+ \\ \delta\hat{a}_-^\dagger \end{pmatrix} =  \mathbf{\bar{a}} + \mathbf{\delta \hat{a}}
\end{equation}
and it then follows that the quadrature operators can also be expressed as
\begin{equation}
    \mathbf{\hat{u}_\phi}  = \mathbf{R}(\phi) \, \mathbf\Gamma  \,(\mathbf{\bar{a}} + \mathbf{\delta \hat{a}})  = \mathbf{\bar{u}_\phi}  + \mathbf{\delta \hat{u}_\phi}. 
\end{equation}

For the vacuum state \(|0\rangle\), we have \(\bar{\alpha}_\pm = 0\) and thus \(\mathbf{\hat{a}} = \mathbf{\delta \hat{a}}\). Since we will always consider fluctuations around the mean value, we will systematically use the notation \(\delta\hat{a}\) to refer to the annihilation operator, unless specified otherwise, as well as assume the vacuum state as the reference when we write average values as \(\langle \cdot \rangle \equiv \langle 0 | \cdot | 0 \rangle\). All the above definitions and properties thus apply to the fluctuation operators (commutation relations, Fourier transforms, etc.).

\subsubsection{Amplitude and Phase Quadratures}
Considering the mean field amplitude \(\bar{\alpha} = |\bar{\alpha}| e^{i\bar{\varphi}}\), we will often refer to the amplitude and phase quadratures, defined respectively as the quadratures at angles \(\phi = \bar{\varphi}\) and \(\phi = \bar{\varphi} + \pi/2\). As the angle \(\bar{\varphi}\) defines the mean field phase relative to a reference (e.g. a local oscillator), we will assume without loss of generality that \(\bar{\varphi} = 0\), such that the amplitude and phase quadratures correspond to \(\hat{a}_1\) and \(\hat{a}_2\) respectively. We will then relabel them as
\begin{equation}
    \delta \hat{p}\equiv \delta \hat{a}_{\phi=0} = \delta \hat{a}_1, \quad
    \delta \hat{q} \equiv \delta \hat{a}_{\phi=\pi/2} = \delta \hat{a}_2.
\end{equation}



\subsubsection{Noise Spectral Density Matrix}
A central concept in this thesis is the two-sided Noise Spectral Density matrix of the quadrature fluctuations, which characterizes the second-order statistical properties of the quantum state in the frequency domain. Namely, it describes the spectral distribution of the variances and covariances of the quadrature fluctuations. For a given quadrature angle \(\phi\), it is defined as
\begin{equation}
\begin{split}
  \mathbf{S}_{\phi} [\Omega]& =   \dfrac{1}{2} \int \dfrac{\delta \Omega^{'} }{2\pi} \langle \{\delta \mathbf{\hat{u}}_{\phi}\, , \delta \mathbf{\hat{u}}_{\phi}^{\dagger} \} \rangle \\
  & =  \dfrac{1}{2}    \int \dfrac{\delta \Omega^{'} }{2\pi} \mathbf{R}(\phi) \langle \{\delta \mathbf{\hat{u}}\, , \delta \mathbf{\hat{u}}^{\dagger} \} \rangle \mathbf{R}(-\phi)  \\
  & = \dfrac{1}{2}  \int \dfrac{\delta \Omega^{'} }{2\pi}  \mathbf{R}(\phi)  \begin{pmatrix}
  \langle \{\delta \hat{p}\, , \delta \hat{p}^\dagger \} \rangle & \langle \{ \delta \hat{p}\, , \delta \hat{q}^\dagger \}\rangle \\[4pt]
  \langle \{\delta \hat{q}\, , \delta \hat{p}^\dagger \}\rangle & \langle \{ \delta \hat{q}\, , \delta \hat{q}^\dagger \} \rangle 
  \end{pmatrix}  \mathbf{R}(-\phi)
\end{split}
\label{eq:Sa}
\end{equation}
where \(\{\hat{A},\hat{B}\} = \hat{A}\hat{B} + \hat{B}\hat{A}\) denotes the anticommutator, immplicitely evaluated at frequencies \(\Omega\) and \(\Omega^{'}\), and integrated over \(\Omega'\). The diagonal elements of the noise spectral density matrix correspond to the power spectral densities of the quadrature fluctuations, while the off-diagonal elements represent the cross-spectral densities between different quadratures. The noise spectral density matrix is a Hermitian matrix, reflecting the physical properties of the quantum state. We will particularly focus on the amplitude and phase quadrature noise spectral density matrix, obtained by setting \(\phi = 0\), and we will denote it as \(\mathbf{S}[\Omega]\equiv \mathbf{S}_{\phi=0}[\Omega]\). The subscripts will then denote whether we refer to the transmitted or reflected fields of an optical cavity, the output spectrum of a squeezer, etc. 
For completeness we introduce the single-sided noise spectral density matrix, defined as
\begin{equation}
  \mathbf{\bar S}_{\phi}[\Omega]= \dfrac{1}{2} (\mathbf{S}_{\phi}[\Omega]+ \mathbf{S}_{\phi}[-\Omega])
\end{equation}
such that the variance of a quadrature operator can be retrieved by integrating the single-sided noise spectral density over positive frequencies only (as one would with a real signal in a spectrum analyzer). 
A generalized version of the Heisenberg uncertainty relation can be expressed in terms of the noise spectral density matrix as
\begin{equation}
  \det \mathbf{S}_{\phi}[\Omega] \geq 1
\end{equation}
which sets a fundamental limit on the simultaneous knowledge of the quadrature fluctuations at a given frequency \(\Omega\). 
\subsubsection{Vacuum state}
For the vacuum state \(|0\rangle\), we derive the noise spectral density matrix using the commutation relations and the fact that \(\langle \delta \hat{a}  \rangle = 0\) (see Annexe A). The calculation yields
\begin{equation}
\mathbf{S}_{\text{vac}}[\Omega] =
\begin{pmatrix}
1 & 0 \\
0 & 1
\end{pmatrix}
\end{equation}
for any angle \(\phi\) and frequency \(\Omega\). This result indicates that the vacuum state has equal fluctuations in both quadratures, with no correlations between them, as expected for a minimum uncertainty state. The noise spectral density matrix of the vacuum state serves as a reference point for comparing other quantum states, such as coherent and squeezed states, which exhibit different fluctuation properties.

\subsubsection{Linear Optical Systems}
As we will develop further in the next section, the output fields of various optical systems can be expressed in a general linear form as
\begin{equation}
\mathbf{\delta \hat{u}_{\mathrm{out}}}
=
\mathbf{T}\, \mathbf{\delta \hat{u}_{\mathrm{in}}}
+ \mathbf{L}  \mathbf{\delta \hat{u}_{\mathrm{vac}}}.
\end{equation}
where $\mathbf{T}$ and $\mathbf{L}$ are $2\times 2$ frequency-dependent transfer matrices. The input and vacuum fields are assumed to be in the vacuum state, as well as being uncorrelated such that 
\begin{equation}
  \begin{split}
    \langle \mathbf{\delta \hat{u}_{\mathrm{in}}}  \mathbf{\delta \hat{u}_{\mathrm{in}}}^{\dagger} \rangle & = 2\pi \delta(\Omega+\Omega')\mathbf{S}_{\mathrm{in}}[\Omega]  \\
    \langle \mathbf{\delta \hat{u}_{\mathrm{vac}}}  \mathbf{\delta \hat{u}_{\mathrm{vac}}}^{\dagger} \rangle &=  2\pi \delta(\Omega+\Omega')\mathbf{1}  \\
    \langle \mathbf{\delta \hat{u}_{\mathrm{in}}}  \mathbf{\delta \hat{u}_{\mathrm{vac}}}^{\dagger} \rangle &= \mathbf{0}
  \end{split}
\end{equation}
Computing the noise spectra is then straightforward : 
\begin{equation}
  \mathbf{S}_{\mathrm{out}}[\Omega] = \mathbf{T} \mathbf{S}_{\mathrm{in}}\mathbf{T}^{\dagger} + \mathbf{L} \mathbf{L}^{\dagger} 
\end{equation}
with $\mathbf{S}_{\mathrm{vac}} = \mathbf{1}$ as seen above. For an arbitrary quadrature angle $\phi$, we simply rotate the transfer matrices as
\begin{equation*}
  \mathbf{T}_\phi = \mathbf{R}(\phi) \, \mathbf{T} \, \mathbf{R}(-\phi), \quad
  \mathbf{L}_\phi = \mathbf{R}(\phi) \, \mathbf{L} \, \mathbf{R}(-\phi)
\end{equation*}
such that
\begin{equation}
  \mathbf{S}_{\mathrm{out},\phi}[\Omega] = \mathbf{T}^{\vphantom{\dagger}}_\phi \mathbf{S}_{\mathrm{in},\phi}\mathbf{T}_\phi^{\dagger} + \mathbf{L}^{\vphantom{\dagger}}_\phi \mathbf{L}_\phi^{\dagger}
\end{equation}


\subsubsection{Graphical Representation of Gaussian States}
Gaussian states can actually be pictured in a 2D space, where the two axes correspond to the two quadratures $\hat a_1$ and $\hat a_2$. In the case where the mean phase is zero, these quadratures correspond to the amplitude and phase quadratures $\hat p$ and $\hat q$. The quantum state can then be represented as a 2D Gaussian distribution centered around the mean values of the quadratures, with the shape and orientation of the distribution characterized by the off-diagonal elements of the noise spectral density matrix. The uncertainties in the quadratures are represented by the widths of the Gaussian distribution along each axis, while correlations between the quadratures are represented by the tilt of the distribution. This graphical representation provides an intuitive way to visualize and understand the properties of Gaussian quantum states, such as coherent and squeezed states, in terms of their quadrature fluctuations and correlations. Examples of such representations are shown in Fig.\ref{fig:quantum_states}. 

\begin{figure}
\centering
\includegraphics[width=\textwidth]{./chap2/fig/quantum_states.pdf}
\caption{Phase-space representations of Gaussian quantum states. 2D cuts of the Wigner function in the quadrature plane $(a_1[\Omega], a_2[\Omega])$ at a given frequency.
(a) vacuum state: a circular Gaussian centered at the origin, featuring equal quantum fluctuations in both $a_1$ and $a_2$ quadratures.
(b) coherent state: a displaced circular Gaussian, showing a shift in phase space along an angle $\varphi$ with vacuum fluctuations. This corresponds to either the carrier ($\Omega=0$), or a sideband frequency with a non zero modulation.  
(c) vacuum squeezed state: an elliptical Gaussian centered at the origin, with reduced noise along a rotated quadrature and increased noise in the orthogonal direction. 
(d) bright squeezed state: an ellipse shifted away from the origin, combining anisotropic fluctuations and a nonzero mean amplitude. The displacement angle $\varphi$ and squeezing angle $\theta$ are independent. } 
\label{fig:quantum_states}
\end{figure}

\subsection{Coherent and Squeezed States}
We now turn to standard optical quantum states, in particular Gaussian states i.e.\ full positive in Wigner function representations such as coherent and squeezed states, that we will denote in braket notation as $|\alpha\rangle$ and $|\alpha,r, \theta\rangle $.
\subsection*{Coherent States:}
The monochromatic coherent state $\ket{\alpha}$ is an eigenstate of the annihilation operator:
\begin{equation}
\hat{a}_+\ket{\alpha} = \alpha \delta(\Omega) \ket{\alpha}
\label{II.14}
\end{equation}
where $\alpha = |\alpha| e^{i\bar{\varphi}}$ is a complex number representing the mean coherent amplitude. In this notation, the angle $\bar{\varphi}$ is the mean angle of the distribution, used to describe the relative phase to a reference (e.g. a local oscillator), as in Fig \ref{fig:quantum_states}. The $\hat{a}$ linear decomposition above (Eq.\ref{II.8}) then yields $\alpha = \bar{\alpha}$ for a coherent state. A generic multimode coherent state is generated by the displacement operator $\hat D(\alpha)$ such that 
\begin{equation}
\ket{\alpha} = \hat D(\alpha) \ket{0}
\end{equation}
where the general expression for the displacement operator acting on the vacuum state is given by
\begin{equation}
  \hat D (\alpha) = \exp\!\left(\int \dfrac{d\Omega}{2\pi} \left[ \alpha(\Omega) \hat{a}^\dagger_- - \alpha^*(\Omega) \hat{a}_+ \right] \right)
\end{equation}
which collapses to 
\begin{equation}
  \hat D (\alpha) = \exp\!\left( \dfrac{1}{2\pi} \left[ \alpha \hat{a}^\dagger[\omega_0] - \alpha^* \hat{a}[\omega_0] \right] \right)
\end{equation}
when defining a monochromatic coherent state $\alpha(\Omega) = \alpha \delta(\Omega)$, that is a coherent state at the carrier frequency only. 
Upon the action of this displacement operator, the sideband operator is transformed as
\begin{equation}
D^\dagger(\alpha) \hat{a}_+ D(\alpha) = \hat{a}_+ + \alpha \delta(\Omega)
\label{II.18}
\end{equation}  
such that we can verify the eigenvalue equation (Eq.\ref{II.14}) straightforwardly as 
\begin{equation*}
  \begin{split}
    \hat{a}_+ \ket{\alpha} & = D(\alpha) (D^\dagger(\alpha) \hat{a}_+ D(\alpha)) \ket{0} \\
    & = \hat D(\alpha) (\alpha \delta(\Omega) + \hat a_+ )  \ket{0} \\
    &= \alpha \delta(\Omega)  \ket{\alpha}
  \end{split}
\end{equation*}

\noindent \textbf{Expectation values}: Using the quadrature vector $\mathbf{\hat{u}}_\phi$ (Eq.\ref{II.4}), and the $\mathbf{R}\mathbf\Gamma$ identity, the expectation values in a coherent state are
\begin{equation}
\langle \hat{D}^\dagger\mathbf{\hat{\mathbf u}} _\phi \hat D\rangle
= \mathbf R(\phi)\,\langle \hat D^\dagger \mathbf{\hat{\mathbf u}} \hat D\rangle
=
2 \delta (\Omega)\begin{pmatrix}
\mathrm{Re}\big( |\alpha| e^{i(\bar{\varphi}-\phi)}\big) \\[2pt]
\mathrm{Im}\big(|\alpha| e^{i(\bar{\varphi}-\phi)}\big)
\end{pmatrix}
\end{equation}
such that the components reduce to $2\mathrm{Re} (\alpha)$ and  $2\mathrm{Im} (\alpha)$ if $\phi=0$, and to $2|\alpha|$ and $0$ if $\phi = \bar{\varphi}$, or equivalently if we set $\bar{\varphi} = \phi  = 0$ as mentioned earlier (such that $\mathbf{\hat u}$ corresponds to the amplitude and phase quadratures). We also notice the delta function at $\Omega = 0$, indicating that the coherent state has a non-zero mean field only at the carrier frequency. \\

\noindent \textbf{Spectrum:} For a coherent state, the fluctuations are identical to that of the vacuum state, seen directly from Eq.\ref{II.18}. Since the fluctuation operators are unchanged by the displacement, the noise spectral density matrix remains that of the vacuum:
\begin{equation}
\mathbf{S}_{\text{coh}}[\Omega] = \mathbf{S}_{\text{vac}} = 
\begin{pmatrix}
1 & 0 \\
0 & 1
\end{pmatrix}
\end{equation}
for any angle $\phi$ and frequency $\Omega$. Relating to the linear optical systems introduced, this is equivalent to having identity transfer matrices $\mathbf{T} = \mathbf{1}$ and $\mathbf{L} = \mathbf{0}$, such that no additional noise is added to the input vacuum fluctuations. Coherent states only differ from vacuum by their non-zero mean field amplitudes at the carrier frequency $\Omega = 0$, symbolized by the delta function in the expectation values above.

\subsection*{Squeezed States:}

Squeezed states $|\alpha, r, \theta\rangle $ are quantum Gaussian states of light in which the noise (variance) of one quadrature is reduced below the vacuum level, at the expense of increased noise in the conjugate quadrature. A generic squeezed state is characterized by three parameters: the displacement amplitude $\alpha$, the squeezing parameter $r$, and the squeezing angle $\theta$. The so-called 'bright' squeezed state is generated by applying both a displacement and a squeezing operation to the vacuum state: 
\begin{equation}
  |\alpha, r, \theta\rangle = \hat{S}(r, \theta)\hat{D}(\alpha)|0\rangle
\end{equation}
where the squeezing operator $\hat{S}(r, \theta)$ is defined as
\begin{equation}
\hat{S}(r, \theta) = \exp\left( r \int^{\infty}_{-\infty} \dfrac{d\Omega}{2\pi} \,\left[ e^{- i 2\theta(\Omega)} \hat{a}^{\vphantom{\dagger}}_+ \hat{a}^{\vphantom{\dagger}}_- - e^{ i 2\theta(\Omega)} \hat{a}^\dagger_+ \hat{a}^\dagger_- \right] \right)
\end{equation}
where we assumed the squeezing parameter $r$ to be frequency-independent. This operator describes the process of parametric down-conversion, where pairs of photons are created or annihilated in the sideband modes $\hat{a}_\pm$ with a phase relation determined by the squeezing angle $\theta(\Omega)$. We can then write the action of the squeezing operator on the sideband operators as
\begin{equation}
\hat{S}^\dagger \hat{a}_+ \hat{S} = \hat{a}_+ \cosh r - e^{i 2\theta(\Omega)} \hat{a}^\dagger_- \sinh r
\end{equation}
and similarly for $\hat{a}_-$. This transformation shows how the squeezing operator mixes the annihilation and creation operators, leading to modified quadrature fluctuations in the squeezed state. Applying both transformations (displacement and squeezing) to the field vector, we have
\begin{equation}
\hat{D}^\dagger \hat{S}^\dagger \hat{a}_+ \hat{S} \hat{D} = \hat{a}^{\vphantom{\dagger}}_+ \cosh r - e^{i 2\theta} \hat{a}^\dagger_- \sinh r + \gamma \delta (\Omega) \, \label{II.26}
\end{equation}
with $\gamma = \alpha \cosh r - \alpha^* e^{i 2\theta} \sinh r = |\gamma|e^{i\bar{\varphi}'}$ the displaced amplitude at the carrier frequency ($\Omega=0$). We stress that the phase $\bar{\varphi}'$ generally differs from the displacement angle $\bar{\varphi}$ of the coherent amplitude $\alpha$. \\ 

\noindent \textbf{Expectation values: } Similarly as in a coherent state, but this time from Eq.\ref{II.26}, we can derive the expectation values of the quadrature vector in a bright squeezed state as
\begin{equation}
\langle \hat{D}^\dagger  \hat S^\dagger \mathbf{\hat{\mathbf u}} _\phi  \hat S \hat D \rangle
= \mathbf R(\phi)\,\langle \hat{D}^\dagger  \hat S^\dagger \mathbf{\hat{\mathbf u}} \hat S \hat D\rangle
=
2 \delta (\Omega)\begin{pmatrix}
\mathrm{Re}\big( |\gamma| e^{i(\bar{\varphi}'-\phi)}\big) \\[2pt]
\mathrm{Im}\big(|\gamma| e^{i(\bar{\varphi}'-\phi)}\big)
\end{pmatrix}
\end{equation}
indicating that the mean field is shifted by the displaced amplitude $\gamma$ at the carrier frequency. \\

\noindent \textbf{Spectrum: }
We identify the field fluctuation transformation under the squeezing and displacement operators from Eq.\ref{II.26}:
\begin{equation}
\hat D \hat{S}^\dagger \mathbf{\delta \hat{a}} \hat{S} \hat D = \begin{pmatrix}
\cosh r & -e^{i 2\theta(\Omega)} \sinh r \\
-e^{-i 2\theta(\Omega)} \sinh r & \cosh r
\end{pmatrix}
 \mathbf{\delta \hat{a}} 
\end{equation}
such that the quadrature fluctuations read 
\begin{equation}
\hat{D}^\dagger \hat S^\dagger \mathbf{\delta \hat{u}} \hat S \hat D
=  \mathbf{T} \, \mathbf{\delta \hat{u}} 
\end{equation}
where we defined the transfer matrix
\begin{equation*}
\begin{split}
\mathbf{T} & =\begin{pmatrix}
\cosh r - \sinh r \cos 2\theta(\Omega) & -  \sinh r \sin 2\theta(\Omega) \\[6pt]
-  \sinh r \sin 2\theta(\Omega) & \cosh r + \sinh r \cos 2\theta(\Omega) \\
\end{pmatrix} \\
& = \mathbf{R}(-\theta) e^{-r \sigma_z} \mathbf{R}(\theta)
\end{split}
\end{equation*}
Using the linear optical system formalism introduced earlier, we identify the transfer matrices $\mathbf{T}$ and $\mathbf{L} = \mathbf{0}$ (no additional noise). The noise spectral density matrix of a bright squeezed state is then computed as
\begin{equation}
  \mathbf S_{\text{sqz}}[\Omega] = \mathbf{T}\mathbf{S}_{\text{vac}} \mathbf{T}^{\dagger} = \mathbf{T}\mathbf{T}^{\dagger}.
\end{equation}
Explicitly, this yields
\begin{equation}
  \begin{split}
\mathbf{S}_{\text{sqz}}[\Omega] &  = \mathbf{R}(-\theta) e^{-2r \sigma_z} \mathbf{R}(\theta)\\
& = \begin{pmatrix}
\cosh 2r - \sinh 2r \cos 2\theta(\Omega) & -\sinh 2r \sin 2 \theta(\Omega) \\[6pt]
-\sinh 2r \sin 2\theta(\Omega) & \cosh 2r + \sinh 2r \cos 2\theta(\Omega)
\end{pmatrix}. 
\label{II.xx4}
  \end{split}
\end{equation}
This result shows how the squeezing parameter \(r\) and squeezing angle \(\theta\) influence the quadrature fluctuations in the squeezed state. The diagonal elements of the noise spectral density matrix represent the variances of the quadrature fluctuations, while the off-diagonal elements represent the correlations between the quadratures. At an arbitrary measurement angle \(\phi\), the noise spectral density matrix is given by 
\begin{equation}
  \begin{split}
  \mathbf{S}_{\text{sqz},\phi}[\Omega] & = \mathbf{R}(\phi) \, \mathbf{S}_{\text{sqz}}[\Omega] \, \mathbf{R}(-\phi) \\
  & =  \mathbf{R}(\phi-\theta ) e^{-2r \sigma_z} 
\mathbf{R}(\theta-\phi).
  \end{split} 
\end{equation}
such that measuring along the squeezing angle \(\phi = \theta(\Omega)\) yields the minimum variance in the first quadrature.\\ 



\begin{figure}
\centering
\includegraphics[width=\textwidth]{./chap2/fig/quantum_quadraturesBis.pdf}
\caption{Phase-space representations of bright squeezed states with the different quadratures choices.
(a) generic bright squeezed state. 
(b) projection of the quantum noise on the standard quadratures  ($a_1,a_2$). 
(c) projection of the quantum noise on the ellipse major axes quadratures ($a_\theta,a_{\theta + \pi/2}$), with $\theta$ the ellipse angle with respect to the standard quadratures. 
(d) projection of the quantum noise on the amplitude and phase quadratures ($p,q$). } 
\end{figure} 

\noindent \textbf{Amplitude and Phase squeezed states: }
Considering a displaced squeezed state, two special cases are of interest: the amplitude squeezed state where $\theta=\bar{\varphi}$ and the phase squeezed state where $\theta = \bar{\varphi}+\pi/2$. In the first case, the amplitude quadrature $\hat{p}$ is squeezed, while the phase quadrature $\hat{q}$ is anti-squeezed. In the second case, the phase quadrature is squeezed, while the amplitude quadrature is anti-squeezed. The covariance matrices for these states can be derived from Eq.~\eqref{II.xx4} by setting $\psi = 0$ or $\psi = \pi/2$, respectively. \\
\subsection{Classical Modulations}
A key ingredient in our study is the concept of sidebands generated by classical modulations of a coherent field. These sidebands are frequency components that appear around the carrier frequency of the field due to the modulation process. We will consider two types of classical modulations: amplitude modulation (AM) and phase modulation (PM). These are instrumental in experimental physics, as they notably allow to extract science signals and usable error signals to stabilize and lock various parameters of an optical setup, such as the length of a cavity or the phase of a local oscillator. Additionally, as we will see later, the optomechanical interaction itself can be seen as a phase modulation of the intracavity field by the mechanical motion, generating sidebands and noises that carry information about the mechanical position. \\ 

\noindent \textbf{Amplitude Modulation (AM) :} Let the classical amplitude be modulated at $\Omega_{\text{mod}}$ in amplitude:
\begin{equation}
  \alpha(t) = \bar{\alpha} \left(1 + \epsilon_a \cos(\Omega_{\text{mod}} t)\right)
\end{equation}
with $\epsilon_a \ll 1$, the field amplitude modulation depth. While the DC term lives at frequency $\omega_0$, the modulation introduces sidebands at frequencies $\omega_0 \pm \Omega_{\text{mod}}$, seen by expanding the cosine:
\begin{equation}
  \alpha(t) = \bar{\alpha} \Big( 1 + \frac{\epsilon_a }{2}\, e^{i\Omega_{\text{mod}} t} + \frac{\epsilon_a }{2} \,  e^{-i\Omega_{\text{mod}} t} \Big)
\end{equation}


\noindent \textbf{Phase Modulation (PM) :} Now let the classical amplitude be modulated in phase at frequency $\Omega_{\mathrm{mod}}$:
\begin{equation}
\alpha(t) = \bar{\alpha} \, e^{i \epsilon_{\phi} \cos(\Omega_{\mathrm{mod}} t)}
\label{eq:PM_def}
\end{equation}
with $\epsilon_{\phi} \ll 1$ the field phase modulation depth. Expanding to first order in $\epsilon_{\phi}$ gives:
\begin{equation}
\alpha(t) \approx \bar{\alpha} \Big( 1 + \frac{i \epsilon_{\phi} }{2} \, e^{i\Omega_{\mathrm{mod}} t} + \frac{i \epsilon_{\phi} }{2} \, e^{-i\Omega_{\mathrm{mod}} t} \Big)
\label{eq:PM_expand}
\end{equation}
While the carrier term lives at frequency $\omega_0$, the modulation introduces sidebands at $\omega_0 \pm \Omega_{\mathrm{mod}}$, both shifted in phase by $\pi/2$ relative to the carrier.\\ 

In both cases, amplitude or phase modulations, the field contains a carrier at frequency $\omega_0$ and two sidebands at $\omega_0 \pm \Omega_{\mathrm{mod}}$. Amplitude modulation results in sidebands that are in phase with the carrier, while phase modulation produces sidebands with a $\pm \pi/2$ phase shift relative to the carrier. We also note a general modulation process as :
\begin{equation}
\alpha(t) = \bar{\alpha} \left(1 + \varepsilon(t) \right)
\end{equation}
where $\varepsilon(t) \in \mathbb{C}$ is a modulation function that weakly modulates the complex amplitude in time, and that features information about the modulation frequency and depth. It then follows that the linearized amplitude-phase operators can be expressed as
\begin{equation}
\mathbf{\hat{\mathbf u}} _{\bar{\varphi}} (t) = 2|\bar{\alpha}| \begin{pmatrix}
  1 \\ 0 
\end{pmatrix} + 2|\bar{\alpha}|\begin{pmatrix}
  \mathrm{Re}\big(\varepsilon(t) \big) \\
  \mathrm{Im}\big(\varepsilon(t) \big)
\end{pmatrix}
+ \begin{pmatrix}
  \delta \hat{p}(t) \\
  \delta \hat{q}(t)
\end{pmatrix}
\end{equation}





Computing the Fourier transform for amplitude and phase modulations yields
\begin{equation}
  \begin{split}
    \varepsilon^{AM}(\Omega) & = \frac{\epsilon_a}{2} \Big(\delta(\Omega - \Omega_{\text{mod}}) +\delta(\Omega + \Omega_{\text{mod}}) \Big)\\
    \varepsilon^{PM}(\Omega) & = \frac{i\epsilon_\phi}{2} \Big(\delta(\Omega - \Omega_{\text{mod}}) + \delta(\Omega + \Omega_{\text{mod}})\Big)
  \end{split}
\end{equation}
And the quadrature operators of a modulated field can be expressed as
\begin{equation}
\mathbf{\hat{\mathbf u}} _{\bar{\varphi}} [\Omega] =2|\bar{\alpha}| \begin{pmatrix}
  1 \\ 0 
\end{pmatrix}\delta(\Omega) + 2|\bar{\alpha}|\begin{pmatrix}
  \mathrm{Re}\big(\varepsilon[\Omega] \big) \\
  \mathrm{Im}\big(\varepsilon[\Omega] \big)
\end{pmatrix}
+ \begin{pmatrix}
  \delta \hat{p}[\Omega] \\
  \delta \hat{q}[\Omega]
\end{pmatrix}
\end{equation}
We illustrate this by computing the spectra of a coherent field modulated in amplitude. The amplitude-phase quadrature fluctuation part reads
\begin{equation}
  \delta \mathbf{\hat{\mathbf u}}_{\bar{\varphi}}[\Omega] = |\bar{\alpha}|\epsilon_a \begin{pmatrix}
  \delta(\Omega - \Omega_{\text{mod}}) +\delta(\Omega + \Omega_{\text{mod}}) \\
  0 
\end{pmatrix}
+ \begin{pmatrix}
  \delta \hat{p}[\Omega] \\
  \delta \hat{q}[\Omega]
\end{pmatrix}
\end{equation}
such that its covariance matrix reads 
\begin{equation}
\mathbf{S}_{\bar{\varphi}}[\Omega] =  2|\bar{\alpha}|^{2}\epsilon_a^{2}
\Big[ \delta(\Omega - \Omega_{\mathrm{mod}})
     + \delta(\Omega + \Omega_{\mathrm{mod}}) \Big] \begin{pmatrix}
  1 & 0 \\
  0 & 0
     \end{pmatrix} + \mathbf{1}
\end{equation}
As seen in the above expression, the covariance matrix display a sum of Dirac functions corresponding to a classical amplitude modulation of the field, as well as a flat vacuum noise across all frequencies.



\subsection{Quantum Sideband Diagram }

We now have all the tools to graphically represent the quantum states of light in the frequency domain. The so-called \emph{quantum sideband diagram} is a useful representation to visualize the quantum states of light, especially when dealing with modulated fields or squeezed fields \cite{chelkowski2007,chua2013}. In this representation, we plot the carrier frequency \(\omega_0\) at the center, and the sidebands at frequencies \(\omega_0 \pm \Omega\). Once in the frame rotating at \(\omega_0\), the carrier field is a DC term, while the sidebands are located at \(\pm \Omega\) and rotate at these frequencies. Each sideband mode is then associated with a quantum state: it can, for instance, be in a given squeezed state or in a given coherent state. The \emph{noise properties} of the state are conveniently represented in phase space, and accessed experimentally with a (typically) well-matched homodyne detection scheme, measuring the field quadratures. In practice, experiments probe these noise properties through repeated measurements and averaging. This experimental access to the statistical properties of the quantum state defines the so-called \emph{balloon-on-stick} picture of quantum sidebands. The mean value of the field distribution can be zero or non-zero, while its variance is always non-zero. Looking only at the fluctuations part, we can represent the quantum state of light at a given sideband frequency \(\Omega\), as shown in Fig.~\ref{fig:ballonstick}(a--c). In the case of a coherent state (Fig.\ref{fig:ballonstick}(a)-b), the distribution is circular and centered at the origin, indicating equal vacuum fluctuations in both quadratures.\\ 

\begin{figure}[h!]
\centering
\includegraphics[width=\textwidth]{./chap2/fig/ballonstick.pdf}
\caption{Balloon-on-stick representation of quantum sidebands.
(a) Coherent state sideband photons at some frequency $\pm \Omega_1$, represented as circular Gaussian distributions centered at the origin (b). The photons share no correlations between quadratures, resulting in equal vacuum fluctuations.
(c) Squeezed state sideband photons at some frequency $\pm \Omega_1$, drawn from a thermal distribution larger than vacuum. (d) The photons are decomposed into a correlated displacement along the squeezing axis (arrow) and an independent isotropic sub-vacuum fluctuation (circle) seen in (e). (f) The resulting distribution is an ellipse oriented along the squeezing axis, with reduced variance in one quadrature and increased variance in the orthogonal quadrature.}
\label{fig:ballonstick}
\end{figure}

We start by describing a coherent state modulated at $\Omega_\mathbf{mod}$, as seen in Fig.~\ref{fig:QSD_mod}(a)-(b).As detailed in the previous section, the modulation sidebands are created with a specific phase relation to the carrier depending on the nature of the modulation (amplitude or phase), while all the other sideband modes rest in the vacuum state (as seen at frequencies $\Omega_1$ and $\Omega_2$). Upon time evolution, the signal at the modulation frequency $\Omega_\mathbf{mod}$ can be seen as the vectorial sum of the two sidebands rotating in opposite directions at $\pm \Omega_\mathbf{mod}$ (Fig.~\ref{fig:QSD_mod}). When averaging these signals, with a spectrum analyzer for instance, we extract the PSD of the resulting averaged signal. \\ 
\begin{figure}[h!]
\centering
\includegraphics[width=\textwidth]{./chap2/fig/QSD_mod.pdf}
\caption{Quantum sideband diagram of a modulated coherent state.
(a) Amplitude modulation at frequency $\Omega_{\mathrm{mod}}$ generates sidebands in phase with the carrier. (b) Phase modulation at frequency $\Omega_{\mathrm{mod}}$ generates sidebands in dephased by $\pm \pi/2$ relative to the carrier. The top plots show the time evolution of the sidebands. The middle plots show the resulting signal at $\Omega_{\mathrm{mod}}$ as the vectorial sum of the two sidebands as a function of time. The bottom plots show the corresponding single-sided averaged signal, whose PSD provides experimmentally accessible spectra. As expected, all noises are above the vacuum level (red circle).}
\label{fig:QSD_mod}
\end{figure}


We now focus on the representation of squeezed states. We reproduce the quadrature statistics of squeezed states fluctuations using an equivalent \textit{sum of random variables} picture. Let $\mathbf u=(-\sin\theta,\cos\theta)^T$ be the unit vector defining the (rotated) quadrature direction (note the convention $\theta\to\theta+ \pi/2$), and draw a \textit{correlated} displacement constrained to that direction,
$X_{\rm a}=Z\mathbf u$ with $Z\sim\mathcal N(0,\,2\sinh 2r)$, together with an independent isotropic fluctuation
$X_{\rm b}\sim\mathcal N(0,\,e^{-2r}\mathbf{1})$ (vacuum-normalized). The detected point is $X=X_{\rm a}+X_{\rm b}$, hence $\langle X\rangle=0$ and the covariance is the sum
$\Sigma=\langle XX^{ T}\rangle=e^{-2r}\mathbf 1+2\sinh 2r\,\mathbf u\mathbf u^{ T}$, i.e.
\begin{equation}
\begin{split}
\Sigma= & 
\begin{pmatrix}
e^{-2r}+2\sinh 2r\,\sin^2\theta & -\,\sinh 2r\,\sin 2\theta\\
-\,\sinh 2r\,\sin 2\theta & e^{-2r}+2\sinh 2r\,\cos^2\theta
\end{pmatrix}\\ 
&=
\begin{pmatrix}
\cosh 2r-\sinh 2r\,\cos 2\theta & -\,\sinh 2r\,\sin 2\theta\\
-\,\sinh 2r\,\sin 2\theta & \cosh 2r+\sinh 2r\,\cos 2\theta
\end{pmatrix}.
\end{split}
\end{equation}
This is strictly the same noise matrix obtained in the two-photon formalism, where the diagonal terms give the variances and the off-diagonal terms encode the correlations/anticorrelations between quadratures; the eigen-axes of $\Sigma$ have variances $e^{\mp 2r}$, as expected for a squeezed state. Additionally if $r=0$, the correlated displacement vanishes and we recover the vacuum noise $\Sigma=\mathbf 1$. Graphically, it corresponds to photons \textit{clicking} in the upper and lower sidebands and drawn from a thermal distribution (larger than vacuum), as seen in Fig.~\ref{fig:ballonstick}(c). These clicks are then decomposed in a correlated part/arrow of variance $2\sinh 2r$ along the squeezing axis, and a random isotropic and sub-vacuum fluctuation of variance $e^{-2r}$ (Fig.~\ref{fig:ballonstick}(d)-(e)). The arrows are correlated such that they always mirror each other with respect to the correlation axis set by the phase of the pump (Fig.~\ref{fig:ballonstick}(d)). The resulting signal at $\Omega_1$ can then be seen as the sum of these two contributions. The resulting distribution, as seen in the single-sided spectrum is then an ellipse oriented along the correlation axis, with a reduced variance in one quadrature and an increased variance in the orthogonal quadrature. This picture is very useful to intuitively understand how squeezed states behave when going through various optical systems, and particularly for frequency-dependent squeezing, as we will see in Chapter II. \\

\begin{figure}[h!]
\centering
\includegraphics[width=\textwidth]{./chap2/fig/QSD_sqz2.pdf}
\caption{Quantum sideband diagram of a squeezed state. Contrary to a modulated field, the sideband photons are drawn from a thermal distribution i.e. they \textit{click} at random quadrature angles. However, as seen in Fig.\ref{fig:ballonstick}, these photons are correlated such that their sum results in a squeezed state with reduced fluctuations in one quadrature and increased fluctuations in the orthogonal quadrature. This happens at all sideband frequencies $\Omega_1$ and $\Omega_2$ within a frequency band determined by the squeezing source, with a frequency-independent angle provided by the pump phase. The three plots correspond to phase squeezing (a), arbitrary angle squeezing (b), and amplitude squeezing (c).} 
\label{fig:QSD_sqz}
\end{figure}



Interestingly, the thermal distribution from which the sideband photons are drawn can be given a physical meaning. First, the \textit{non-zero} number of photons in the quantum state can be interpreted as the energy pumped into the system to generate the squeezed state, for instance by a pump field in a parametric down-conversion process. Second, if we imagined measuring the sidebands individually rather than jointly as in a homodyne detection (where they \textit{beat} together), we would find that each sideband is in a thermal state, highlighting that entanglement is lost when tracing the quantum state on a single sideband space. 

\section{Optical Cavities}
Optical cavities are at the heart of this work, as they are used to coherently enhance the light-matter interaction in various systems, and also to filter and manipulate quantum states of light. In this section, we review the basic properties of optical cavities, their resonance conditions, and we derive the covariance matrices of their output fields. 

\subsection{Cavity Geometries and Stability Conditions}
An optical cavity is a structure that \textit{traps} photons by means of reflection between two or more mirrors. They can be either standing-wave cavities, where the light bounces back and forth between two mirrors, or traveling-wave cavities, where the light circulates in a loop. In both cases, the cavity supports discrete resonant modes determined by its geometry and the boundary conditions imposed by the mirrors. The stability criteria of a specific cavity configuration is derived considering the round trip ABCD matrix of the cavity describing how the complex beam parameter $q(z)$ introduced in \eqref{II.1} is transformed after one round trip \cite{Kogelnik66}. The stability condition then simply reads as $-1 < (A+D)/2 < 1$. In the case of planar travelling-wave cavities, one needs to consider both the tangential and sagittal planes, as these cavities are astigmatic. The stricter condition, generally the sagittal plane one, then defines the stability range of the cavity.\\

\begin{figure}
\centering
\includegraphics[width=\textwidth]{./chap2/fig/cavity_types.pdf}
\caption{Geometries of various cavity types used in this work. 
(a) Linear concave-concave cavity (confocal in the $L \sim 2R$ case). 
(b) Linear plano-concave cavity. (a) and (b) are both standing wave cavities. 
(c) Planar triangular cavity. 
(d) Planar bow-tie cavity. (c) and (d) are both travelling wave cavities. } 
\label{fig:cavity_types}
\end{figure}


\noindent \textbf{Linear standing-wave cavities:} We first consider the two linear cavities used in this work, namely a concave-concave cavity (Fig \ref{fig:cavity_types}.(a)) with two identical concave mirrors, and a plano-concave cavity with one flat mirror and one concave mirror  (Fig \ref{fig:cavity_types}.(b)). Using the ABCD formalism for a confocal cavity of length $L$ formed by two identical mirrors of radii of curvature $R$, the stability condition reads
\begin{equation}
    L < 2R
\end{equation}
For the plano-concave cavity, the stability condition reads
\begin{equation}
    L < R
\end{equation}

\noindent \textbf{Planar traveling-wave cavities:} We now consider a triangular cavity formed by two concave mirrors of radius of curvature $R$ and one flat mirror  (Fig \ref{fig:cavity_types}.(c)). The stability condition reads
\begin{equation}
   \quad 0 < L_{rt} < 2R \cos(\theta)
\end{equation}
where $L_{rt} = 2L + l$ is the cavity round trip length, and $\theta$ is the angle of incidence of the beam onto the curved mirror. This condition is the sagittal one, and is more stringent than the tangential one.\\

Considering now a bow-tie cavity formed by two concave mirrors of radius of curvature $R$ and two flat mirrors  (Figure \ref{fig:cavity_types}.(d)), the full stability condition reads
\begin{equation}
  0 < \left( 1 - \frac{L_1 + 2L'}{R \cos\theta} \right) \left( 1 - \frac{L_2}{R \cos\theta} \right) < 1
\end{equation}
where $L_1$ is the distance between the two concave mirrors, $L_2$ the distance between the two flat mirrors, and $L'$ the distance between a concave and a flat mirror (assuming a symmetric cavity). A simple design rule guaranteeing stability is then to set $L_1 + 2L' < R \cos\theta$ and $L_2 < R \cos\theta$.\\


\subsection{Cavity Resonances}
If the cavity is stable, it will then feature a discrete set of resonant modes, each corresponding to a cavity length that is an integer multiple of half the wavelength $\lambda/2$ (standing-wave cavity) or the wavelength $\lambda$ (traveling-wave cavity). In the frequency domain, modes are spaced by the free spectral range $\omega_{\mathrm{FSR}}$ of the cavity, defined as
\begin{equation}
  \omega_{\mathrm{FSR}} = \frac{\pi c}{L} \quad \text{(linear cavity)}, \quad \omega_{\mathrm{FSR}} = \frac{2 \pi c}{L_{rt}} \quad \text{(traveling-wave cavity)}
\end{equation}
such that the resonant frequencies are given by
\begin{equation}
  \omega_N = N \, \omega_{\mathrm{FSR}}, \quad N \in \mathbb{N}
\end{equation}
and the cavity is on resonance when the input laser frequency $\omega_0$ matches one of the resonant frequencies $\omega_N$ i.e. $\omega_0 = \omega_N$. To achieve this, one can either tune the laser frequency or the cavity length. In our experiments, we use the second option by mounting one of the cavity mirrors on a piezoelectric actuator. Changing the cavity length $L$ by $\delta L$ shifts the resonant frequencies by
\begin{equation}
  \delta \omega_N = - N \, \frac{\pi c}{L^2} \, \delta L = - \frac{\omega_N}{L} \, \delta L
\end{equation}

\subsection{Mode-Matching}
A cavity also supports TEM$_{mn}$ transverse modes, each with a specific spatial profile and resonant frequency. The resonant frequencies of these transverse modes are shifted relative to the fundamental mode by an amount that depends on the cavity geometry and the mode indices $(m,n)$. Coupling an incoming beam into a stable optical cavity requires that the spatial mode of the beam matches that of the cavity. This means that the mode function of the incoming beam, assumed to be a TEM$_{00}$ Gaussian mode
 $f_{0}(\mathbf{r})$, must overlap with the cavity’s fundamental mode $f_{0}'(\mathbf{r})$. If the basis functions are not perfectly matched, the incoming field can be expanded in the orthonormal basis of cavity modes as
\begin{equation}
    f_{0}(\mathbf{r}) = c_{0}\, f_{0}'(\mathbf{r}) + \sum_{m>0} c_{m}\, f_{m}'(\mathbf{r}),
\end{equation}
where the coefficients $c_{m}$ quantify the projection of the incident field onto the cavity eigenmodes. Only the component $c_{0} f_{0}'$ couples efficiently to the fundamental cavity mode due to the mirror geometry, while any mismatch excites higher-order transverse modes $f_{m}'$. The mode-matching procedure therefore consists in maximizing the overlap integral
\begin{equation}
    \eta = \left| \int d^{3}\mathbf{r}\, f_{0}^{*}(\mathbf{r})\, f_{0}'(\mathbf{r}) \right|^{2},
\end{equation}
which ensures that essentially all the incoming photons populate the desired cavity mode, while suppressing excitation of spurious modes. \\

\subsection{\textrm{Simple} Cavities} \label{sec:simple_cavities}

We consider a single field cavity mode described by the annihilation operator \(\hat{a}\), interacting with several independent noise inputs. The system is governed by a Hamiltonian 
\begin{equation}
\hat{H} = - \hbar \Delta  \hat{a}^\dagger \hat{a} 
\end{equation}
 with  $\Delta\equiv\omega_0 - \omega_c$ the cavity detuning to the laser frequency, and each input introduces dissipation characterized by a decay rate \(\kappa_i = T_i/\tau\), with $T_i$ the power transmittivity of the mirror and $\tau=2L/c$ the roundtrip time of the cavity. This is if we consider an input coupler (mirror) with decay rate $\kappa_1$ and an output coupler (mirror) with decay rate $\kappa_2$. The laser field is shone onto the cavity by the input coupler. \\ 
 
 
 In the frame rotating at the laser frequency, the dynamics of \(\hat{a}\) is given by the Quantum Langevin Equation (QLE):
%
\begin{equation}
\begin{split}
  \frac{d}{dt} \hat{a}(t) & = -\frac{i}{\hbar} [\hat{a}, \hat{H}] - \frac{\kappa}{2} \hat{a}(t) + \sqrt{\kappa_1} \, \hat{a}_{\mathrm{in}}(t)  + \sqrt{\kappa_2} \, \delta \hat{a}_{\mathrm{vac}}(t) + \sqrt{\kappa_0} \, \delta \hat{a}_{\mathrm{l}}(t) \\
  & = -\Big(\frac{\kappa}{2}-i\Delta\Big) \hat{a}(t) + \sqrt{\kappa_{\mathrm{1}}} \, \hat{a}_{\mathrm{in}}(t)  + \sqrt{\kappa_2} \, \delta \hat{a}_{\mathrm{vac}}(t)  + \sqrt{\kappa_0} \, \delta \hat{a}_{\mathrm{l}}(t) 
\label{eq:qle}
\end{split}
\end{equation}
where  \(\kappa = \kappa_0 + \kappa_1 + \kappa_2\) is the total decay rate, with $\kappa_0=\mathcal{L}/\tau$ and $ \delta \hat{a}_{\mathrm{l}}(t)$ the rate and fluctuation operator of additional losses. Here losses $\mathcal{L}$ are unitless. Another key element to deriving both steady state behaviour as well as quadrature spectra is the input-output formula given by \cite{gardiner}:
\begin{equation}
  \hat{a}_{\mathrm{ref}} = \sqrt{\kappa_{1}}\hat{a} - \hat{a}_{\mathrm{in}} , \quad \hat{a}_{\mathrm{trans}} = \sqrt{\kappa_{2}}\hat{a} - \delta \hat{a}_{\mathrm{vac}} 
\end{equation}
for both the reflected and transmitted field. In the input-output formula, the $\hat{a}_{\mathrm{in}}$ refers to the field incoming on the coupler considered, which are for instance simple vacuum fluctuations $\delta \hat{a}_{\mathrm{vac}}$ on the output coupler since we don't shine any laser beam by this port. Importantly, this formula describes how open quantum systems exchange information with their environment, linking the internal cavity field to the external fields. Here, the external fields are the two photon fields defined at the start of this chapter, expressed in units of $1/\sqrt{\text{HZ}}$, and the internal field $\hat a$ is unitless. This key relation allows to compute how a discrete quantum system (the cavity mode) interacts with continuous quantum fields (the input and output fields), enabling the analysis of phenomena such as reflection, transmission, and noise properties of the cavity. \\


As introduced in the previous subsection, one can split the annihilation operator in a mean field part $\alpha$ and a fluctuation part $\mathbf{\delta \hat{a}}(t)$ (vector form) such that this equation turns into both a scalar differential equation, and an operator differential equation, that is:
\begin{equation}
\left\{
\begin{aligned}
0 &= -\Big(\dfrac{\kappa}{2}-i\Delta\Big)\,\bar{\alpha}
    + \sqrt{\kappa_1}\,\bar{\alpha}_{\mathrm{in}} \\
 \mathbf{\delta \dot{\hat{a}}}(t)&
  = -\begin{pmatrix}
        \kappa/2-i\Delta & 0 \\
        0 & \kappa/2+i\Delta
      \end{pmatrix}\!
      \delta\hat{\mathbf{a}}(t)
     + \sqrt{\kappa_1}\,\delta\hat{\mathbf{a}}_{\mathrm{in}}(t)
     + \sqrt{\kappa_2}\,\delta\hat{\mathbf{a}}_{\mathrm{vac}}(t)
     + \sqrt{\kappa_0}\,\delta\hat{\mathbf{a}}_{\mathrm{l}}(t)
\end{aligned}
\right.
\tag{II.62}
\end{equation}

\begin{figure}
\centering
\includegraphics[width=\textwidth]{./chap2/fig/CavitySimple.pdf}
\caption{Resonance and filtering properties of optical cavities. 
(a) Cavity diagram and definitions. $\hat a$'s are the various fields at play. $\kappa$'s are the various couplings involved i.e. input and output mirrors, as well as intrinsic cavity cavity losse, with $\delta \hat  a$'s the associated fluctuations. 
(b) Amplitude and phase response of an optical cavity as a function of the laser detuning (in cavity linewidth unit). In this case, both mirrors are identical ($\kappa_1 = \kappa_2$) and cavity losses are negligible ($\kappa_0 \ll \kappa_1$). 
(c) Transfer functions of the input classical noises as in \eqref{eq:MCIR}}. 
\label{fig:CavitySimple}
\end{figure}



\noindent \textbf{Mean field solution (Static case): }Taking the first scalar equation and expressing the mean intracavity field gives 
\begin{equation}
  \bar{\alpha} =  \frac{\sqrt{\kappa_1}}{\kappa/2-i\Delta}  \bar{\alpha}_{\mathrm{in}} 
\end{equation}
Patching it up with the input-output formula this gives 
\begin{equation}
  \bar{\alpha}_{\mathrm{ref}} =  \Bigg( \frac{\kappa_1}{\kappa/2-i\Delta} - 1 \Bigg)  \bar{\alpha}_{\mathrm{in}}   \quad \quad \quad  \bar{\alpha}_{\mathrm{trans}} =  \frac{\sqrt{\kappa_1 \kappa_2}}{\Big(\kappa/2-i\Delta\Big)} \bar{\alpha}_{\mathrm{in}}.
\end{equation}
The reflection and transmission coefficients are then
\begin{equation}
R(\Delta) = \left|\frac{\bar{\alpha}_{\mathrm{ref}}}{\bar{\alpha}_{\mathrm{in}}}\right|^2
= \frac{\bigl(\kappa_1-\kappa/2\bigr)^2+\Delta^2}{\bigl(\kappa/2)^2+\Delta^2} \quad \quad
T(\Delta) = \left|\frac{\bar{\alpha}_{\mathrm{trans}}}{\bar{\alpha}_{\mathrm{in}}}\right|^2
= \frac{\kappa_1\kappa_2}{\bigl(\kappa/2\bigr)^2+\Delta^2}.
\end{equation}
The cavity linewidth (FWHM) is then given by $\kappa/2\pi$ (Hz), as illustrated In Fig \ref{fig:CavitySimple}.(b). We will then refer to $\kappa/4\pi$ (HWHM) as the cavity bandwidth (Hz). Plugging back the expression of $\kappa_i = T_i/\tau$ in the reflection coefficient, we have 
\begin{equation}
  R(\Delta \gg \kappa) = 1  \qquad R(\Delta =0) = \Bigg(\frac{T_1 - T_2 - \mathcal{L}}{T_1 + T_2 + \mathcal{L}}\Bigg)^2
\end{equation}
such that the relative depth of the resonance dip gives us information about the cavity losses and couplings. In particular, the resonance dip vanishes when $T_1 = T_2 + \mathcal{L}$, which is the so called \textit{impedance matching} condition: no light is reflected at resonance and all of it is transmitted or lost. Similarly, the transmission coefficient reads
\begin{equation}
  T(\Delta \gg \kappa) = 0  \qquad T(\Delta =0) = \frac{4 T_1 T_2}{(T_1 + T_2 + \mathcal{L})^2}
\end{equation}

We also define the cavity finesse $\mathcal{F}$, which is a measure of the sharpness of the resonance peaks relative to its FSR, as
\begin{equation}
  \mathcal{F} = \frac{\omega_{\mathrm{FSR}}}{\kappa} = \frac{\pi c}{L \kappa} = \frac{2\pi}{T_1 + T_2 + \mathcal{L}}
\end{equation}
which also gives the average number of round trips a photon makes before escaping the cavity i.e. $\langle n_{rt} \rangle = \mathcal{F}/\pi$. For a given cavity length (so same FSR), the higher the finesse, the longer the photon lifetime in the cavity $\kappa^{-1}$. \\

Measuring all these quantities then allows to fully characterize the cavity parameters $T_1$, $T_2$ and $\mathcal{L}$. This is done by measuring the reflection and transmission coefficients at resonance and far from resonance, as well as the cavity linewidth (or finesse). \\ 



\noindent \textbf{Mean field solution (Dynamical case): } 
We now let the detuning vary linearly in time, and express it in units of cavity bandwidth as $\Delta(t)= \Delta_0 + v \frac{\kappa^2}{2}t$ where we define $v$ as the sweep speed in units of cavity bandwidth per $\kappa^{-1}$. The intracavity field yields the utile differential equation
\begin{equation}
  \dot{\bar{\alpha}}(t) = -\Bigg(\frac{\kappa}{2} - i\Big(\Delta_0 + \frac{v \kappa^2}{2}t\Big)\Bigg) \bar{\alpha}(t) + \sqrt{\kappa_1} \, \bar{\alpha}_{\mathrm{in}}
  \label{eq:dyn_cav}
\end{equation} 
This is solved by the means of integration factor method, such that we find
\begin{equation}
\begin{split}
\alpha(t) =\;
& \exp\!\Bigg[ \Big(-\dfrac{\kappa}{2} + i\Delta_{0}\Big)t
+ i\,\dfrac{v\kappa^{2}}{4}\,t^{2} \Bigg] \\
& \times \left[
\alpha(0) + \sqrt{\kappa_{1}}\,\bar{\alpha}_{\mathrm{in}}
\int_{0}^{t} \exp\!\Bigg( \Big(\dfrac{\kappa}{2}-i\Delta_{0}\Big)s
- i\,\dfrac{v\kappa^{2}}{4}\,s^{2} \Bigg)\,ds
\right].
\end{split}
\end{equation}


This expression describes the transient response of the intracavity field as the detuning is swept through resonance. When scanning over the cavity resonance at a rate exceeding the cavity bandwidth, photons at various detuning start to build up in the cavity without reaching the steady state value. This results in a characteristic asymmetric lineshape, where these different \textit{colored} photons start beating against each other, leading to oscillations in the transmitted and reflected intensities. This is illustrated in Fig \ref{fig:opticalRD}.(a) for different sweep speeds. The above does feature an analytical formula involving error functions erf, such that can either fit the data by performing a numerical integration or the analytical formula. However if the data array feature too few points numerical integration becomes numerically costly.  \\


The finesse–extraction procedure of Poirson \textit{et al.}~\cite{Poirson:97} can be expressed in our notation by identifying their cavity length $d_{0}$ with our $L$, and their bandwidth $\Omega_{r}$ with $\kappa/2$. In the fast--sweep regime—equivalent to our linear detuning sweep $\Delta(t)=\Delta_{0}+v_{\Delta} t$—the transmitted intensity displays a sequence of maxima originating from the same interference mechanism responsible for the oscillatory behaviour in our analytical solution (\ref{eq:dyn_cav}). Poirson \textit{et al.} show that the time interval $\Delta t$ between the first two transmitted maxima satisfies, near $\Delta t\simeq \tau$, the relation
\begin{equation}
    \frac{\pi c}{L}\,\Delta t \;\simeq\; \frac{F}{e}\,\frac{I_{1}}{I_{2}},
    \label{eq:poirsoneq21}
\end{equation}
which is their Eq.~(21) rewritten using our notation. Since the intensity decay time is $\tau = 2/\kappa = FL/(\pi c)$, Eq.~\eqref{eq:poirsoneq21} links the measured ratio $I_{1}/I_{2}$ and the peak spacing $\Delta t$ directly to the cavity finesse. This provides an experimentally simple and robust method to determine $F$, fully consistent with the dynamical intracavity--field model developed here. \\ 



\begin{figure}
\centering
\includegraphics[width=\textwidth]{./chap2/fig/opticalRD.pdf}
\caption{Optical ringing effect upon a fast cavity detuning sweep. (a) Examples of optical ringing for four different sweep speeds. (b) Plotting the time interval $\Delta t$ between the first two transmitted maxima as a function of the ratio $I_1/I_2$ of their intensities allows to extract the cavity finesse $\mathcal{F}$ using the relation \eqref{eq:poirsoneq21}. Here the data is simulated using the parameters of a realistic cavity i.e. $\mathcal{F} = 20000, L=2$cm. As $v$ tends to 0, the second maximum vanishes and one recovers the lorentzian lineshape.} 
\label{fig:opticalRD}
\end{figure}



\noindent \textbf{Fluctuations solution: }To derive the covariance matrix we go to Fourier space such that 
\begin{equation}
     \mathbf{M}_\Delta \mathbf{\delta \hat{a}}[\Omega]  = \sqrt{\kappa_{\mathrm{1}}} \, \mathbf{\delta \hat{a}_{\mathrm{in}}}[\Omega]  + \sqrt{\kappa_2} \, \mathbf{\delta \hat{a}_{\mathrm{vac}}}[\Omega]   + \sqrt{\kappa_0} \, \mathbf{\delta \hat{a}_{\mathrm{l}}}[\Omega]   
\end{equation}
with 
\begin{equation*}
  \mathbf{M}_\Delta =\begin{pmatrix}
  \kappa/2-i(\Delta+\Omega) & 0 \\ 
  0 & \kappa/2+i(\Delta-\Omega)\\ 
  \end{pmatrix} 
\end{equation*}
For notational convenience, we will drop the explicit dependence on $\Omega$ in the following. Inverting the above relation and plugging it in the input-output relations gives the reflected and transmitted fields as
\begin{equation}
  \begin{split}
  \mathbf{\delta \hat{a}_{\mathrm{ref}}}  &= ( \kappa_1 \, \mathbf{M}^{-1}_\Delta - \mathbf{1} )\, \mathbf{\delta \hat{a}_{\mathrm{in}}} +  \sqrt{\kappa_1} \,\mathbf{M}^{-1}_\Delta (\sqrt{\kappa_2} \mathbf{\delta \hat{a}_{\mathrm{vac}}} + \sqrt{\kappa_0}  \mathbf{\delta \hat{a}_{\mathrm{l}}} ) \\
  \mathbf{\delta \hat{a}_{\mathrm{trans}}} & =  \sqrt{\kappa_2} \,\mathbf{M}^{-1}_\Delta (\sqrt{\kappa_1} \mathbf{\delta \hat{a}_{\mathrm{in}}} + \sqrt{\kappa_0}  \mathbf{\delta \hat{a}_{\mathrm{l}}} ) +  (\kappa_2 \,\mathbf{M}^{-1}_\Delta - \mathbf{1}) \, \mathbf{\delta \hat{a}_{\mathrm{vac}}}. 
  \end{split}
\end{equation}
Using $\delta \hat{\mathbf{a}} = \mathbf \Gamma^{-1} \delta \hat{\mathbf{u}}$ the reflected and transmitted quadratures read
\begin{equation}
  \begin{split}
  \mathbf{\delta \hat{u}_{\mathrm{ref}}} &= ( \kappa_1 \, \mathbf{\Gamma} \mathbf{M}^{-1}_\Delta \mathbf{\Gamma}^{-1}- \mathbf{1} )\, \mathbf{\delta \hat{u}_{\mathrm{in}}} +  \sqrt{\kappa_1} \,\mathbf{\Gamma}  \mathbf{M}^{-1}_\Delta \mathbf{\Gamma}^{-1} (\sqrt{\kappa_2} \mathbf{\delta \hat{u}_{\mathrm{vac}}} + \sqrt{\kappa_0}  \mathbf{\delta \hat{u}_{\mathrm{l}}} ) \\
  \mathbf{\delta \hat{u}_{\mathrm{trans}}} & =  \sqrt{\kappa_2} \,\mathbf{\Gamma}  \mathbf{M}^{-1}_\Delta \mathbf{\Gamma}^{-1} (\sqrt{\kappa_1} \mathbf{\delta \hat{u}_{\mathrm{in}}} + \sqrt{\kappa_0}  \mathbf{\delta \hat{u}_{\mathrm{l}}} ) +  (\kappa_2 \,\mathbf{\Gamma}  \mathbf{M}^{-1}_\Delta \mathbf{\Gamma}^{-1}- \mathbf{1}) \, \mathbf{\delta \hat{u}_{\mathrm{vac}}}
  \end{split}
\label{eq:cavity_quad}
\end{equation}
where the matrix product above explicitely reads 
\begin{equation*}
  \mathbf{\Gamma}  \mathbf{M}^{-1}_\Delta \mathbf{\Gamma}^{-1}
= \frac{1}{\left(\kappa/2-i\Omega\right)^{2}+\Delta^{2}}
\begin{pmatrix}
\kappa/2-i\Omega & -\Delta \\[6pt]
\Delta & \kappa/2-i\Omega
\end{pmatrix}.
\end{equation*}
This structure is the engine behind frequency-dependent squeezing. When the cavity is detuned and sidebands are correlated, the off-diagonal terms asymmetrically mix the upper and lower sidebands; in the two-photon picture this is a frequency-dependent rotation and scaling of the $(p,q)$ basis. The amplitude (Lorentzian) part sets how strongly each sideband is transmitted, while the phase accumulated inside the cavity sets the rotation angle that now varies with $\Omega$. A broadband field with a single squeezing angle at the input is therefore converted into an output whose squeezing angle “twists” with frequency: near one band it can align with the phase quadrature, and at another it can align with the amplitude quadrature. This is exactly the mechanism exploited by filter cavities (FC) in precision interferometry: by choosing bandwidth, detuning, and coupling, one tailors the rotation profile to the target noise crossover. Practically, the attainable rotation and the preserved squeezing are limited by the frequency range over which the cavity can efficiently manipulate the sidebands, which can require very high-finesse cavities. \\

\noindent \textbf{Resonant behaviour: } On resonance ($\Delta=0$), the output quadratures in Eq.\eqref{eq:cavity_quad} are written as 
\begin{equation}
  \begin{split}
  \mathbf{\delta \hat{u}_{\mathrm{ref}}}   &= \dfrac{\kappa_1-\kappa/2+i\Omega}{\kappa/2-i\Omega}  \,  \mathbf{\delta \hat{u}_{\mathrm{in}}}   +   \dfrac{\sqrt{\kappa_1 \kappa_2} }{\kappa/2-i\Omega}  \mathbf{\delta \hat{u}_{\mathrm{vac}}} + \dfrac{\sqrt{\kappa_1 \kappa_0} }{\kappa/2-i\Omega}  \mathbf{\delta \hat{u}_{\mathrm{l}}}  \\
  \mathbf{\delta \hat{u}_{\mathrm{trans}}}   &= \, \dfrac{ \sqrt{\kappa_1 \kappa_2}}{\kappa/2-i\Omega}  \, \mathbf{\delta \hat{u}_{\mathrm{in}}} +  \dfrac{\kappa_2-\kappa/2+i\Omega}{\kappa/2-i\Omega}   \mathbf{\delta \hat{u}_{\mathrm{vac}}}   + \dfrac{\sqrt{\kappa_2 \kappa_0} }{\kappa/2-i\Omega}  \mathbf{\delta \hat{u}_{\mathrm{l}}}
  \end{split}
\end{equation}
and their noise spectra are 
\begin{equation}
  \begin{split}
   \mathbf{S}_{\mathrm{ref}}[\Omega] & =\frac{(\kappa_1-\kappa/2)^2+\Omega^2}{(\kappa/2)^2+\Omega^2}\,\mathbf{S}_{\mathrm{in}}+\frac{\kappa_1}{(\kappa/2)^2+\Omega^2}\,\Big(\kappa_0 \,\mathbf{1} + \kappa_2 \, \mathbf{1}\Big) \\ 
     \mathbf{S}_{\mathrm{trans}}[\Omega] & =\frac{(\kappa_2-\kappa/2)^2+\Omega^2}{(\kappa/2)^2+\Omega^2}\,\mathbf{1}+\frac{\kappa_2}{(\kappa/2)^2+\Omega^2}\,\Big(\kappa_0 \, \mathbf{1}+\kappa_1 \mathbf{S}_{\mathrm{in}} [\Omega]\,\Big) \\ 
   \end{split}
  \label{eq:MCIR}
\end{equation}
where the vacuum and loss covariance matrices are equal to \textbf{1}. As these two vacua sum up linearly, it is equivalent to consider a single vacuum with an effective decay rate $\kappa_2 + \kappa_0 \rightarrow \kappa_2$ to lighten the notation. We then absorb intrinsic losses into the output coupler, and consider only two ports: the input coupler with decay rate $\kappa_1$ and the output coupler with decay rate $\kappa_2$. We stress that this substitution is only valid when considering the \textbf{reflected} quadratures. When focusing on the transmitted quadratures, one can perform a similar redefinition with $\kappa_1$ i.e. $\kappa_1 + \kappa_0 \rightarrow \kappa_1$. \\ 

\noindent \textbf{Transfer matrices and Spectra:} Expressing the reflected and transmitted quadratures of Eq.\eqref{eq:cavity_quad} in a compact form yields
\begin{equation}
  \begin{split}
   \mathbf{\delta \hat{u}_{\mathrm{ref}}} & = \mathbf{T}_{\mathrm{ref}}\mathbf{\delta \hat{u}_{\mathrm{in}}} + \mathbf{L}_{\mathrm{ref}}  \mathbf{\delta \hat{u}_{\mathrm{vac}}}
    \\
    \mathbf{\delta \hat{u}_{\mathrm{trans}}} & = \mathbf{T}_{\mathrm{trans}}  \mathbf{\delta \hat{u}_{\mathrm{in}}} + \mathbf{L}_{\mathrm{trans}}  \mathbf{\delta \hat{u}_{\mathrm{vac}}}
  \end{split}
\end{equation}
where the transfer matrices for the input and loss ports are given by
\begin{equation}
  \begin{split}
     \mathbf{T}_{\mathrm{ref}}= \kappa_1 \, \mathbf{\Gamma} \mathbf{M}^{-1}_\Delta \mathbf{\Gamma}^{-1}- \mathbf{1}, \quad \mathbf{L}_{\mathrm{ref}}=  \sqrt{\kappa_1 \kappa_2} \, \mathbf{\Gamma}  \mathbf{M}^{-1}_\Delta \mathbf{\Gamma}^{-1}\\
      \mathbf{T}_{\mathrm{trans}}=   \sqrt{\kappa_1 \kappa_2} \, \mathbf{\Gamma}  \mathbf{M}^{-1}_\Delta \mathbf{\Gamma}^{-1}, \quad \mathbf{L}_{\mathrm{trans}}= \kappa_2 \, \mathbf{\Gamma} \mathbf{M}^{-1}_\Delta \mathbf{\Gamma}^{-1}- \mathbf{1}
  \end{split}
\label{eq:cav_transf}
\end{equation}
From these, one can derive the general transfer matrices and noise spectra for any detuning $\Delta$ in both transmission and reflection. \\ 

Although Example I below illustrates the Mode Cleaner working principle, we will focus on the reflected quadratures regarding the detuned case, since the optomechanical cavities and quantum filters studied in this thesis are mostly operated in reflection, in the single-port cavity configuration i.e. $\kappa_2 = 0$ and $\kappa \sim \kappa_1$ and where we assume negligible losses. We can then approximate the reflected quadrature transfer matrices in Eq.\ref{eq:cav_transf} as
\begin{equation}
      \mathbf{T}_{\mathrm{ref}} = \kappa \, \mathbf{\Gamma} \mathbf{M}^{-1}_\Delta \mathbf{\Gamma}^{-1}- \mathbf{1}  \quad \text{and} \quad 
      \mathbf{L}_{\mathrm{ref}}  =  \mathbf{L}_{\mathrm{trans}} = \mathbf{T}_{\mathrm{trans}} = 0.
\end{equation}
We introduce the single sideband reflected amplitude and phase of the cavity as
\begin{equation*}
  r_+ e^{i \phi_+} = \frac{\kappa/2 + i(\Delta + \Omega)}{\kappa/2 - i(\Delta + \Omega)} \quad \text{,} \quad r_- e^{-i \phi_-} = \frac{\kappa/2 - i(\Delta - \Omega)}{\kappa/2 + i(\Delta - \Omega)} 
\end{equation*}
such that in the lossless case we have $r_\pm = 1$, and the the up/down sideband phase term and its derivative with respect to frequency explicitely read
\begin{equation}
    \phi_{\pm} = 2 \arctan\Bigg(\frac{2(\Delta \pm \Omega)}{\kappa}\Bigg), \quad \text{and} \quad
  \dfrac{\partial \phi_\pm}{\partial \Omega} = \pm \dfrac{4\kappa}{\kappa^2 + 4(\Delta \pm \Omega)^2}.
\end{equation}
We provide the derivative of the phase terms as they are used later to \textit{phase-match} the filter cavity to the optomechanical cavity, in order to compute the optimal filter parameters. The transfer matrix is then expressed as 
\begin{equation}
\begin{split}
      \mathbf{T}_{\mathrm{ref}} & = \frac{1}{2}\begin{pmatrix}
        e^{i \phi_+} + e^{-i \phi_-} & i (e^{i \phi_+} - e^{-i \phi_-}) \\
        -i (e^{i \phi_+} - e^{-i \phi_-}) & e^{i \phi_+} + e^{-i \phi_-}.
      \end{pmatrix} \\
      & = e^{i\delta \phi} \mathbf{R}\left(- \bar \phi \right)
\end{split}
\label{eq:ideal_FC}
\end{equation}
with the two-photon average and differential phase shifts as
\begin{equation*}
  \bar \phi = \frac{\phi_+ + \phi_-}{2} \quad \text{,} \quad  \delta \phi = \frac{\phi_+ - \phi_-}{2}.
\end{equation*}
Here, the cavity acts as an ideal frequency dependent phase shifter, rotating the quadratures by an angle $- \bar \phi$ and adding an overall phase shift $\delta \phi$ that vanishes when computing noise spectra. Explicitely, the two-photon average phase shift and its derivative are given by 
\begin{equation}
    \bar \phi =  \arctan \dfrac{\Delta \kappa}{(\kappa/2)^2 - \Delta^2 + \Omega^2}, \quad \text{and} \quad \dfrac{\partial \bar \phi}{\partial \Omega} = \frac{1}{2} \left( \dfrac{\partial \phi_+}{\partial \Omega} + \dfrac{\partial \phi_-}{\partial \Omega} \right)
\end{equation}
where don't give the full expression of $\partial_\Omega \bar \phi$ because we will do further approximations in Example II. \\ 

\noindent \textbf{Note:} In the more general lossy two-port case, a similar reasoning can be followed by introducing the sideband reflectivities and transmissivities, and rewriting all transfer matrices in terms of these. The resulting expressions for reflection are then 
\begin{equation}
  \mathbf{T}_{\mathrm{ref}} = e^{i\delta \phi} \mathbf{R}\left(- \bar \phi \right) \Big(\bar r \,  \mathbf{1} + i \, \delta r \,  \mathbf{R}\left(\pi/2\right)\Big) \quad \text{and} \quad
  \mathbf{L}_{\mathrm{ref}} = e^{i\delta \theta} \mathbf{R}\left(- \bar \theta \right)\Big(\bar t \mathbf{1} + i \delta t \mathbf{R}  \left(\pi/2\right)\Big)
\end{equation}
with $\bar t$, $\delta t$, $\bar \theta$ and $\delta \theta$ defined from the field complex transmission in the same way as $\bar r$, $\delta r$, $\bar \phi$ and $\delta \phi$.  


\subsubsection{Example I: Resonant symmetric cavity - Mode Cleaner }
Let us consider a configuration such that $\kappa_1 = \kappa_2 \approx \kappa/2$, and where we assume negligible losses $\kappa_0 \ll \kappa_{1,2}$. It represents a cavity where the input and output mirror transmittivities are equal, and we set the laser resonant to the cavity ($\Delta=0$), such that the transmitted spectra in \ref{eq:MCIR} reduces to
\begin{equation}
  \mathbf{S}_{\mathrm{trans}}[\Omega] =\frac{(\kappa/2)^2}{(\kappa/2)^2+\Omega^2} \, \mathbf{S}_{\mathrm{in}}+\frac{\Omega^2}{(\kappa/2)^2+\Omega^2}  \,  \mathbf{1}
  \label{eq:MCIR2}
\end{equation}
Now consider that the input fluctuations are above those of vacuum i.e. the input field features classical noise. We would then have $S_{pp}^{\rm in}> S_{pp}^{\rm vac}=1$ and/or $S_{qq}^{\rm in}> S_{qq}^{\rm vac}=1$. One can notice that the prefactor to the input noises is a Lorentzian function - a low-pass filter. Hence, the noises of the input fields are low-pass filtered by the cavity, while the vacuum fluctuations are high-pass filtered precisely above the same cutoff $\kappa/2$. The mean field of the \textit{bright} coherent input is fully transmitted, but its classical fluctuations, potentially classically modulated, are filtered by the cavity. Taking a high finesse cavity such that the cutoff frequency is low, the transmitted field now features vacuum sidebands: it has been \textit{cleaned} from classical noise. This is the principle of a \textit{mode cleaner} cavity, which is used in many precision experiments to provide a spectrally pure laser field, as well as a spatially filtered beam such that the transmitted beam is a pure $\mathrm{TEM}_{00}$.\\


\subsubsection{Example II: Detuned single-port cavity - Filter Cavity}
As seen in Eq.\eqref{eq:ideal_FC}, a detuned single-port cavity can be expressed as a rotation matrix in the two-photon formalism. This is the principle behind \textit{filter cavities}, which are used to rotate the squeezing angle of a squeezed field in a frequency-dependent way. We will now consider the tabletop case of a high-finesse cavity such that $\kappa \ll |\Delta|$, and where we focus on sideband frequencies $\Omega \sim \Delta$. In this case, the $\partial_\Omega \phi_+$ term is negligible, and we write
\begin{equation}
  \dfrac{\partial \bar \phi}{\partial \Omega} \approx  \dfrac{1}{2} \, \dfrac{\partial \phi_-}{\partial \Omega} = - \dfrac{2\kappa}{\kappa^2 + 4(\Delta - \Omega)^2}.
\end{equation}
The derivative at $\Omega = \Delta$ then reads 
\begin{equation}
  \left.\dfrac{\partial \bar \phi}{\partial \Omega}\right|_{\Omega = \Delta} = - \dfrac{2}{\kappa}
\end{equation}
which shows that the slope of the rotation angle at the detuning frequency is inversely proportional to the cavity bandwidth. A narrower cavity leads to a steeper rotation angle variation with frequency, which is desirable to \textit{phase-match} the filter cavity to the optomechanical cavity, as will be seen in Chap. II. 

\subsection{Non-Linear Cavities}
We now turn to the description of optical cavities with a $\chi^{(2)}$ medium embedded within. This non-linear medium can be used both for sum frequency generation, or difference frequency generation. The generic Hamiltonian describing a  degenerate $\chi^{(2)}$ parametric process is 
\begin{equation}
  \hat H = \hbar \omega_p \hat{b}^{\dagger}\hat{b} + \hbar \omega_0 \hat{a}^\dagger \hat{a} + \frac{i\hbar\epsilon}{2}( \hat{b} \, \hat{a}^{\dagger2} - \hat{b}^\dagger \, \hat{a}^2).
\end{equation}
where $\hat{a}$ and $\hat{b}$ are the annihilation operators of the fundamental field at frequency $\omega_0$ and the pump field at frequency $\omega_p = 2\omega_0$ respectively, and $\epsilon$ is the non-linear coupling strength. The second term describes the up-conversion process where two photons at $\omega_0$ are converted into a single photon at $\omega_p$, while the third term describes the down-conversion process where a single photon at $\omega_p$ is converted into two photons at $\omega_0$.
We assume perfect phase matching for simplicity ($\epsilon \in\mathbb{R}$). The first non-linear term $\hat{b} \, \hat{a}^{\dagger2}$ corresponds to Parametric Down Conversion (PDC), while the second term $\hat{b}^\dagger \, \hat{a}^2$ corresponds to Second Harmonic Generation (SHG). Both processes are illustrated in Fig. \ref{fig:chi2_process}. In our experiment wih squeezed light, we use both processes to first generate a pump field using a SHG scheme, then use the generated field to \textit{pump} a degenerate Optical Parametric Oscillator (OPO). The equations of motion of both fields are very similar in their structure, yet different in their phenomenology. Here we outline the main results and predictions for both. 
\begin{figure}
\centering
\includegraphics[width=\textwidth]{./chap2/fig/schema_SHG_OPO_2photons.pdf}
\caption{Diagrams of $\chi^{(2)}$ non-linear processes. 
(a) Second Harmonic Generation ($2\omega_0 = \omega_p$). (b) Parametric Down Conversion. The outcoming photons are entangled.  } 
\label{fig:chi2_process}
\end{figure}


\subsubsection{Second Harmonic Generation}

The SHG scheme consists in shining a laser field at frequency $\omega_0$ onto the cavity, and the non-linear medium generates a field at frequency $\omega_p = 2\omega_0$, that is, two photons at $\omega_0$ described by operator $\hat{a}$, are converted into a single photon at $\omega_p$ described by operator $\hat{b}$. The input field is thus $\hat{a}_{\mathrm{in}}$ at $\omega_0$, while the input fields at $\omega_p$ are vacua $\hat{b}_{\mathrm{in}} = \delta {\hat b}_{\mathrm{l}}= \delta \hat{b}_{\mathrm{vac}}$. We restrain the theoretical description to our experiment, where the end mirror reflectivity is $\sim 1$ for our generated green beam, as seen in the figure below $\kappa_{2,b}=0$. We will not derive the noise spectra for this scheme as they are not of interest in this work. \\ 
\begin{figure}[h]
\centering
\includegraphics[width=\textwidth]{./chap2/fig/cavitySHG copy.pdf}
\caption{ Cavity diagram for the Second Harmonic Generation. $\hat a$'s are the various fields at play, $\kappa$'s are the various couplings involved, with $\delta \hat a$'s the associated fluctuations, similar as in Fig I.4, now considering both the infrared pump, and the generated green beam. The difference in transverse size has been exagerayed for clarity.  } 
\end{figure}

We rather focus on the mean field solution. The scalar part of the QLE on resonance for both fields are given by 
 \begin{equation}
  \begin{split}
0 &= -\dfrac{\kappa_a}{2}\,\bar{\alpha} 
    + \epsilon\,\bar{\alpha}^{*}\,\bar{\beta} 
    + \sqrt{\kappa_{1,a}}\,\bar{\alpha}_{\mathrm{in}}, \\[6pt]
0 &= -\dfrac{\kappa_b}{2}\,\bar{\beta} 
    + \dfrac{\epsilon}{2}\,\bar{\alpha}^{2}.
  \end{split}
  \end{equation}
where subscript $a$ and $b$ refer to the $\omega_0$ and $\omega_p$ fields respectively. Solving for the $\bar{\beta}$ field and computing the output field $\bar{\beta}_{\mathrm{out}}$ from the input mirror using the input-output relations, yields an output intensity of 
\begin{align}
|\bar{\beta}_{\rm out}|^2
&= \frac{\kappa_{a}^{2}\kappa_{1,b}^{2}}{4\,\varepsilon^{2}}
\Bigg[
\Bigg(1+\frac{108\,\varepsilon^{2}\kappa_{1,a}}{\kappa_a^{3}\kappa_b}\,|\bar{\alpha}_{\rm in}|^{2}
\bigg(1+\sqrt{1+\frac{\kappa_a^{3}\kappa_b}{54\,\varepsilon^{2}\kappa_{1,a}\,|\bar{\alpha}_{\rm in}|^{2}}} \, \bigg)\Bigg)^{\!1/6} \nonumber \\[6pt]
&\qquad -
\Bigg(1+\frac{108\,\varepsilon^{2}\kappa_{1,a}}{\kappa_a^{3}\kappa_b}\,|\bar{\alpha}_{\rm in}|^{2}
\bigg(1+\sqrt{1+\frac{\kappa_a^{3}\kappa_b}{54\,\varepsilon^{2}\kappa_{1,a}\,|\bar{\alpha}_{\rm in}|^{2}}} \, \bigg)\Bigg)^{\!-1/6 \, }
\, \Bigg]^{\, 4}.
\end{align}
consistent with \cite{Sorensen1998PhD,burks_thesis_2010}. This cumbersome expression can be simplified in two limits. In the low input power limit, the output power scales quadratically with the input power, whereas at high powers it scales as $|\alpha_{\mathrm{in}}|^{4/3}$. \\

\noindent \textbf{Pseudo linear behaviour:} For intermediate powers, the output power scales almost linearly with the input power, which is precisely the regime in which we will operate. The crossover between these regimes is set by the non-linear gain $\epsilon$ and the cavity decay rates $\kappa_{a,b}$. \\ 

\subsubsection{Optical Parametric Oscillation \& Amplification}
For this scheme, we consider a pump field with frequency $\omega_p = 2\omega_0$. A first key difference from the SHG scheme can be highlighted by the fact that we are now pumping at $2\omega_0$, such that pairs of entangled photons are generated at $\omega_0 + \Omega$ and $\omega_0 - \Omega$, with $\Omega$ a sideband frequency allowed by the cavity bandwidth, hence conserving energy. 
\begin{figure}[h]
\centering
\includegraphics[width=\textwidth]{./chap2/fig/OPO2.pdf}
\caption{Cavity diagram for the Optical Parametric Oscillator. $\hat a$'s are the various fields at play, $\kappa$'s are the various couplings involved, with $\delta \hat a$'s the associated fluctuations, similar as in Fig I.4, now considering both the green pump, and the generated infrared squeezed beam. The beams are shifted for illustrative purposes but share the same optical axis in the experiment. } 
\end{figure}

We further consider the pump is not \textit{depleted} and we disregard the $\hat{b}$ fluctuations in the equations of motion for simplicity, such that we can change $\hat{b}$ to its mean field value $|\bar{\beta}|e^{i\bar{\varphi}_b}$. A careful and complete derivation could also be carried out by keeping all terms in the equations of motion, but it is not serving our purpose here so we make these assumptions to lighten the notation. The total effective non-linear gain is defined as $g = \epsilon |\bar\beta|$, and the QLEs for the steady state and fluctuation parts of the $\hat{a}$ field yields: 
 \begin{equation}
  \left\{
  \begin{split}
  0 &= -\Big(\frac{\kappa}{2}-i\Delta\Big) \bar{\alpha} +g e^{i\bar{\varphi}_b} \, \bar{\alpha}^* + \sqrt{\kappa_1} \, \bar{\alpha}_{\mathrm{in}} \\
  \mathbf{\delta \dot{\hat{a}}}(t)&= - \begin{pmatrix}
  \kappa/2-i\Delta & -g e^{i\bar{\varphi}_b}\\
   -g e^{-i\bar{\varphi}_b} & \kappa/2+i\Delta \\
  \end{pmatrix}  \mathbf{\delta \hat{a}}(t) + \sqrt{\kappa_{\mathrm{1}}} \, \mathbf{\delta \hat{a}_{\mathrm{in}}}(t)  + \sqrt{\kappa_2} \, \mathbf{\delta \hat{a}_{\mathrm{vac}}}(t)
  \end{split}
  \right.
\end{equation}
\noindent \textbf{Mean field solution (Static case): }
Assuming a real input field $\bar{\alpha}_\textrm{in}=|\bar{\alpha}_\textrm{in}|$, the transmitted field is given by: 
\begin{equation}
   \bar{\alpha}_{\mathrm{trans}} = \frac{\sqrt{\kappa_1\kappa_2}}{\kappa/2} \, \frac{1+i\left(2\Delta/\kappa\right)+xe^{i\bar{\varphi}_b}}{1+\left({2\Delta}/{\kappa}\right)^2 - |x|^2}  |\bar{\alpha}_\textrm{in}|
\end{equation}
where we define the normalised pump parameter $x = 2g / \kappa \in\mathbb{R}$. This normalised pump parameter also equals the ratio of the pump field amplitude by the pump field threshold often written $B/B_{\mathrm{thr}}$. For a resonant cavity, the expression reduces to the well-known parametric amplification/deamplification scheme 
\begin{equation}
   \bar{\alpha}_{\mathrm{trans}} =  \frac{\sqrt{\kappa_1\kappa_2}}{\kappa/2}\frac{1+x e^{i\bar{\varphi}_b}}{1 - |x|^2} | \bar{\alpha}_\textrm{in} |
\end{equation}
in which the amplification or deamplification processes are set by the phase of the pump $\bar{\varphi}_b$. In the absence of a non linear medium $x=0$ we recover the standard cavity results shown above. The threshold is defined at $x=1$, where the rate of generation of entangled pairs exceeds the rate at which they leak from the cavity. In other words, $x$ is unity when the round trip gain equals the round trip losses. That's precisely the point where the no depletion approximation breaks down, as illustrated by the divergence seen in transmitted field at this very value. We also notice two special cases, when $\bar\varphi_b=\{0,\pi\}$, coinciding with the amplification and the deamplification processes respectively. \\

\begin{figure}[h!]
\centering
\includegraphics[width=\textwidth]{./chap2/fig/amp_deamp.pdf}
\caption{Classical amplification-deamplification of an IR seed in an Optical Parametric Oscillator below threshold. 
(a) Gain of the infrared seed as a function of the green pump phase. The color variations correspond to the pump phase. 
(b) Amplification-Deamplification of a infrared seed as a function of the normalised pump paramter $x$ ($<1$). The colors correspond to the ones on figure (a) (its extremas). } 
\end{figure}

\noindent \textbf{Fluctuations solution: } The general expression of the QLE in Fourier space is given by 
\begin{equation}
  \tilde{\mathbf{M}}_\Delta \,  \mathbf{\delta \hat{a}}[\Omega]  = \sqrt{\kappa_{\mathrm{1}}} \, \mathbf{\delta \hat{a}_{\mathrm{in}}}[\Omega]  + \sqrt{\kappa_2} \, \mathbf{\delta \hat{a}_{\mathrm{vac}}}[\Omega]
\end{equation}
with
\begin{equation*}
  \tilde{\mathbf{M}}_\Delta = \begin{pmatrix}
  \kappa/2-i(\Delta+\Omega) & -g e^{i\bar{\varphi}_b}\\
   -g e^{-i\bar{\varphi}_b} & \kappa/2+i(\Delta-\Omega) \\
  \end{pmatrix}
\end{equation*}
where we defined $\tilde{\mathbf{M}}_\Delta$ to not be confused with the matrix $\mathbf{M}_\Delta$ defined earlier for a simple cavity.
Note that a genuine \textit{frequency-dependent} squeezing angle could be obtained by detuning the OPO cavity, but the frequency range over which the squeezing angle varies is limited by the cavity bandwidth, which is typically large compared to the frequency range of interest in our experiment. This phenomenon was realised experimentally a few years ago \cite{detunedOPOsqz}, but is not the focus of our work. \\

In the context of our work, we will assume : 
\begin{itemize}
  \item the pump phase is locked to $\bar{\varphi}_b = \{0,\pi\}$ i.e. amplification or deamplification regime,  
  \item the cavity is resonant $\Delta=0$,
\end{itemize}
We further normalise all frequencies to the cavity bandwidth $\kappa/2$ such that $\Omega \rightarrow \Omega/(\kappa/2)$ and $g \rightarrow g/(\kappa/2) = x$, such that the off-diagonal terms below can simply be written $\mp x$, factoring out the cavity bandwidth. We carry out the derivation for $\bar \varphi_b = 0$ (amplification) for simplicity, and the $\bar \varphi_b = \pi$ (deamplification) case is obtained by changing $x$ to $-x$ in the final expressions. The matrix QLE in Fourier space is written as
\begin{equation}
     \tilde{\mathbf{M}}_0 \,  \mathbf{\delta \hat{a}}[\Omega]  = \sqrt{\kappa_{\mathrm{1}}} \, \mathbf{\delta \hat{a}_{\mathrm{in}}}[\Omega]  + \sqrt{\kappa_2} \, \mathbf{\delta \hat{a}_{\mathrm{vac}}}[\Omega]  
\end{equation}
with 
\begin{equation*}
  \tilde{\mathbf{M}}_0 = \frac{\kappa}{2}\begin{pmatrix}
  1-i2\Omega/\kappa & - x\\ 
  - x  & 1-i2\Omega/\kappa\\ 
  \end{pmatrix} 
\end{equation*}

\noindent \textbf{Transfer matrices and Spectra: }
As before with a simple cavity, the transmitted quadratures at resonance are then 
\begin{equation}
\begin{split}
  \mathbf{\delta \hat{u}_{\mathrm{OPO}}}[\Omega] & =  \mathbf{T}_{\mathrm{OPO}}[\Omega] \mathbf{\delta \hat{u}_{\mathrm{in}}}[\Omega] +  \mathbf{L}_{\mathrm{OPO}}[\Omega] \, \mathbf{\delta \hat{u}_{\mathrm{vac}}}[\Omega] 
\end{split}
\end{equation}
where we defined the transfer matrices for the input and loss ports as
\begin{equation*}
  \mathbf{T}_{\mathrm{OPO}}[\Omega]=  \sqrt{\kappa_1 \kappa_2 } \,\mathbf{\Gamma}  \tilde{\mathbf{M}}^{-1}_0 \mathbf{\Gamma}^{-1}, \quad \mathbf{L}_{\mathrm{OPO}}[\Omega]=  \kappa_2 \,\mathbf{\Gamma}  \tilde{\mathbf{M}}^{-1}_0 \mathbf{\Gamma}^{-1}- \mathbf{1}.
\end{equation*}

After a bit of algebra, the covariance matrix of the transmitted field at $\bar \varphi_b = 0$ is then computed as
\begin{equation}
      \mathbf{S}^{0}_{\rm OPO}[\Omega] =\begin{pmatrix}
        1 + \dfrac{\kappa_2}{\kappa} \dfrac{4x}{(1- x)^2+ \left(2\Omega/\kappa\right)^2} & 0 \\[10pt]
        0 & 1 - \dfrac{\kappa_2}{\kappa} \dfrac{4x}{(1+ x)^2+ \left(2\Omega/\kappa\right)^2} 
      \end{pmatrix}
      \label{eq:II85}
\end{equation}




On a side note, when deriving the noise spectra for the intracavity field, the maximum amount of squeezing is limited to 3 dB, while the transmitted field can feature arbitrarily high squeezing levels. This is interpreted as additional correlations between vacuum fluctuations being reflected at the output port of the OPO and the squeezed field leaking from this very same output port, allowing for strong squeezing. \\

\noindent \textbf{The perfect squeezer: } Starting from Eq.\eqref{eq:II85}, in the idealized limit of perfect escape efficiency ($\eta_{\rm esc}=1$) and for analysis frequencies much smaller than the cavity bandwidth ($\Omega/\kappa \to 0$), the expression simplifies to

\begin{equation}
      \mathbf{S}^{0}_{\rm OPO}[\Omega] =\begin{pmatrix}
         \dfrac{(1 + x)^2}{(1 - x)^2} & 0 \\[10pt]
        0 & \dfrac{(1 - x)^2}{(1 + x)^2} 
      \end{pmatrix}
\end{equation}
Introducing the standard squeezing parameter $r$ through the relation $x=\tanh \frac{r}{2}$, one can rewrite the numerator and denominator as
\begin{equation*}
1 + \tanh \frac{r}{2} = \frac{e^{+ \frac{r}{2}}}{\cosh \frac{r}{2}}, 
\qquad
1 - \tanh \frac{r}{2} = \frac{e^{- \frac{r}{2}}}{\cosh \frac{r}{2}},
\end{equation*}
such that
\begin{equation*}
\frac{(1 \pm \tanh \frac{r}{2})^2}{(1 \mp \tanh \frac{r}{2})^2}
= \left(\frac{e^{\pm \frac{r}{2}}}{e^{\mp \frac{r}{2}}}\right)^2
= e^{\pm 2r}.
\end{equation*}
Thus when $\bar\varphi_b=\{0,\pi\}$, in the lossless, low-frequency limit the transmitted noise levels reduce to the well-known parametric result
\begin{equation}
      \mathbf{S}^0_{\rm OPO}[\Omega] =\begin{pmatrix}
         e^{+ 2r} & 0 \\
        0 & e^{- 2r} 
      \end{pmatrix} \quad \quad \text{and} \quad \quad
      \mathbf{S}^\pi_{\rm OPO}[\Omega] =\begin{pmatrix}
         e^{- 2r} & 0 \\
        0 & e^{+ 2r}
      \end{pmatrix}
      \label{eq:perfect_squeezer}
\end{equation}
where we can now establish that an amplified field ($\bar\varphi_b=0$) corresponds to a squeezed phase quadrature and an anti-squeezed amplitude quadrature, while a deamplified field ($\bar\varphi_b=\pi$) corresponds to a squeezed amplitude quadrature and an anti-squeezed phase quadrature. 
Later on, we will use this idealized expression to describe how squeezed light interacts with a mechanical resonator whose frequency is much smaller than the OPO bandwidth. \\

\noindent \textbf{Losses: } Squeezing is very sensitive to optical losses, which couple uncorrelated vacuum fluctuations into the squeezed field and degrade the squeezing level. The escape efficiency $\eta_{\rm esc}=\kappa_2/\kappa$ of the OPO cavity yields a first lower bound, but there are many others in a real experiment: propagation loss mechanisms in real experiments: mode-mismatch, non-unity quantum efficiency of the photodetectors, etc. One can then distinguish between \textit{intracavity} losses, which are accounted for in the escape efficiency, and \textit{extracavity} losses, which we denote by $\eta_{\rm ext}$ and lump all other loss mechanisms into a single effective loss. The effect of these losses can be modeled as a beam-splitter mixing the squeezed field with vacuum fluctuations, such that the lossy covariance matrix is given by
\begin{equation}\mathbf{S}_{\rm det}[\Omega] = (1-\eta) \, \mathbf{S}^{\bar \varphi_b}_{\mathrm{OPO}}[\Omega] + \eta \, \mathbf{1}
  \label{II.xx3}
\end{equation}
This expression is actually true for any Gaussian state suffering from losses. As seen in Fig.\ref{fig:OPO_noises}, losses have a dramatic effect on the squeezing level. Fig.\ref{fig:OPO_noises}(a) shows the frequency dependence of the squeezing and antisqueezing levels for various escape efficiencies, while Fig.\ref{fig:OPO_noises}(b) shows the squeezing and antisqueezing levels at frequencies inside teh cavity bandwidth as a function of the escape efficiency, where we notice losses have a stronger effect on squeezing than antisqueezing. Finally, Fig.\ref{fig:OPO_noises}(c) shows how extrinsic losses degrade the output squeezing level as a function of the input squeezing level (right at the OPO output). For example, 1 \% losses will limit the maximum squeezing level to $\sim 20$ dB, while 50 \% losses will limit it to $\sim 3$ dB only. Ths squeezing-antisqueezing asymetry is then used to estimate the total losses. 
\begin{figure}[h!]
\centering
\includegraphics[width=\textwidth]{./chap2/fig/OPONoises.pdf}
\caption{Squeezing degradation properties of a non-perfect OPO generating 5 dB of squeezing. 
(a) Squeezing-Antisqueezing levels obtained as a function of frequency (in cavity linewidth unit). The squeezing-antisqueezing levels are maximised at 100\% escape efficiency and inside the cavity linewidth (see dark red and dark blue curves). 
(b) Squeezing-antisqueezing levels as a function of the escape efficiency. 
(c) Output Squeezing level as a function of the Input Squeezing level (right at the OPO output) considering various optical loss values (extrinsic losses). } 
\label{fig:OPO_noises}
\end{figure}


\noindent\textbf{Frequency dependence: } Similarly to what was seen earlier considering general quantum states, squeezing at an arbitrary angle $\theta$ can be obtained by rotating the covariance matrix. However, one can now make the squeezing angle frequency dependent
above as
\begin{equation}
\mathbf{S}^{\theta}_{\rm OPO}[\Omega] = \mathbf R(\theta[\Omega])
\mathbf{S}^0_{\rm OPO}[\Omega] 
\mathbf R^{\dagger}(\theta[\Omega]).
\label{II.xx4}
\end{equation}
The $\mathbf{S}[\Omega]$ noise matrix can either be the full cavity one, or the idealized one. As already mentionned, the mechanical frequencies of interest will be deep in the OPO bandwidth such that we will use the ideal squeezer expression \eqref{eq:perfect_squeezer} in addition with extrinsic losses \eqref{II.xx3}. The explicit expression of the covariance matrix at a frequency-dependent angle is then
\begin{equation}
      \mathbf{S}^{\theta}_{\rm OPO}[\Omega] =\begin{pmatrix}
         \cosh 2r  + \sinh 2r \, \cos 2\theta[\Omega]  & -\sinh 2r \, \sin 2\theta[\Omega]  \\[10pt]
        -\sinh 2r \, \sin 2\theta[\Omega]  & \cosh 2r  - \sinh 2r \, \cos 2\theta[\Omega] 
      \end{pmatrix}
\end{equation}


\subsection{Optomechanical Cavities}
We now turn to standard optomechanical cavities. As in the simple Fabry-Perot case, we consider a cavity mode, in which we now allow one of the coupler (traditionnaly the output coupler), to be itself a \textit{mechanical} harmonic oscillator with annihilation operator $\hat{c}$, effective mass $m$, angular resonance frequency $\Omega_m$ and damping rate $\Gamma_m$. In canonical optomechanical systems the mechanics operators are usually denoted as $\hat b$ but in our case it would be redundant with the operators describing the pump field in non-linear systems. The position can be expressed in terms of our bosonic operators as  $\hat{x}=x_0(\hat{c}+\hat{c}^{\dagger})$ with $x_0 = \sqrt{\hbar/(2m\Omega_m)}$ the resonator's zero-point fluctuations. 

\begin{figure}[h!]
\centering
\includegraphics[width=\textwidth]{./chap2/fig/CavityOM.pdf}
\caption{Diagram generic optomechanical system. $\hat a$'s are the various fields at play, $\kappa$'s are the various couplings involved, with $\delta \hat a$'s the associated fluctuations. $\hat x$ is the quantum position operator of the mechanical resonator which linearly shifts the cavity resonance frequency. } 
\label{fig:cavity_types}
\end{figure}

\subsubsection{Mechanics \& Radiation Pressure Force }
The classical equation of motion of such an oscillator is
\begin{equation}
  m \, \ddot{ x} = -m \, \Omega_m^2 x - m \Gamma_m \dot{x} + \hat F
\end{equation}
where $\hat F$ is the total force acting on the oscillator.
In Fourier space, we recover the standard linear response form
\begin{equation}
  x [\Omega] = \chi[\Omega] \hat F[\Omega] \quad \text{with} \quad \chi[\Omega] = \frac{1}{m(\Omega_m^2 - \Omega^2 - i \Gamma_m \Omega)}
\end{equation}
where $\chi[\Omega]$ is the susceptibility linearly relating the position $ x [\Omega]$ to the external force $\hat F[\Omega]$. This susceptibility can also be written as 
\begin{equation}
  \chi[\Omega] = |\chi[\Omega]| e^{i \phi_m[\Omega]}
\end{equation}
with 
\begin{equation*}  
  \quad \text{with} \quad \phi_m[\Omega] = \arctan\Big(\frac{\Gamma_m \Omega}{\Omega_m^2 - \Omega^2}\Big) \quad \text{and} \quad |\chi[\Omega]| = \frac{1}{m\sqrt{(\Omega_m^2 - \Omega^2)^2 + (\Gamma_m \Omega)^2}}.
\end{equation*}

We can define the mechanical quality factor, defined as 
\begin{equation}
  Q = \frac{\Omega_m}{\Gamma_m}
\end{equation}
which is the number of oscillations before the resonator's energy is damped by a factor $1/e$. On resonance, the susceptibility is purely imaginary and reads $\chi[\Omega_m] = -i Q/(m \Omega_m^2)$. \\ 

As before, the position is also linearized considering small quantum fluctuations compared to its mean value, such that we write $\hat{x}= x + \delta \hat{x}$. Importantly, the total position fluctuations $\delta \hat{x} = \Sigma \delta \hat{x}_i$ are the sum of individual fluctuations that can arise from various sources, such as a the zero-point fluctuations, thermal fluctuations or radiation pressure induced fluctuations. In the following we will only consider a radiation pressure induced fluctuations $\delta \hat{x}_{\mathrm{RPN}}$, such that $\delta \hat{x} =  \delta \hat{x}_{\mathrm{RPN}}$. \\ 

Due to the continuous yet discrete photon \textit{hits} at a rate exceeding the resonator frequency, the resonator \textit{feels} an effective force. This radiation pressure force is expressed as 
\begin{equation}
  \hat F = 2 \frac{\hbar k_L}{\tau_c}\hat{a}^\dagger \hat{a} = 2 \frac{\hbar k_L}{\tau_c} \, |\bar{\alpha}|^2 +2 \frac{\hbar k_L}{\tau_c} \, |\bar \alpha| \, \delta \hat{p} + \mathcal{O}(\delta \hat{a}^\dagger \delta \hat{a})
\label{eq:Frad}
\end{equation}
where $k_L = 2\pi / \lambda$ is the laser wavevector, and $\tau_c=2L/c$ is the cavity round-trip time, and we neglect second-order terms. This force then features a static component shifting the resonator away from its equilibrium position, that be the $x$ component, as well as a fluctuating component $\delta \hat F\propto \delta \hat{p}$ which forces the resonator to jitter around its mean displacement: $\delta \hat x_{\mathrm{RPN}}$. The position mean value and its fluctuations under radiation pressure can therefore be expressed to first order as 
\begin{equation}
  x = \frac{2\hbar k_L |\bar \alpha|^2}{\tau_c}  \,  \chi[0] \, , \quad \delta \hat x_{\mathrm{RPN}} [\Omega]= \frac{2\hbar k_L |\bar \alpha|}{\tau_c}  \,  \chi[\Omega] \,  \delta \hat{p}[\Omega]. \label{eq:dx}
\end{equation} 



\subsubsection{Optomechanical QLE}
Considering an optomechanical cavity of length $L$ at rest, such that the average resonator position is initialy 0, the bare cavity free spectral range is given by $\omega_{\mathrm{FSR}}=\pi c /L$ and the cavity frequency $\omega_c = N \omega_{\mathrm{FSR}}$. Injecting light inside this cavity then shifts the mechanical resonator position as seen above, which in turn changes the cavity length $L \rightarrow L+x$, thus its resonance frequency. Writing the Hamiltonian, we simply make a Taylor expansion to first order in $\hat{x}$ the cavity frequency $\omega_c(\hat{x})=\omega_c + \hat x \, \partial \omega_c / \partial x$ such that we have: 
\begin{equation}
\hat{H} = - \hbar \Delta  \hat{a}^\dagger \hat{a} + \hbar \, G  \hat{x} \, \hat{a}^{\dagger} \hat{a} + \hbar \, \Omega_m \hat{c}^\dagger \hat{c}
\end{equation}
where $G=  \partial \omega_c / \partial x = - \omega_c/L$ (in the simple linear case). One can also identify a useful identity by considering the radiation pressure force \eqref{eq:Frad} and the Hamiltonian above, such that
\begin{equation}
  \hat F_{\textrm{rad}} = - \frac{\partial \hat H}{\partial \hat x} = - \hbar G \hat{a}^\dagger \hat{a} \quad \Rightarrow \quad G = - 2 \frac{k_L}{\tau_c}
\end{equation}
which is consistent with our previous expression of $G$ such that we rewrite the position fluctuations as $\delta \hat x_{\mathrm{tot}} [\Omega]= - \hbar G |\bar \alpha| \chi[\Omega]\, \delta\hat p[\Omega]$. 
Plugging in the QLE and ignoring vacuum and loss fluctuations for notational simplicity, the field equations are written as 
\begin{equation}
  \left\{
  \begin{split}
  0 &= -\Big(\frac{\kappa}{2}-i\bar\Delta\Big) \bar{\alpha} + \sqrt{\kappa_1} \, |\bar{\alpha}_{\mathrm{in}}| \\
  \mathbf{\delta \dot{\hat{a}}}(t)&= - \begin{pmatrix}
  \kappa/2-i\bar\Delta & 0 \\ 
   0 &\kappa/2+i\bar\Delta \\ 
  \end{pmatrix}  \mathbf{\delta \hat{a}}(t) + iG\bar{\alpha}\delta \hat{x} \begin{pmatrix} +1 \\ -1\end{pmatrix}  +  \sqrt{\kappa_{\mathrm{1}}} \, \mathbf{\delta \hat{a}_{\mathrm{in}}}(t)  + \sqrt{\kappa_2} \, \mathbf{\delta \hat{a}_{\mathrm{vac}}}(t) 
  \end{split}
  \right.
\end{equation}
where we have introduced the radiation pressure induced detuning $\bar\Delta = \Delta - G x$ (the mean resonator displacement shifts the cavity frequency, hence the detuning), and where we have assumed the input field to be real. \\ 

This so-called \textit{dispersive} coupling, where the cavity frequency $\omega_c(x)$ linearly depends on the resonator's position to first order, is the hallmark of the optomechanical interaction. In the canonical model, the cavity linewidth $\kappa$ does not depend on the resonator's position. \\

\noindent \textbf{Mean field solution \& Bistability:}
Writing the mean intracavity amplitude by keeping the \textit{undisturbed} detuning $\Delta$ for clarity and substituting for the static displacement $x$, we get
\begin{equation}
  \bar{\alpha} = \frac{\sqrt{\kappa_1}}{ \kappa/2 - i \left( \Delta - \dfrac{\hbar G^2  |\bar{\alpha}|^2}{m_{\text{eff}} \Omega_m^2 } \right)} |\bar{\alpha}_{\text{in}}|
\label{eq:alpha}
\end{equation}
where the $|\bar{\alpha}|^2$ dependence in disguise in the mean mechanical displacement is the root of the bistable behaviour of optomechanical cavities. We show the induced hysteresis in Fig.\ref{fig:bistab}. \\ 
 \begin{figure}[h!]
\centering
\includegraphics[width=\textwidth]{./chap2/fig/bistab_theos.pdf}
\caption{Optomechanical bistability. Intracavity intensity as a function of the input intensity for various detunings $\Delta$. The different colors correspond to increasing optical input powers. The dashed lines correspond to unstable solutions.} 
\label{fig:bistab}
\end{figure}


For moderate injected powers, this is the standard intracavity field formula where we simply relabel $\Delta - Gx \rightarrow \Delta$ to lighten the notation. When resonant, the intracavity field does not pick up any phase and is real i.e. $\bar{\alpha} = |\bar{\alpha}| = 2\sqrt{\kappa_1}/\kappa \, |\bar{\alpha}_{\mathrm{in}}|$. \\ 

Optomechanical cavities may display optical ringdowns, as detailed in the cavity subpart above, but this is a purely optical phenomenon: the mechanics plays no role in this optical ringdown. \\


\noindent \textbf{Fluctuations solution:} As previously, going to Fourier space now yields 
\begin{equation}
     \mathbf{M}_{\bar\Delta} \mathbf{\delta \hat{a}}[\Omega]  = i \,  G \bar{\alpha}_{\mathrm{}} \delta \hat{x}[\Omega]   \begin{pmatrix} +1 \\ -1\end{pmatrix}  +\sqrt{\kappa_{\mathrm{1}}} \, \mathbf{\delta \hat{a}_{\mathrm{in}}}[\Omega]  + \sqrt{\kappa_2} \, \mathbf{\delta \hat{a}_{\mathrm{vac}}}[\Omega] 
\end{equation}
where we have injected the mean field solution \eqref{eq:alpha} in our equations, assuming moderate input power to ignore bistable behaviour. We focus on the resonant case to derive the noise spectra, such that $\mathbf{M}_0 = (\kappa/2 - i\Omega) \mathbf{I}$  and the intracavity quadratures are
\begin{equation}
  \mathbf{\delta \hat{u}}[\Omega] =  \frac{2G|\bar \alpha|}{\kappa/2 - i\Omega}\delta \hat{x}[\Omega] \begin{pmatrix} 0\\ 1\end{pmatrix}  +\frac{\sqrt{\kappa_{\mathrm{1}}}}{\kappa/2 - i\Omega} \, \mathbf{\delta \hat{u}_{\mathrm{in}}}[\Omega]  +\frac{\sqrt{\kappa_{\mathrm{2}}}}{\kappa/2 - i\Omega} \, \mathbf{\delta \hat{u}_{\mathrm{vac}}}[\Omega]  
\end{equation}
Writing explicitely the amplitude-phase quadratures then gives 
\begin{equation}
  \begin{split}
    \delta \hat{p}[\Omega] &= \frac{\sqrt{\kappa_{\mathrm{1}}}}{\kappa/2 - i\Omega} \, \delta \hat{p}_{\mathrm{in}}[\Omega] + \frac{\sqrt{\kappa_{\mathrm{2}}}}{\kappa/2 - i\Omega} \, \delta \hat{p}_{\mathrm{vac}}[\Omega] \\
    \delta \hat{q}[\Omega] &=  \frac{2G|\bar \alpha|}{\kappa/2 - i\Omega}\delta \hat{x}[\Omega]  +\frac{\sqrt{\kappa_{\mathrm{1}}}}{\kappa/2 - i\Omega} \, \delta \hat{q}_{\mathrm{in}}[\Omega]  + \frac{\sqrt{\kappa_{\mathrm{2}}}}{\kappa/2 - i\Omega} \, \delta \hat{q}_{\mathrm{vac}}[\Omega] 
  \end{split}
   \label{eq:intra_quad}
\end{equation}
This expression highlights the fact that for a cavity at resonance, only the phase is affected by the resonator position fluctuations. Physically, this can be understood by considering first that a fluctuating field amplitude leads to a fluctuating radiation pressure force, which in turn \textit{shakes} the mechanical resonator, which changes the phase of the field reflected. The reciprocal process does not happen: a fluctuating phase does not lead to a fluctuating radiation pressure force, hence the output amplitude fluctuations are unaffected by the mechanics. \\

Importantly, considering the field reflected off the cavity, we define the displacement to phase fluctuation transduction $\mathcal{C}[\Omega]$ such that 
\begin{equation}
  \delta \hat{q}_{\mathrm{ref}}[\Omega] = \mathcal{C}[\Omega] \, \delta \hat{x}[\Omega] \quad \text{with} \quad \mathcal{C}[\Omega] = \frac{2\sqrt{\kappa_1}G|\bar \alpha|}{\kappa/2 - i\Omega} = \frac{\kappa_1}{\kappa} \dfrac{16 \mathcal{F}\sqrt{\bar{I}_{\rm in}}}{\lambda (1- i2\Omega/\kappa)}
\end{equation}
where we plugged in useful experimental parameters $\mathcal{F}$, $\lambda$ and $\bar I_{\rm in}$. The prefactor $\kappa_1/\kappa$ is the analog of the escape efficiency for optomechanical cavities, and is unity for single-port cavities. We stress that the total phase fluctuations are the sum of various contributions, including the input phase fluctuations, the vacuum fluctuations entering from the loss port, and the position induced phase fluctuations, whether they arise from radiation pressure or other sources. This transduction factor will be used later to express the displacement sensitivity/spectra in terms of experimental parameters. \\


Plugging in the position fluctuations derived earlier (Eqs.\eqref{eq:dx} and \eqref{eq:Frad}) in the intracavity phase fluctuations we get
\begin{equation}
  \begin{split}
\delta \hat{q}[\Omega] =  &\frac{ \mathcal{C}^2[\Omega]}{2\kappa_1}  \,\hbar  \chi[\Omega] \, \bigg( \sqrt{\kappa_1} \, \delta \hat{p}_{\mathrm{in}}[\Omega] + \sqrt{\kappa_2} \, \delta \hat{p}_{\mathrm{vac}}[\Omega] \bigg) \\ 
 & +\frac{1}{\kappa/2 - i\Omega} \, \bigg( \sqrt{\kappa_1} \, \delta \hat{q}_{\mathrm{in}}[\Omega] + \sqrt{\kappa_2} \, \delta \hat{q}_{\mathrm{vac}}[\Omega] \bigg) 
  \end{split}
\end{equation}
such that we can readily express the intracavity quadratures in matrix form as
\begin{equation}
\delta\mathbf{\hat{\mathbf u}} [\Omega]
=
\left(
\begin{array}{cc}
\dfrac{1}{\kappa/2 - i\Omega} & 0\\[6pt]
\dfrac{\mathcal{K}[\Omega]}{\kappa_1} & \dfrac{1}{\kappa/2 - i\Omega}
\end{array}
\right)
\bigg( 
\sqrt{\kappa_1} \delta\mathbf{\hat{\mathbf u}} _{\mathrm{in}}[\Omega] +\;
\sqrt{\kappa_2} \delta\mathbf{\hat{\mathbf u}} _{\mathrm{vac}}[\Omega] \bigg)  \, .
\end{equation}
with
\begin{equation}
  \mathcal{K}[\Omega] = \frac{ \mathcal{C}^2[\Omega]}{2}  \,\hbar  \chi[\Omega]= \Bigl(\dfrac{\kappa_{\mathrm{1}}}{\kappa}\Bigr)^2 \dfrac{128 \hbar \mathcal{F}^2 \bar I_{\rm in}}{\lambda^2(1 - i2\Omega/\kappa)^2}  \,  \chi[\Omega]
\label{eq:K}
\end{equation}

We then obtain the reflected and transmitted quadrature fluctuations 
\begin{equation}
\begin{split}
\mathbf{\delta \hat{u}_{\mathrm{ref}}} &= 
\mathbf{T}_{\mathrm{ref}}
\delta\hat{\mathbf u}_{\mathrm{in}} + 
\mathbf{L}_{\mathrm{ref}}
\delta\hat{\mathbf u}_{\mathrm{vac}} \, \\ 
\mathbf{\delta \hat{u}_{\mathrm{trans}}} &= \mathbf{T}_{\mathrm{trans}}  \delta\hat{\mathbf u}_{\mathrm{in}} +
\mathbf{L}_{\mathrm{trans}}
\delta\hat{\mathbf u}_{\mathrm{vac}} \, .
\end{split}
\end{equation}
 where we defned the transfer and loss  matrices 
\begin{equation*}
  \begin{alignedat}{3}
     \mathbf{T}_{\mathrm{ref}} &=\begin{pmatrix}
  \dfrac{\kappa_1}{\kappa/2-i\Omega}  -1  & 0 \\
  \mathcal{K}[\Omega]  &  \dfrac{\kappa_1}{\kappa/2-i\Omega}  -1 
\end{pmatrix}  \quad
 & \mathbf{L}_{\mathrm{ref}} &= \begin{pmatrix}
   \dfrac{ \sqrt{\kappa_1 \kappa_2}}{\kappa/2-i\Omega}   & 0 \\
  \sqrt{\dfrac{\kappa_2}{\kappa_1}} \, \mathcal{K}[\Omega] &   \dfrac{ \sqrt{\kappa_1 \kappa_2}}{\kappa/2-i\Omega}  
\end{pmatrix} \\
        \mathbf{T}_{\mathrm{trans}} &=  \begin{pmatrix}
   \dfrac{ \sqrt{\kappa_1 \kappa_2}}{\kappa/2-i\Omega}   & 0 \\
  \sqrt{\dfrac{\kappa_2}{\kappa_1}} \, \mathcal{K}[\Omega] &   \dfrac{ \sqrt{\kappa_1 \kappa_2}}{\kappa/2-i\Omega}  
\end{pmatrix}  \quad 
& \mathbf{L}_{\mathrm{trans}}&= \begin{pmatrix}
  \dfrac{\kappa_2}{\kappa/2-i\Omega}  -1  & 0 \\
  \dfrac{\kappa_2}{\kappa_1}\mathcal{K}[\Omega]  &  \dfrac{\kappa_2}{\kappa/2-i\Omega}  -1 
\end{pmatrix} 
  \end{alignedat}
\end{equation*}


\noindent \textbf{Convergence to Virgo-LIGO notation:} To sanity check this expression, we need to make sure we recover the standard expressions used in the Virgo-LIGO community. We will assume the mechanical resonator is free, that is $\Omega \gg \Omega_m$ and $\Gamma_m \rightarrow 0$. The susceptibility then reduces to $\chi[\Omega] = -1/ M \Omega^2$, and we will consider sideband frequencies $\Omega \ll \kappa/2$ such that all the filtering terms can be neglected. We also consider a single-port cavity such that $\kappa_1 = \kappa$ and $\kappa_2=0$. The reflected quadrature fluctuations then read
\begin{equation}
\mathbf{\delta \hat{u}_{\mathrm{ref}}}
=
\begin{pmatrix}
  1   & 0 \\[12pt]
  \dfrac{32 \omega_0P_{\mathrm{in}}}{M L^2\kappa^2 \Omega^2}    &  1
\end{pmatrix}
\delta\hat{\mathbf u}_{\mathrm{in}}.
\end{equation}
In GW papers, the prefactor will often be 8 (and not 32) as they use the cavity halfwidth at half maximum rather than $\kappa$. We indeed recover the standard expression used in the GW community, which is a good sanity check of our derivation. We do stress however that this expression is only valid for a free mass, and that the full expression including the mechanical resonance is required to describe optomechanical cavities in general. \\


\noindent \textbf{Reflected spectra:} We can now compute the covariance matrix of the reflected quadratures, assuming vacuum fluctuations both at the input and at the loss port. We additionally consider a quasi single-port cavity for simplicity $\kappa_1 \gg \kappa_2$, such that $\kappa_1\sim\kappa$, as well as the bad cavity limit $\Omega \ll \kappa/2$. The reflected covariance matrix is then given by
\begin{equation}
      \mathbf{S}_{\rm ref} =\mathbf{T}_{\mathrm{ref}} \mathbf{S}_{\text{in}} \mathbf{T}_{\mathrm{ref}}^{\dagger}  = \begin{pmatrix}
        1 &  \mathcal{K}[\Omega]\\[6pt]
          \mathcal{K}^{*}[\Omega]& 1 +  |\mathcal{K}[\Omega]|^2
      \end{pmatrix}
\end{equation}
where the off-diagonal terms are complex conjugates of each other, ensuring the covariance matrix is Hermitian as required. The diagonal terms are the amplitude and phase noise spectra respectively, while the off-diagonal terms quantify correlations between amplitude and phase. The presence of these correlations is the hallmark of optomechanical/ponderomotive squeezing i.e. using the non-linear response of the resonator to squeeze light. This effect is not seen neither sought in our experiment, but is a very active field of research in the optomechanics community. \\

One now sees two essential components in the reflected phase spectrum. The first is the direct phase fluctuations, which is simply shot noise seen as $1$. The second is the back-action term $\propto |\mathcal{K}[\Omega]|^2$, which is the phase fluctuations induced by the resonator motion driven by radiation pressure fluctuations. 


\section{Detection}
Having layed out the theoretical framework to describe the optical fields interacting with our various cavities, we now turn to the detection schemes used to probe these fields. To detect the optical field reflected or transmitted from these optical systems, we will use two main techniques: direct detection and balanced homodyne detection. 
\begin{figure}[h!]
\centering
\includegraphics[width=\textwidth]{./chap2/fig/detection.pdf}
\caption{Detection schemes used to probe the optical fields reflected or transmitted from the cavities. (a) Direct detection scheme, where the intensity of the field is measured using a photodiode. $\eta_d$ represents the detection efficiency. (b) Direct detection of the interference between a signal and a local oscillator (LO) field. (c) Direct detection of two fields at slightly different frequencies (smaller than the PD bandwidth). (d) Balanced homodyne detection scheme, where the signal field is interfered with a strong local oscillator (LO) field on a 50:50 beam-splitter, and the difference of intensity between the two output ports is measured, allowing effective access to arbitrary quadrature fluctuations of the signal field, while rejecting classical noises. }
\label{fig:cavity_types}
\end{figure}


\subsection{Direct detection}
Direct detection consists in measuring the intensity of the optical field impinging on a photodiode. We will detail three cases: the single field case, where only the signal field is incident on the photodiode, the case where a local oscillator (LO) field is added to the signal field, and finally the case where two beams at slightly different frequencies are incident on the photodiode. \\

\noindent \textbf{Single field: } The photocurrent operator, originating from the photoelectric effect is given by
\begin{equation}
  \begin{split}
    \hat I &= e \, \hat a^{\dagger}\hat a^{\vphantom{\dagger}}
  \end{split}
\end{equation}
with $e$ the electron charge. We introduce the quantum efficiency of the photodiode $\eta_d$ to account for non unity detection efficiency, such that the detected field operator is written as 
\[\hat a \rightarrow \sqrt{\eta_d} \, \hat a + \sqrt{1-\eta_d} \, \hat a_{\mathrm{vac}}\]
where $\hat a_{\mathrm{vac}}$ are vacuum fluctuations entering due to non-unity detection efficiency. Assuming a real mean field $\bar \alpha$, photocurrent operator then reads
\begin{equation}
    \hat I = \eta_d \, e \, \bigg(|\bar \alpha|^2 + \bar \alpha \delta p + \sqrt{\eta_d(1-\eta_d)} \, \bar \alpha \, \delta p_{\mathrm{vac}}\bigg)
\end{equation}
where we neglected second-order terms. The photocurrent fluctuations in Fourier space are then given by
\begin{equation}
    \delta \hat I[\Omega] = \eta_d \, e \, \bar \alpha \, \left( \delta p[\Omega] +  \sqrt{\dfrac{1-\eta_d}{\eta_d}} \,  \delta p_{\mathrm{vac}}[\Omega] \right)
\end{equation}
such that the photocurrent noise spectrum is 
\begin{equation}
    S_{II}[\Omega] =  \eta_d^2 \, e^2 \, |\bar \alpha|^2 \, \left( S_{pp}[\Omega] + \dfrac{1-\eta_d}{\eta_d} \right) 
\end{equation}
where $S_{pp}[\Omega]$ is the amplitude quadrature noise spectrum of the incident field. This expression highlights that direct detection is only sensitive to amplitude quadrature fluctuations. \\

\noindent \textbf{Two fields: } Let's now consider an auxiliary field at the same frequency $\hat a_{\mathrm{LO}}$, called the local oscillator (LO), which is a coherent field dephased from our real signal field $\hat a$ by a phase $\phi_{\mathrm{LO}}$ such that the total field impinging on the photodiode is $\hat a_{\mathrm{tot}}= \hat a + \hat a_{\mathrm{LO}}$. So far, we do not consider the LO to be consequently stronger than the signal field, as we will do in the homodyne detection. This coherent addition can be performed using a beam-splitter or a polarizing beam-splitter, depending on the experimental implementation. The mean field of the total field is then given by $\bar \alpha_{\mathrm{tot}} = \bar \alpha + |\bar \alpha_{\mathrm{LO}}|e^{i\phi_{\mathrm{LO}}}$, and its fluctuations are $\delta \hat a_{\mathrm{tot}} = \delta \hat a + \delta \hat a_{\mathrm{LO}}$. For simplicity we will assume a quantum efficiency of 1 in the following. The photocurrent operator mean values is then given by 
\begin{equation}
  \begin{split}
    \bar I &= e \, \bigg(|\bar \alpha|^2 + |\bar \alpha_{\mathrm{LO}}|^2 + 2 |\bar \alpha| |\bar \alpha_{\mathrm{LO}}| \cos \phi_{\mathrm{LO}} \bigg)
  \end{split}
\end{equation}
where we see the interference term between the signal and the LO: scanning the LO phase $\phi_{\mathrm{LO}}$ (with a piezoelectric actuator) will lead to interference fringes on the mean photocurrent, which can be used to lock the LO phase. We won't developp the full expression of the photocurrent fluctuation spectrum here (see Annex B), as they feature a cumbersome linear combination of the amplitude and phase quadrature noise spectra of both the signal and the LO fields, as well as cross correlation terms between the two fields (if any), which is not very interesting experimentally. However, we can already sense that adding a LO field allows to access phase quadrature fluctuations of the signal field, which is not possible with direct detection alone. \\

Let's consider 'slow' (hence low frequency) classical fluctuations of the LO phase $\delta \phi_{\mathrm{LO}}(t)$ around a mean value $\bar \phi_{\mathrm{LO}}$, such that $\phi_{\mathrm{LO}}(t) = \bar \phi_{\mathrm{LO}} + \delta \phi_{\mathrm{LO}}(t)$ with $\delta \phi_{\mathrm{LO}}(t) \ll 1$. Developing the photocurrent to first order in these classical fluctuations, the mean photocurrent fluctuations then reads 
\begin{equation}
    \delta \bar I(t) \propto \delta \phi_{\mathrm{LO}}(t) 
\end{equation}
such that slow phase fluctuations of the LO are directly transduced into photocurrent fluctuations. The classical phase noise of the LO can therefore pollute the photocurrent noise spectrum at low frequency, as well as limit the lock stability of the LO phase. Let's now consider the case where the LO is phase modulated at a frequency $\Omega_{\mathrm{mod}}$ as seen previously 
\begin{equation*}
  \alpha_{\mathrm{LO}}(t) \approx \bar{\alpha}_{\mathrm{LO}} \Big( 1 + i \epsilon_{\phi} \cos(\Omega_{\mathrm{mod}} t) \Big)
\end{equation*}
such that the mean photocurrent fluctuations are now given by
\begin{equation}
    \delta \bar I(t) \propto \, \cos(\Omega_{\mathrm{mod}} t) \delta \phi_{\mathrm{LO}}(t)
\end{equation}
so that the LO phase noise is spectrally only transduced around the modulation frequency $\Omega_{\mathrm{mod}}$. Demodulating the photocurrent at $\Omega_{\mathrm{mod}}$ then yields an error signal proportional to the LO phase fluctuations, which can be used to lock the LO phase to a desired value $\bar \phi_{\mathrm{LO}}$, while rejecting low frequency phase noise of the LO. \\ 

\noindent \textbf{Two fields at different frequencies: } Finally, let's consider the case where the signal and LO fields are at slightly different frequencies, such that $\hat a$ is at frequency $\omega_0$ and $\hat a_{\mathrm{LO}}$ at frequency $\omega_0 +  \omega_{\text{beat}}$. The total field impinging on the photodiode is then written as $\hat a_{\mathrm{tot}} = \hat a + \hat a_{\mathrm{LO}} e^{-i \omega_{\text{beat}} t}$ since our operators are defined in a frame rotating at $\omega_0$. The mean photocurrent is then given by
\begin{equation}
  \begin{split}
    \bar I &= e \, \bigg(|\bar \alpha|^2 + |\bar \alpha_{\mathrm{LO}}|^2 + 2 |\bar \alpha| |\bar \alpha_{\mathrm{LO}}| \cos (\omega_{\text{beat}} t + \phi_{\mathrm{LO}}) \bigg)
  \end{split}
\end{equation}
where we see that the interference term now oscillates at the beatnote frequency $\omega_{\text{beat}}$. Demodulating the photocurrent at a frequency $\omega_{\text{ref}} \sim\omega_{\text{beat}}$, phase $\tilde \phi$, and low-pass filtering the photocurrent then gives 
\begin{equation}
\bar I_{\rm demod} \propto   \cos ((\omega_{\text{beat}} - \omega_{\text{ref}})t + \phi - \tilde \phi).
\end{equation}
This very signal can then be used to lock the frequency of an auxiliary laser to the desired frequency offset $\omega_{\text{ref}}$ from the main laser. However, this signal featuring many zero crossings, one needs to tune the auxiliary laser frequency close enough to the desired offset so that it ensures the feedback loop locks to the correct zero crossing. This is generally done manually by scanning the auxiliary laser frequency until the right zero crossing is found, confirmed by monitoring the beatnote on a spectrum analyzer. 

\subsection{Balanced Homodyne Detection}
Balanced homodyne detection (HD) is a common technique to measure arbitrary quadratures of an optical field with high sensitivity. It consists in mixing the signal field $\hat a$ with a strong local oscillator (LO) field $\hat a_{\mathrm{LO}}$ on a 50:50 beam-splitter, and detecting the two output ports with identical photodiodes. The beamsplitter operation reads
\begin{equation}
  \left\{
  \begin{split}
    \hat a_{\mathrm{out,1}} &= \frac{1}{\sqrt{2}} \big( \hat a + \hat a_{\mathrm{LO}} \big) \\
    \hat a_{\mathrm{out,2}} &= \frac{1}{\sqrt{2}} \big( \hat a - \hat a_{\mathrm{LO}} \big)
  \end{split}
  \right.
\end{equation}
The two photodiodes then measure the photocurrents $\hat I_1 = e \, \hat a_{\mathrm{out,1}}^{\dagger}\hat a_{\mathrm{out,1}}^{\vphantom{\dagger}}$ and $\hat I_2 = e \, \hat a_{\mathrm{out,2}}^{\dagger}\hat a_{\mathrm{out,2}}^{\vphantom{\dagger}}$. The HD photocurrent is then defined as the difference between the two photocurrents $\hat I_{\mathrm{BHD}} = \hat I_1 - \hat I_2$, which reads
\begin{equation}
  \hat I_{\mathrm{HD}} = e \, \big( \hat a_{\mathrm{LO}}^{\dagger} \hat a^{\vphantom{\dagger}} + \hat a^{\dagger} \hat a_{\mathrm{LO}}^{\vphantom{\dagger}} \big)
\end{equation}
Assuming a real mean field for the signal $\bar \alpha$ and a phase shifted LO mean field $\bar \alpha_{\mathrm{LO}} = |\bar \alpha_{\mathrm{LO}}| e^{i \phi_{\mathrm{LO}}}$ such that $|\bar \alpha_{\mathrm{LO}} |\gg |\bar \alpha|$, we can linearize the HD photocurrent to first order in the fluctuations as
\begin{equation}
  \hat I_{\mathrm{HD}} = 2 e \, |\bar \alpha_{\mathrm{LO}}| \, | \bar \alpha | \, \cos \phi_{\mathrm{LO}} + e \, |\bar \alpha_{\mathrm{LO}}| \, \big( \cos \phi_{\mathrm{LO}} \, \delta \hat p + \sin \phi_{\mathrm{LO}} \, \delta \hat q \big)
\end{equation}
where we recognise the mean photocurrent term in $2 e \, |\bar \alpha_{\mathrm{LO}}| \, | \bar \alpha | \, \cos \phi_{\mathrm{LO}}$ as in the two fields direct detection case. This slowly varying mean photocurrent can be used to lock the LO phase $\phi_{\mathrm{LO}}$ to a desired value, as previously described, with a piezoelectric actuator and phase modulation/demodulation scheme if needed. The HD photocurrent fluctuations in Fourier space are then given by
\begin{equation}
    \delta \hat I_{\mathrm{HD}}[\Omega] = e \, |\bar \alpha_{\mathrm{LO}}| \, \big( \cos \phi_{\mathrm{LO}} \, \delta \hat p[\Omega] + \sin \phi_{\mathrm{LO}} \, \delta \hat q[\Omega] \big)
\end{equation}
such that the HD photocurrent noise spectrum reads
\begin{equation}
    S_{II}^{\mathrm{HD}}[\Omega] =  e^2 \, |\bar \alpha_{\mathrm{LO}}|^2 \, \big( \cos^2 \phi_{\mathrm{LO}} \, S_{pp}[\Omega] + \sin^2 \phi_{\mathrm{LO}} \, S_{qq}[\Omega] + 2 \sin \phi_{\mathrm{LO}} \cos \phi_{\mathrm{LO}} \, S_{pq}[\Omega] \big)
\end{equation}
where $S_{pp}[\Omega]$, $S_{qq}[\Omega]$ and $S_{pq}[\Omega]$ are respectively the amplitude, phase and cross correlation noise spectra of the signal field. By tuning the LO phase $\phi_{\mathrm{LO}}$, one can therefore measure arbitrary quadratures of the signal field with high sensitivity thanks to the strong LO field amplifying the signal fluctuations. This is the main advantage of HD over direct detection, where only amplitude quadrature fluctuations can be measured. To calibrate the HD detection efficiency, one can block the signal field, such that the LO now probes vacuum fluctuations only. This reference is then used to evaluate the squeezing level of the signal field when unblocked. \\ 

The practical implementation of these detection schemes and the associated locks are detailed in chapter III. 

  





