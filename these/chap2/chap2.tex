% !TeX encoding = UTF-8
% !TeX spellcheck = fr_FR
% !TeX root = mythesis.tex
\chapter{ Theory: background} \label{chap:theory}
This chapter will cover the elementary concepts required to describe an membrane based optomechanical system in a quantum regime. We will first recall basics on optical field quantization as well describing coherent and squeezed light field, to then turn to the more specific frequency dependent squeezed light field. Secondly, we will cover the mathematical description of a mechanical resonator interacting with a generic coherent optical field, highlighting the differences with the seminal optomechanical system of a mirror on a spring. Finally, we will derive the equations of motions of a membrane based optomechanical system with frequency dependent squeezed optical fields. 

\section{Quantum Optics Concepts}
\subsection{Quantum Description of Light}
We introduce briefly field quantization concepts needed to describe monochromatic field propagation and measurements.
key words : eigenmodes of the field, quantization, annihilation operators, quadratures, phase space, displacement operators, squeezing operators, coherent states, generic squeezed states,


\subsection*{Quantised Electromagnetic Field}

We consider the quantised electromagnetic field in volume $V$. The electric field operator can be expressed as:
\begin{equation}
\hat{\mathbf{E}}(\mathbf{r}, t) = i \sum_{ \ell} \mathcal{E}_l \left[ \hat{a}_{\ell}(t) \mathbf{f}_{\ell}(\mathbf{r}) - \hat{a}_{\ell}^\dagger(t) \mathbf{f}_{\ell}^*(\mathbf{r}) \right]
\end{equation}
where $\mathcal{E}_l = \sqrt{\frac{\hbar \omega_l}{2 \varepsilon_0 V}}$ is the field per photon in mode $\ell$, $\hbar$ the reduced Planck constant, $\varepsilon_0$ the vacuum permittivity, $\mathbf{f}_{\ell}(\mathbf{r})$ are spatial mode functions satisfying orthonormality, and $\hat{a}_{\ell}(t)$, $\hat{a}_{\ell}^\dagger(t)$ are the time dependent annihilation and creation operators associated with each mode $(\ell)$. Here, we describe 

These operators satisfy the canonical commutation relations:
\begin{equation}
[\hat{a}_{\omega,\ell}(t), \hat{a}_{\omega',\ell'}^\dagger(t)] = \delta_{\omega,\omega'}\delta_{\ell,\ell'}, \quad [\hat{a}_{\omega,\ell}(t), \hat{a}_{\omega',\ell'}(t)] = 0
\end{equation}

The Hamiltonian for the electromagnetic field becomes a sum of harmonic oscillator energies:
\begin{equation}
\hat{H} = \sum_{\omega,\ell} \hbar \omega \left( \hat{a}_{\omega,\ell}^\dagger \hat{a}_{\omega,\ell} + \frac{1}{2} \right)
\end{equation}

\subsection*{Quadrature Operators}

To describe the phase space properties of a field mode, we define the quadrature operators:
\begin{align}
\hat{q}_{\omega,\ell} &= \frac{1}{\sqrt{2}}(\hat{a}_{\omega,\ell} + \hat{a}_{\omega,\ell}^\dagger) \\
\hat{p}_{\omega,\ell} &= \frac{1}{\sqrt{2}i}(\hat{a}_{\omega,\ell} - \hat{a}_{\omega,\ell}^\dagger)
\end{align}
These are Hermitian operators corresponding to measurable observables and satisfy:
\begin{equation}
[\hat{q}_{\omega,\ell}, \hat{p}_{\omega,\ell}] = i
\end{equation}

Generalised quadratures are defined via:
\begin{equation}
\hat{q}_{\theta}^{\omega,\ell} = \hat{q}_{\omega,\ell} \cos\theta + \hat{p}_{\omega,\ell} \sin\theta
\end{equation}

\subsection*{Uncertainty Principle and Quantum Noise}

From the commutation relation, the uncertainty principle follows:
\begin{equation}
\Delta q_{\omega,\ell} \Delta p_{\omega,\ell} \geq \frac{1}{2}
\end{equation}

This defines the minimum amount of quantum noise (vacuum fluctuations) in the electromagnetic field.

\subsection{*Coherent States}

% \begin{figure}
% \centering
% \includegraphics[width=0.5\textwidth]{./chap2/fig/tryout.pgf}
% \end{figure}
Coherent states $|\alpha_{\omega,\ell}\rangle$ are eigenstates of the annihilation operator:
\begin{equation}
\hat{a}_{\omega,\ell}|\alpha_{\omega,\ell}\rangle = \alpha_{\omega,\ell}|\alpha_{\omega,\ell}\rangle
\end{equation}
They can be generated by displacing the vacuum:
\begin{equation}
|\alpha_{\omega,\ell}\rangle = \hat{D}(\alpha_{\omega,\ell})|0\rangle, \quad \hat{D}(\alpha) = \exp(\alpha \hat{a}^\dagger - \alpha^* \hat{a})
\end{equation}
They exhibit:
\begin{itemize}
  \item Minimum uncertainty: $\Delta q = \Delta p = 1/\sqrt{2}$
  \item Classical-like dynamics
  \item Poissonian photon statistics
\end{itemize}

\subsection{Squeezed States}

Squeezed states reduce the variance of one quadrature below vacuum level:
\begin{align}
|\xi_{\omega,\ell}\rangle &= \hat{S}(\xi_{\omega,\ell}) |0\rangle \\
\hat{S}(\xi) &= \exp\left[\frac{1}{2}(\xi^* \hat{a}^2 - \xi \hat{a}^{\dagger 2})\right], \quad \xi = r e^{i\phi}
\end{align}
For phase quadrature squeezing ($\phi = 0$):
\begin{equation}
\Delta q_{\omega,\ell} = e^{-r}/\sqrt{2}, \quad \Delta p_{\omega,\ell} = e^{r}/\sqrt{2}
\end{equation}
Squeezed light is a key resource for precision metrology and quantum information.

\subsection{Field Operators}

The quantised vector potential and electric field can be expressed as:
\begin{align}
\hat{\mathbf{A}}(\mathbf{r}) &= \sum_{\omega,\ell} \sqrt{\frac{\hbar}{2\varepsilon_0 \omega V}} \left[ \hat{a}_{\omega,\ell} \mathbf{u}_{\omega,\ell}(\mathbf{r}) + \hat{a}_{\omega,\ell}^\dagger \mathbf{u}_{\omega,\ell}^*(\mathbf{r}) \right] \\
\hat{\mathbf{E}}(\mathbf{r}) &= i \sum_{\omega,\ell} \sqrt{\frac{\hbar \omega}{2\varepsilon_0 V}} \left[ \hat{a}_{\omega,\ell} \mathbf{u}_{\omega,\ell}(\mathbf{r}) - \hat{a}_{\omega,\ell}^\dagger \mathbf{u}_{\omega,\ell}^*(\mathbf{r}) \right]
\end{align}
These operators are central to describing interactions between light and matter in cavity QED, optomechanics, and other quantum platforms.

\vspace{1em}
This concludes our introduction to the quantum description of light, setting the stage for modelling interactions between quantum optical fields and mechanical resonators.

\subsection{\textcolor{red}{Optical Field Modulations}}
\subsection{Quantum Noise and Uncertainty}
\subsection{Sideband Representation}
\hspace{1pt}

\section{Optical Cavities : Basics}
\subsection{Cavity types and Resonance Conditions}
\subsection{Spatial and Longitudinal Modes}
\subsection{Static and Dynamical effects}
\hspace{1pt}

\section{\texorpdfstring{\color{red}Optical Cavities : Three Mirror Cavities}{Optical Cavities : Three Mirror Cavities}}
\subsection{}
\hspace{1pt}

\section{Cavity Optomechanics}
\subsection{Radiation Pressure Coupling}
\subsection{Quantum Langevin Equations}
\subsection{\texorpdfstring{Mechanical Resonators}{Mechanical Resonators}}
\subsection{Noise spectra}
\subsection{\texorpdfstring{\color{red} Three Mirror Cavities as Novel Optomechanical Systems}{Three Mirror Cavities as Novel Optomechanical Systems}}
\hspace{1pt}

\section{Squeezed Light Theory}
\subsection{Single-mode Squeezing}
\subsection{Noise Spectra }
\subsection{Frequency-dependent Squeezing and its use}
\hspace{1pt}

\section{\texorpdfstring{Numerical Simulations of \\ FDS Optomechanical Experiments}{Numerical Simulations of FDS Optomechanical Experiments}}

lets write smth here 
