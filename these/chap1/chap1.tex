\chapter*{Introduction}\label{chap:intro}
\addcontentsline{toc}{chapter}{Introduction} % ensures it appears in the main TOC
\mtcaddchapter

\section*{Historical background}
\addcontentsline{toc}{section}{Historical background} % ensures it appears in the main TOC


The field of quantum optomechanics investigates the interaction between light confined in an optical cavity and mechanical motion, which can take many forms: thermally driven vibrations, radiation-pressure induced displacements, gravitational waves or quantum zero-point fluctuations. The underlying principle is the same: photons change the resonator dynamics (via radiation pressuren photothermal effects, etc.), and conversely, the motion of the mechanical element modifies the cavity optical field. Measuring the output optical field thereby provides information about the mechanical motion (and/or the environment it is coupled to), while appropriately engineering the input fields enables control of the resonator dynamics. The crucial question that arises for both mechanical motion readout and control is: what are the fundamental limits imposed by quantum mechanics? \\

In the context of gravitational wave detection, it was realized in the 1970s that the sensitivity is ultimately limited by quantum fluctuations of the light field used to probe the mechanical motion \cite{caves1981}. The quantum nature of light imposes two contributions: shot-noise, arising from the intrinsic phase fluctuations of the optical field, and radiation-pressure noise, arising from the optical field amplitude fluctuations driving the mechanical resonator, and translating into an extra phase noise contribution in the detected optical field. These two contributions trade-off against each other, leading to an ultimate sensitivity known as the Standard Quantum Limit (SQL) \cite{braginsky_quantum_1992, clerk_introduction_2010}. For gravitational wave interferometers (GWI), where the goal is to measure the mechanical motion induced by gravitational waves, it was then proposed to inject squeezed states of light into the interferometer dark port to reduce quantum noise and improve sensitivity beyond the SQL \cite{caves_quantum-mechanical_1981}. On the experimental side, while the shot-noise has been observed in optical interferometers since the 1980s \cite{loudon_quantum_1983}, the radiation-pressure noise has however proven very challenging to observe experimentally, as it competes with thermal noise effects which usually dominate for micro-scale mechanical resonators, even at cryogenic temperatures. \\ 

Experimental \textit{tabletop} optomechanics then truly kicked off in the mid 1990s, when stable laser sources, low-loss optical coatings, quantum-limited detection setups and high-Q mm-scale mechanical resonators became available. The idea at the time was to take advantage of the smaller mass of the resonators (compared to the kg-scale of GW interferometer mirrors) to (actually, not so) easily demonstrate the effect of quantum fluctuations of radiation pressure upon the motion of the moving mirror. Notable milestones include the demonstration of radiation-pressure cooling of mechanical resonators in 1999 \cite{cohadon_cooling_1999}, ground state cooling of a mechanical resonator in 2006 \cite{arcizet2006,Gigan2006}, the observation of strong optomechanical coupling in 2009 \cite{groblacher2009}, and the direct observation of radiation-pressure noise in a micro-optomechanical system in 2013 \cite{regal2013}. \\ 

Present research in quantum optomechanics now focus on broad range of topics, that we can loosely put in four main categories: (i) the study of quantum limits in position measurements; (ii) the development of practical optomechanical sensors; (iii) the investigation of the fundamental quantum physics of the mechanical degree of freedom; and (iv) applications to quantum information. These topics all take advantage of the versatility of optomechanical systems, than can be coupled to a large spectrum of systems, such as laser and microwave light, electric and magnetic fields, two-level systems (spins, color centers...), acceleration and forces, etc. 

\section*{Relevance of this work}
\addcontentsline{toc}{section}{Relevance of this work} % ensures it appears in the main TOC
This thesis belongs to the first category, where we focus on the use and manipulation of squeezed light to improve the measurement sensitivity beyond the SQL. We do so by leveraging two main experimental ingredients:
\begin{itemize}
    \item \textbf{Frequency-dependent squeezed light} as a sub-SQL probe (Chap. II and V) 
    \item \textbf{Membrane based optomechanical systems} as state-of-the-art tunable optomechanical platforms. (Chap. II and IV)
\end{itemize}

\noindent \textbf{Frequency-dependent squeezed light:} Squeezed states of light are non-classical states where the quantum fluctuations of one quadrature (amplitude or phase) are reduced below the vacuum level, at the expense of increased fluctuations in the orthogonal quadrature, by virtue of the Heisenberg inequality. Experimentally demonstrated in 1985 \cite{slusher_observation_1985}, squeezed light has since then found applications in quantum information, quantum metrology and fundamental tests of quantum mechanics \cite{}. In the context of optomechanical measurements, squeezed light can be used to reduce quantum noise and improve sensitivity beyond the SQL \cite{caves_quantum-mechanical_1981}. However, due to the trade-off between shot-noise and radiation-pressure noise, frequency-independent squeezed light can only improve sensitivity in a limited frequency band. To achieve broadband sensitivity improvement, it is necessary to use frequency-dependent squeezed light, where the squeezing angle rotates as a function of frequency \cite{kimble_conversion_2001}, and where the frequency range over which the squeezing angle rotates is to be tuned to a specific mechanical resonator, from kg-scale GWI mirrors to ng-scale micro-resonators. The implementation of frequency-dependent squeezing has been a major milestone in the field, with recent demonstrations in gravitational wave interferometers such as Advanced LIGO and Advanced Virgo \cite{tse_quantum-enhanced_2019, acernese_enhanced_2019}. \\

Tabletop frequency dependent squeezing, however, remains a fairly unexplored territory, with only a handful of experimental demonstrations reported in the literature \cite{vahlbruch2010, khalaidovski2012, jaekel2020}, mainly focusing on the GWI applications. 

\noindent \textbf{Membrane based optomechanical systems:} Membrane-based optomechanical systems consist of a thin dielectric membrane placed inside an optical cavity, forming a so-called "membrane-in-the-middle" (MIM) or "membrane-at-the-end" (MATE) configuration \cite{thompson_strong_2008, jacquet_membrane-at-the-end_2020}. Introduced in 2008, the MIM architecture has become a popular platform for optomechanics. The smart workaround of using a thin membrane as the mechanical resonator, rather than a moving end-mirror, enables the use of high-finesse cavities with low mechanical loss resonators embedded within, hence splitting the optical and mechanical design constraints. This has led to numerous experimental demonstrations, including ground-state cooling of mechanical resonators \cite{teufel_sideband_2011, chan_laser_2011}, strong optomechanical coupling \cite{groblacher2009}, quantum state transfer between light and mechanics \cite{palomaki_coherent_2013}, and quantum squeezing of mechanical motion \cite{wollman_quantum_2015, pirkkalainen_squeezing_2015}. \\ 

The MATE configuration, however, remains less explored compared to the MIM configuration, with only a handful of experimental realizations reported in the literature \cite{jacquet_membrane-at-the-end_2020, xu_cavity_2021}. The MATE/MIM geometries display appealing features, such as a large linear coupling range, quadratic points, and greater tunability than standard moving-mirror cavities, making them promising candidates for future optomechanical experiments.



\section*{Thesis outline}
\addcontentsline{toc}{section}{Thesis outline} % ensures it appears in the main TOC
 


\noindent \textbf{Chapter I} presents the theoretical background required to motivate and discuss the experiments reported in this manuscript: the optical field (classical spatial-mode description, classical modulations used for sensing and locking, and quantum treatment), Fabry–Pérot cavities as the workhorse of the experimental platforms, and the main detection schemes. \\ 

\noindent \textbf{Chapter II} introduces squeezed light and the mechanical response of a mirror to radiation-pressure forces, establishing the Standard Quantum Limit and motivating squeezed-light injection to improve displacement sensitivity; both QSN and QRPN are discussed, together with numerical tools used to compute noise spectra in realistic situations. \\ 

\noindent \textbf{Chapter III} provides an overview of the experimental methods developed throughout my PhD work, with emphasis on locking and detection techniques largely implemented within the PyRPL architecture. \\

\noindent \textbf{Chapter IV} focuses on the membrane-at-the-end (MATE) cavity geometry used to realize a moving-mirror cavity, including the motivation for this approach, the experimental characterization of early prototypes, and design considerations for a cryogenic MATE cavity incorporating a SiN membrane resonator. \\ 

\noindent \textbf{Chapter V} describes the development, upgrade and characterization of the squeezed-light source, detailing its layout and operation, presenting preliminary squeezing results, and briefly outlining the work carried out in 2023 on the Advanced Virgo squeezing filter cavity.  \\

\noindent The manuscript concludes with a summary of the experimental results and perspectives toward a broadband demonstration of sub-SQL measurements.