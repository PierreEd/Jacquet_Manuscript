\chapter*{Conclusion}\label{chap:conclusion}
\addcontentsline{toc}{chapter}{Conclusion} % ensures it appears in the main TOC

We now reflect on the work presented in this thesis, summarizing the main results and discussing future outlooks.

\section*{Summary of Work}
\addcontentsline{toc}{section}{Summary of Work} % ensures it appears in the main TOC

\noindent \textbf{In Chap. I and II}, we reviewed the theoretical background of cavity optomechanics and squeezed light generation. Using a formal two-photon framework, we derived the key equations governing optomechanical interactions and the generation of squeezed states of light via parametric down-conversion in a nonlinear crystal, and derived useful expressions for the conception of tabletop filter cavities. We also introduced three-mirror cavities as promising candidates for both optomechanics and filter cavity applications. \\

\noindent \textbf{In Chap. III}, we presented the versatility of PyRPL as an affordable and flexible solution for the implementation of complex digital control systems in quantum optics experiments. We detailed the design and implementation of a variety of digital locks, with a focus on their practical implementation in quantum optics experiments. \\

\noindent \textbf{In Chap. IV}, we presented the MATE optomechanical cavity used in this thesis, detailing its design, fabrication, and characterization. We showed that the cavity exhibits tunability regarding its optomechanical couplings and optical properties, that it can implement mechanical ringdowns, and that it exhibits photothermal bistability. This confirms MATE systems as promising candidates for future optomechanics experiments, as well as filter cavity applications, owing to the wide range of physical effects that can be studied in such systems. We then presented the steps taken towards the fabrication of a novel fibered MATE cavity, which would provide a compact and robust platform for cryogenic optomechanics experiments. \\

\noindent \textbf{In Chap. V}, we presented the upgrades of our squeezed light source, including the latest frequency-independent bright squeezing measurement results. We showed the flexibility of the setup, allowing us to switch (fairly) easily between two configurations without major realignments. We measured bright squeezing at modest powers of $\sim 10\,\mu$W, with levels of about 1.5~dB at 10~MHz, limited by the pump noise at lower frequencies. While the power would be sufficient for optomechanics experiments, further pump noise mitigation is required to reach higher squeezing levels. We also presented some results on the Virgo filter cavity, and discussed the feasibility of implementing bichromatic locking schemes for tabletop filter cavities. 

\section*{Outlooks}
\addcontentsline{toc}{section}{Outlooks} % ensures it appears in the main TOC

\noindent \textbf{Short-term prospects:} for the MATE cavity, the next direct step involves characterizing the laser-machined LMA fibers in order to make the fibered MATE cavity input coupler. This involves measuring their mode field diameter and radii of curvature to select the best candidates for the fibered cavity. The next crucial step is the development of fiber holders compatible with the LMA facilities, to bring their expertise in low-loss optical coatings to the fibered MATE cavity project. Once the fibered cavity is assembled, its optical and optomechanical properties will need to be characterized, following the same steps as for the free-space MATE cavity. For the squeezed light source, the next step involves mitigating the pump noise at low frequencies, by implementing a pump filter cavity. Once higher squeezing levels are reached, the next step is the making of dual coatings for the tabletop filter cavity (LMA), to first implement a \textit{naive} locking scheme as seen in Chap. V, and then progress towards a bichromatic locking scheme if necessary. \\

\noindent \textbf{Long-term prospects:} once the tabletop frequency-dependent squeezing is implemented, the next step involves integrating it with the MATE optomechanical cavity. This will first require efficient injection in an LMA fiber, and careful characterization of losses along the optical path to estimate the injected squeezing levels. Once done, the two experiments need to be connected together by means of fiber splices, which also need to be carefully characterized to minimize losses. Finally, once frequency-dependent squeezing is injected in the MATE cavity, a first goal is to study the MATE physics at ambient temperature with non-classical light. This could include studying squeezed-light-assisted quadratic couplings, feedback cooling, or optomechanical bistability. The final goal would be to mount the fibered MATE cavity on a cryostat, and study quantum optomechanics at low temperatures with frequency-dependent squeezed light. We once again stress that both platforms were designed to provide a versatile testbed for a variety of optomechanics experiments (frequency-dependent or independent, vacuum or bright squeezing, linear or quadratic, dispersive or dissipative couplings), and that many other research directions could be explored with these setups. Finally, phononic crystals could be integrated to the MATE cavity to further enhance the mechanical quality factors and boost the optomechanical couplings and cooperativity. \\

\noindent \textbf{Final remarks:} overall, this thesis work has laid the groundwork for the implementation of frequency-dependent squeezed light in membrane-based optomechanical systems. The developed experimental tools and theoretical models provide a solid foundation for the future of this project, giving access to a versatile platform for a broad range of quantum optomechanics experiments with non-classical light. This has the potential to significantly advance our understanding of quantum optomechanics and its applications in precision measurements.\\
