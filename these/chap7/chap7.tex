\chapter*{Conclusion}\label{chap:conclusion}
\addcontentsline{toc}{chapter}{Conclusion} % ensures it appears in the main TOC
\mtcaddchapter
\markboth{CONCLUSION}{CONCLUSION} % <-- fixes header marks

We now reflect on the work presented in this thesis, summarizing the main results and discussing future outlooks.

\section*{Summary of Work}
\addcontentsline{toc}{section}{Summary of Work} % ensures it appears in the main TOC

\noindent \textbf{In Chap. I and II}, we reviewed the theoretical background of cavity optomechanics and squeezed light generation. Using a formal two-photon framework and its graphical representation, we derived the key equations governing classical modulations, optomechanical interactions and the generation of squeezed states of light via parametric down-conversion in a nonlinear cavity. We then derived useful expressions for the conception of tabletop filter cavities. We also introduced three-mirror cavities as promising candidates for both optomechanics and filter cavity applications, owing to their tunability regarding key experimental parameters (couplings, finesse, reflection/transmission coefficients). \\ 

\noindent \textbf{In Chap. III}, we presented the versatility of PyRPL as an affordable and flexible solution for the implementation of complex digital control systems in quantum optics experiments. We detailed the design and implementation of a variety of digital locks, with a focus on their practical implementation/wiring in quantum optics experiments. Most of the locks presented in Chap. IV and V were implemented using PyRPL, showcasing its adaptability to different experimental needs. Although the RedPitaya performances are eventually limited by various noise sources (power supply, digitization), we show that it achieves satisfactory results for most locking tasks in quantum optics experiments, with the added benefit of easy reconfigurability and remote operation. The next generation of RedPitaya boards (Gen2) also promises improved performances, requiring an adaptation of the PyRPL software to the new hardware architecture. \\

\noindent \textbf{In Chap. IV}, we presented the MATE optomechanical cavity used in this thesis, detailing its design, fabrication, and characterization. We first showed that the MATE cavities features tunable reflection/transmission and finesse. As seen in Cavity I, the finesse could be tuned from about 6000 to 10 000, which translates to tunable optical linewidths in the range 0.5 - 1 MHz, potentially useful for filter cavity applications (replacing the membrane by a plane mirror). We then turned to the optomechanical characterization of the MATE cavity, showing both tunable linear and quadratic optomechanical couplings, with raw linear couplings in range 0 - 100 GHz/$\mu$m, and quadratic couplings up to $\pm$ 500 GHz/$\mu$m$^2$. Although couplings are modest for a centimetric cavity, they can be significantly enhanced by reducing the cavity length down to the sub-millimeter range using fibered cavities. Locking the cavity, we identified the mechanical modes of the membrane, and used these to implement mechanical ringdowns, yielding modest quality factors up to $10^4$ at room temperature. These Q factors are thought to be limited by the added losses of the flexure tuning mechanism. Finally, we observed photothermal bistability in the MATE cavity, arising from absorption-induced heating of the membrane when shining a powerful (> 10 mW) input IR beam. This effect could be further studied in the future. We then presented the design considerations of a micrometric fibered cavity and the steps taken towards its fabrication. We notably presented the CO$_2$ laser machining setup used to shape the fiber tips into concave mirrors, along with some of the fabricated profiles. We discussed the challenges ahead to realize the fibered MATE cavity, including the fabrication of low-loss fiber holders compatible with the LMA coating facilities. \\

\noindent \textbf{In Chap. V}, we presented the upgrades of our squeezed light source, including the latest frequency-independent bright squeezing measurement results. We showed the flexibility of the setup, allowing us to switch (fairly) easily between two configurations without major realignments. In the first configuration, we measured bright squeezing at powers of $\sim 10\,\mu$W, with levels of about 1.5~dB at 10~MHz, limited by the pump noise at lower frequencies. In the second configuration, 3 dB of vacuum squeezing were measured down to the MHz range (see M. Croquette's PhD), which is still unsatisfactory for MHz mechanical modes. While the bright squeezing power would be sufficient for optomechanics experiments, further pump noise mitigation is required to reach higher squeezing levels, and be limited by quantum noises only. We also presented some results on the Virgo filter cavity, and discussed the feasibility of implementing bichromatic locking schemes for tabletop filter cavities. We finally discussed the design of a new tabletop filter cavity, to be implemented using large mode area fibers and dual-wavelength coatings. \\

\section*{Outlooks}
\addcontentsline{toc}{section}{Outlooks} % ensures it appears in the main TOC

\noindent \textbf{Short-term prospects:} For the MATE cavity, the next direct step involves characterizing the laser-machined LMA fibers in order to make the fibered MATE cavity input coupler. This involves measuring their mode field diameter and radii of curvature to select the best candidates for the fibered cavity. The next crucial step is the development of fiber holders compatible with the LMA facilities, to bring their expertise in low-loss optical coatings to the fibered MATE cavity project. Once the fibered cavity is assembled, its optical and optomechanical properties will need to be characterized, following the same steps as for the free-space MATE cavity. For the squeezed light source, the next step involves mitigating the pump noise at low frequencies, by implementing a green filter cavity. Once higher squeezing levels are reached, the next step is the making of dual coatings for the tabletop filter cavity (LMA), to first implement a \textit{naive} locking scheme as seen in Chap. V, and then progress towards a bichromatic locking scheme if necessary. \\

\noindent \textbf{Long-term prospects:} Once the tabletop frequency-dependent squeezing is implemented, the next step involves integrating it with the MATE optomechanical cavity. This will first require efficient injection in an LMA fiber, and careful characterization of losses along the optical path to estimate the injected squeezing levels. Once done, the two experiments need to be connected together by means of fiber splices, which also need to be carefully characterized to minimize losses. Finally, once frequency-dependent squeezing is injected in the MATE cavity, a first goal is to study the MATE physics at ambient temperature with non-classical light. This could include studying squeezed-light-assisted quadratic couplings, feedback cooling, or optomechanical bistability. The final goal would be to mount the fibered MATE cavity on a cryostat, and study quantum optomechanics at cryogenic temperatures with frequency-dependent squeezed light. We once again stress that both platforms were designed to provide a versatile testbed for a variety of optomechanics experiments (frequency-dependent or independent, vacuum or bright squeezing, linear or quadratic, dispersive or dissipative couplings), and that many other research directions could be explored with these setups. Finally, a plethora of mechanical resonators could be integrated to the MATE cavity to further enhance the mechanical quality factors and boost the optomechanical couplings and cooperativity. Notably, the team developed expertise in fabricating high-stress SiN membranes with integrated phononic shields i.e. a phononic crystal \cite{Ivanov2020}, which could be directly integrated in the MATE cavity to boost the mechanical Q factors by orders of magnitude. Other resonators such as trampolines, ribbons, or photonic crystal membranes could also be considered in the future, all made from high-stress silicon nitride thin films using the same lithography techniques. \\

\noindent \textbf{Final remarks:} Overall, this thesis work has laid the groundwork for the implementation of frequency-dependent squeezed light in membrane-based optomechanical systems. The developed experimental tools and theoretical models provide a solid foundation for the future of this project, giving access to a versatile platform for a broad range of quantum optomechanics experiments with non-classical light. This has the potential to significantly advance our understanding of quantum optomechanics and its applications in precision measurements.\\
